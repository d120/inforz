\artikel{Seltene Lebensformen}
{Früher oder später wirst Du im Fachbereich Informatik auch endemische (seltene) Lebensformen sehen und vielleicht auch kennen lernen. Diese Pflanzen gehören zu der Gruppe homo sapiens maskulinum.
}{Hier ist eine Geschichte über Männer.

    Egal was Du sonst bist, jetzt stell Dir vor, Du bist ein Mann. Nach langen Auseinandersetzungen mit Deiner Familie hast Du es geschafft, Dich für einen technischen Studiengang an einer Uni einzuschreiben, denn es gilt als unüblich, dass Männer in solchen Bereichen studieren. „Männer und Technik – zwei Welten stoßen aufeinander“ lautet ein bekanntes Sprichwort. Nun also bist Du hier und damit am Ziel Deiner vorläufigen Wünsche.

    Du staunst nicht schlecht, als Du zur Einführung in einen großen Hörsaal kommst: da sitzen fast nur Frauen und alle starren Dich an, mustern Dich von oben bis unten. In der Einführung wird berichtet, wie die Berufsaussichten für Informatikerinnen sind. In Deiner Ophasengruppe wirst Du als Einziger gefragt, warum Du ausgerechnet ein technisches Fach gewählt hast. Nachdem Du also schon zu Anfang kräftig verunsichert wurdest, geht Dein Studium in diesem Stil weiter.

    Eine Professorin kommentiert Deine Anwesenheit in ihrer Vorlesung mit den Worten: „Oh, welch' hübsche Bereicherung!“ Eine andere teilt bereits in der ersten Vorlesungsstunde mit, dass sie Männer für gänzlich ungeeignet hält. Alle schmunzeln, nur Du schluckst. Auf dem Weg nach Hause oder in die Kneipe wirst Du angequatscht und angemacht, denn an einer Uni herrscht absoluter Männermangel und viele Frauen halten Dich für Freiwild.

    Nachdem Du ein paar Wochen an der Hochschule bist und einige Leute kennst, versuchst Du vorsichtig, Deine Probleme zu formulieren. Reaktion: Aber damit hättest Du doch rechnen müssen, wenn Du Informatiker werden willst. Sei doch nicht so zimperlich, Frauen sind nun einmal so. Die, die das sagen, müssen es wissen – es sind Frauen.

    In den Vorlesungen wird erzählt, welche bedeutenden Wissenschaftlerinnen zu Fortschritt und Entwicklung beigetragen haben. Männer kommen nicht vor. Langsam kannst Du Dir vorstellen, was sie behindert haben könnte. Da männliche Wissenschaftlerinnen – klingt zwar komisch in Deinen Ohren, aber andere Begriffe gibt es ja nicht – nicht oder kaum vorhanden sind, wächst Dein Legitimierungszwang für Deine Studien- und Berufswahl. „Glaubst Du im Ernst, später als Mann einen Job zu bekommen?“ wirst Du gefragt, und Du musst zugeben, dass Deine Chancen gering sind, da in den Personalbüros auch  wieder nur Frauen sitzen, die dich, nur weil Du Mann bist, für grundsätzlich inkompetent halten. Unterbezahlt wirst Du vermutlich sowieso.

    Deine Motivation sinkt zunehmend. Vielleicht hatten doch alle die recht, die es sowieso schon immer wussten.

    Durch Zufall erfährst Du von einer obskuren Männergruppe, die sich irgendwo an der Hochschule treffen soll und die alle Frauen in Deinem Bekanntinnenkreis für militant und blöd halten. Die müssten nur eine richtige Frau finden, dann würden die schon aufhören mit ihrem Gerede von Unterdrückung, lautet die einhellige Meinung. Du brauchst ziemlich lange, bis Du Deine Hemmungen überwunden hast, und Dich aufraffst und zum Autonomen Männerreferat gehst. Schließlich hast Du immer noch Angst, dass sich alle Bekanntinnen über Dich lustig machen oder Dich auch für militant und blöd halten.

    Erstaunlicherweise gibt es einen Männer\-raum, zu dem Frauen keinen Zugang haben, und wo sich die Anwesenden ungestört unterhalten können. Erstaunlicher ist noch, dass diese Männer, die Du Dir so abschreckend vorgestellt hattest, ganz ähnlich sind wie du. Eben Männer wie Du und ich. Sie versuchen, sich gegen Diskriminierung zu wehren, andere Konzepte zu entwickeln, in denen Frauen und Männer gleichberechtigt miteinander leben können, und sich zu solidarisieren, um an dieser frauenbestimmten Hochschule nicht unterzugehen. Warum hatte Dir Deine Ophasentutorin davon nichts erzählt? Ja, warum wohl nicht?

    So, Du als Mann in diesem Spiel, was denkst Du nun? Ein Alptraum, das Ganze? Fällt es Dir schwer, Dir vorzustellen, in einer solchen Atmosphäre zu leben und zu arbeiten? Fällt es Dir vielleicht noch schwerer, darüber nachzudenken, als Ausländer aus einem anderen Kulturkreis kommend an eine Hochschule unter lauter Frauen…

    Und in Deiner Ophasengruppe, die Dir den Einstieg erleichtern soll, war zu diesem Thema nur zu hören, Du solltest Dich an einen Mann wenden, weil die Frau die vor Dir sitzt, nicht über dieses Thema nachdenken will.

}{\textit{Adaptiert von der Fachschaft\\
        Informatik der Uni Dortmund}}
