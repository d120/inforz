\artikel{Wohnen in Darmstadt}
{Jeden Tag zur Uni zu pendeln ist für viele Studierende keine Option, darum sollte man sich, wenn man von weiter her kommt, für die Dauer des Studiums eine Bleibe vor Ort suchen.
}
{Mit dem Beginn des Studiums wird sich ein guter Teil Deines Lebens in die Uni verlagern. Dort finden alle Deine Veranstaltungen statt, dort triffst Du Deine Kommiliton*innen, lernst und verbringst vielleicht sogar einen Teil Deiner Freizeit bei den verschiedenen Angeboten, die sich in Darmstadt (zum Teil sogar ebenfalls an der Uni) bieten.

    Damit nicht zu viel Zeit für die Wege zu und von der Uni draufgeht, ist es sinnvoll, nach Darmstadt oder in die nähere Umgebung zu ziehen. Dadurch hat man auch weniger Stress, wenn man frühe Vorlesungen hat oder abends bzw. am Wochenende noch mit Kommiliton*innen feiern gehen möchte. Nun ist Darmstadt aber dummerweise ein teures Pflaster für Studierende, was die Lebenshaltungskosten angeht. Insbesondere die Mietpreise können sehr happig sein.

    Wer sich rechtzeitig bewirbt, hat beispielsweise recht gute Chancen auf einen Platz in einem der mittlerweile 13 Studierendenwohnheime in Darmstadt, die vom Studierendenwerk unterhalten werden \footnote{\url{http://studierendenwerkdarmstadt.de/wohnservice/}} und die meist preisgünstigste Wohnmöglichkeit darstellen. Die Zimmer sind aber überwiegend recht klein und in den meisten Fällen teil man sich Küche und Bad mit mehreren anderen Zimmern in einer Flur- oder Wohngemeinschaft. Außerdem ist die Mietdauer in der Regel auf drei Semester begrenzt, idealerweise nutzt man einen Wohnheimplatz also dafür, um neben dem Studium in Ruhe vor Ort eine dauerhaftere Bleibe suchen zu können. Die Warmmietpreise für Wohnheimzimmer reichen von ca. 180 Euro in Flur- bzw. größeren Wohngemeinschaften bis hin zu knapp 460 Euro für Einzelapartments. Dabei ist in den meisten Fällen sogar bereits Strom und Internet inklusive. Für Wohnheimzimmer kannst Du Dich online bewerben \footnote{\url{https://service.studierendenwerkdarmstadt.de/tl1/}}, in manchen Fällen hilft es aber auch, persönlich bei der Wohnraumverwaltung vorbeizuschauen: In manchen Wohnheimen gibt es nämlich Wohngemeinschaften mit Selbstbelegung, in welchen die bereits dort lebenden Mieter*innen entscheiden, wer mit einziehen darf. Die Wohnraumverwaltung weiß darüber Bescheid, in welchen derartigen WGs noch Plätze frei sind und kann bei persönlicher Bewerbung manchmal direkt Kontakte vermitteln.

    Alternativ kann man sich auch eine private WG suchen oder zusammen mit Freund*innen oder Kommiliton*innen eine WG gründen. WG-Zimmer sind selbst in Darmstadt relativ gut zu finden, da es eine Menge einschlägiger Internetportale (u.a. \footnote{\url{https://www.wg-gesucht.de}}, \footnote{\url{https://www.studenten-wg.de}}) gibt, über die viele WGs Mitbewohner*innen suchen. Aber auch offline kann man fündig werden, denn in der Uni gibt es an vielen Orten öffentliche schwarze Bretter, an denen auch oft Aushänge zu WG-Zimmern zu finden sind. Die meisten Zimmer in Wohngemeinschaften sind auch für Studierende gut finanzierbar und bewegen sich in ähnlichem Rahmen wie Wohnheimzimmer.

    Worauf man bei WG-Angeboten des Öfteren mal stößt, sind Studentenverbindungen. Meist tragen diese Vereine Selbstbezeichnungen wie Corps, Burschen-, Turner- oder Sängerschaften und sind anders organisiert als einfache WGs. Verbindungen bestehen üblicherweise schon eine ganze Zeit lang und können über Ehemalige, die den Gruppierungen immer noch verbunden sind, beispielsweise Kontakte in die höheren Riegen verschiedener Industriesparten bieten. Doch Vorsicht: Wer sich einer Verbindung anschließt, muss sich den (bisweilen archaischen) Lebensgewohnheiten und Traditionen der entsprechenden Verbindung anpassen. Hierzu gehören oftmals verpflichtende Fechtkämpfe (das sogenannte "Schlagen") oder andere Aufnahmerituale. Auch Frauen werden von vielen Verbindungen nicht als Mitglieder akzeptiert.
    Bevor man also in eine Verbindung eintritt, sollte man sich also erst einmal mit ihren Gepflogenheiten auseinandersetzen.

    Wer meint, WGs seien ihm zu gesellig oder wer zumindest lieber ein Bad und eine Küche für sich allein hat, kann sich auch ein privates Einzelapartment oder eine kleine Wohnung suchen. Die Preise dafür liegen aber in aller Regel deutlich über denen eines WG- oder Wohnheimzimmers und fangen üblicherweise erst bei 300 Euro im Monat an. Auch kleine Wohnungen und Einzelzimmer speziell für Studierende werden oftmals im Internet angeboten, eine hier noch relativ ergiebige Angebotsquelle sind lokale bzw. regionale Zeitungen (wie \footnote{\url{http://www.echo-online.de}}). Oft kann man auch gute Angebote erhalten, wenn man mal ca. 25 Euro investiert und eine Wohnungssuchanzeige schaltet. Das ist zumindest deutlich günstiger als das Einschalten eines Wohnungsmaklers, der meistens mehrere Monatsmieten an Provision verlangt, aber für diejenigen, die sich das leisten können, recht zuverlässig und schnell eine Bleibe vermittelt.

    Übrigens kann es auch helfen, nicht nur in Darmstadts Kernstadt nach Wohnungen zu suchen. Die meisten Stadtteile (Bessungen, Eberstadt, Arheilgen, Kranichstein), sowie auch die Nachbargemeinde Griesheim sind ausgezeichnet per Straßenbahn an die Innenstadt angebunden und auch aus Pfungstadt, Weiterstadt und Erzhausen hat man noch eine gute Verbindung nach Darmstadt. Etwas weiter entfernt liegen Dieburg, Langen, Bensheim, Frankfurt und Heppenheim. Auch von diesen Orten aus ist Darmstadt noch relativ gut mit dem Zug zu erreichen, die Fahrtzeiten werden dann allerdings doch etwas länger.
}
{}


\newpage
