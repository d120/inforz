\artikel{Erfahrungsberichte von Studierenden}
{Damit Du einen besseren Einblick in das Studentenleben bekommst, haben wir einige Studierenden gefragt, wie sie zur Informatik gekommen sind und ihr Alltag aussieht.%
}
{\textbf{Chris, 1. Semester}

    Als ich noch in der Schule war und darüber nachgedacht habe, was ich nach dem Abitur machen möchte, kam mir natürlich auch Informatik in den Sinn. Bereits in der Schule hat man ja die Gelegenheit gehabt anhand der verschiedenen Fächer abzuschätzen in welche Richtung man später einmal gehen möchte. In meinem Fall kamen Politik und Wirtschaft, Mathematik und Informatik in die engere Auswahl. Von Seiten der Schule aus hatten wir die Gelegenheit, die Hochschul- und Berufsinformationstage (kurz hobit) im Darmstadtium zu besuchen. Dort habe ich dann relativ schnell gemerkt, dass Politikwissenschaft nicht wirklich das war, was ich mir darunter vorgestellt habe. Schwierig wurde es dann bei Informatik und Mathe, denn beides klang an sich relativ interessant. Mein Entschluss für Informatik fiel letztendlich nur durch den Gedanken, dass ich in Informatik auch Mathe haben werde aber in Mathe nicht unbedingt Informatik. Das sollte sich später als beste Entscheidung meiner noch jungen Studienlaufbahn herausstellen. Nachdem ich mich also für Informatik entschieden hatte ging es mit der Überlegung weiter an welche Universität bzw. Hochschule ich gehe. Für die TU Darmstadt habe ich mich eigentlich weniger wegen der Uni an sich entschieden, sondern eher externe Gründe (Nähe zum Wohnsitz, etc.) gehabt. Ich habe mich dann eingeschrieben und bin dann auf die Ophase der Fachschaft Informatik aufmerksam geworden. Anfangs habe ich darüber nachgedacht, ob es sich überhaupt lohnt die Ophase zu besuchen oder ob ich nicht eher "`mal eben"' im Internet suche was ich wissen muss. Letztendlich habe ich mich aber entschlossen doch hin zu gehen und ich habe es nicht bereut. Nicht nur, dass es der ideale Einstieg war, es war auch eine super Gelegenheit die zukünftigen Kommilitonen bereits vor dem Vorlesungsbeginn kennen zu lernen. Es gab auch viele Veranstaltungen, die genau darauf ausgelegt waren wie bspw. Kneipentouren oder Spieleabende. Als dann mit Vorlesungsbeginn die eigentliche Arbeit losging musste ich schnell feststellen, dass mein einstiges Lieblingsfach Mathe leider gar nichts mehr mit dem Mathe aus der Schule zu tun hat. Während ich in den anderen beiden Modulen super zurecht kam bereitete mir Mathe einige Schwierigkeiten und man muss eine Menge Zeit hereinstecken, aber irgendwann hat man auch das geschafft. Letztendlich bereute ich meine Wahl auch nicht, weder die Universität, noch den Studiengang. Im Gegensatz zur Schule hat das Studium neben der viel größeren Flexibilität den Vorteil, dass man thematisch überwiegend mit Themen zu tun hat, welche den persönlichen Interessen entsprechen. Ich habe mich schon nach wenigen Wochen super eingelebt, freue mich jeden Tag aufs neue herzukommen und habe auch direkt die aktive Fachschaft für mich entdeckt. Das Studentenleben ist echt super und man lernt durch die Eigenverantwortlichkeit neben dem Inhaltlichen Stoff auch Arbeit und Freizeit (die auch bei Studenten nicht zu kurz kommt) unter einen Hut zu bringen.\\

    \textbf{Björn, 1. Semester}

    Warum studieren? Warum gerade Informatik? Warum an der TU Darmstadt? Die Antworten auf diese Fragen lassen sich wohl erst richtig beantworten, wenn man mit dem Studium angefangen und die Konsequenzen kennengelernt hat. Zur ersten Frage hat wohl jeder eine andere Antwort. Meine ist diese: Ich will auf jeden Fall etwas Akademisches machen. Ich hatte aus persönlichen Erfahrungen (eigenes Umfeld, Praktika,...) gemerkt, dass ich nicht in die Richtung Handwerk o.ä. gehen will. Außerdem war ich schon immer wissenschaftlich Interessiert. Daher war für mich persönlich klar, dass ich nach dem Abitur studiere.

    Die zweite Frage steht symbolisch für die Frage, was möchte/kann/soll ich studieren. Ich hatte zunächst (etwa bis zur 9 Klasse) vor, mich mit Chemie zu beschäftigen. Nachdem ich ein Praktikum als Chemielaborant gemacht hatte, hatte ich jedoch langsam gemerkt, dass das nichts für mich ist. Zur Informatik bin ich so richtig erst in der 11. Klasse gekommen. Sowohl der Unterricht in der Schule, als auch die private Beschäftigung mit dem Thema (eigenes Programmieren von "`Spielen"' und dadurch das Erlernen / Vertrautmachen mit C und Java) haben mich dann bestärkt, mich für Informatik zu entscheiden.

    Nun bleibt noch die Frage: Wieso Darmstadt? Wieso TU Darmstadt? Nachdem ich mich für Informatik entschieden habe, begann die Suche nach einer Uni. Da ich aus Frankfurt komme, wäre Darmstadt nicht die erste logische Wahl. Jedoch überwog dann der Ruf der Uni und ich entschied mich für Darmstadt.

    Nachdem dann alle Formalitäten erledigt waren, kam der erste Tag an der Uni. Da ich zuvor schon an einem Programmiervorkurs teilgenommen hatte, waren die Gebäude nicht ganz so unbekannt. In der Woche vor dem Beginn gab es die Ophase. Während dieser wurden alle Ersties in Kleingruppen aufgeteilt und bekamen in diesen den Unialltag näher gebracht. Zusätzlich gab es gemeinsame Veranstaltung.

    In der darauf folgenden Woche begannen dann auch schon die Vorlesungen. Dank der Ophase war es nicht ganz der kalte Sprung ins Wasser, den ich erwartet hatte. Zusätzlich hilft es, wenn man sich schon in der Ophase nicht kontaktscheu ist und neue Freundschaften knüpft, da jeder etwas weiß (wenn auch ohne Garantie) und man so häufig alle Fragen beantwortet bekommt.

    Abschließend kann man also sagen, dass jeder, der sich für wissenschaftliche Arbeit interessiert, an der Uni gut aufgehoben ist. Allerdings muss man auch bedenken, dass man an der Uni viel mehr selbst machen muss und dass das Tempo höher ist. Wen das nicht abschreckt, der sollte sich überlegen zu studieren.\\

    \textbf{Philipp, 7. Semester}

    Ich hatte eigentlich nie vor Informatik zu studieren. In der Schule war das Fach nicht gerade spannend, weswegen ich es anfangs 3 Jahre belegt hatte, dann aber in der Oberstufe wieder abgewählt habe. Da Informatik aber gute Berufsaussichten bot und mir keine bessere Alternative eingefallen ist, habe ich mich dann für ein Informatikstudium entschieden. Der Einstieg in das Studium war alles andere als einfach. Ich habe das Studium mit quasi keinen Vorkenntnissen begonnen, da die relevanteren Themen erst in der 12/13 in unserer Schule gelehrt wurden. Hinzu kommt, dass ich mich nicht so direkt mit dem Stoff des Studiums befasst hatte und eigentlich davon ausgegangen bin, nun keine Mathematik mehr machen zu müssen.

    Ich bemerkte besonders in den ersten Wochen, wie schnell Kommilitonen mit Aufgaben fertig wurden, während ich dafür eine gefühlte Ewigkeit gebraucht habe. Trotz guten Vorkenntnissen durch den Mathe LK, war jedoch das "`Mathe für Informatik"' etwas anderes als in der Schule. Keines der Fächer im ersten Semester war für mich demnach einfach. Ohne die Hilfe von anderen hätte ich das erste Semester wohl auch nicht überlebt. Trotzdem ist es mir dann irgendwie gelungen, die ersten Klausuren zu bestehen.

    Das Studium an sich wurde von Semester zu Semester spannender. Dies liegt vor allem am Regelplan der TU Darmstadt in Informatik. In den ersten Semestern werden Grundlagenveranstaltungen sowie Einführungsveranstaltungen in Vertiefungsrichtungen absolviert. Diese müssen auf jeden Fall belegt werden. Bei dem sogenannten Wahlpflichtbereich ist doch etwas anderes. Da man hier seine Veranstaltungen frei wählen kann, machen diese Fächer besonders Spaß. Auch das Bachelorpraktikum sowie die Bachelorarbeit haben sehr viel Spaß gemacht.

    Rückblickend kann man sagen, dass das Studium zwar nicht einfach, aber auf keinen Fall unmöglich zu absolvieren ist. Die Umstellung von Schüler zu Student muss einfach stattfinden. In der Schule war es gut möglich durch Nichtstun gute Noten zu erhalten, oder zumindest mit minimalem Aufwand. Geht man mit der gleichen Einstellung in dieses Studium, wird man wohl kaum eine Klausur bestehen. Somit braucht man für ein erfolgreiches Studium an der TU Darmstadt eigentlich nur die richtige Einstellung, da man in der Schule alles nachgetragen bekommt und zum Arbeiten quasi gedrängt wird und hier eben nicht. Genau das ist auch das Problem vieler Erstsemester, da erst mal der Arbeitsaufwand drastisch unterschätzt wird und diese Einsicht dann in manchen Fällen erst zu spät kommt. Auf der anderen Seite ist es meiner Meinung nach auch einfach, jede Prüfung zu bestehen, sofern man sich ausreichend und gezielt darauf vorbereitet. Das bedeutet nicht einmal, dass man keine Freizeit mehr hat. Im Vergleich zu einer 40-Stunden-Arbeitswoche ist das Studentenleben sehr entspannt. Ich zumindest habe in den letzten 3 Jahren unter dem Semester nie auch nur annähernd 40 Stunden mit dem Studium verbracht. Man muss aber auch in der Lage sein, über viele Wochen hinweg mal mehr als 40 Stunden die Woche zu arbeiten, eben genau dann, wenn die Klausuren näher rücken und man am besten nichts anderes macht als Lernen, Essen und evtl. noch Schlafen. Mit dieser Methode habe ich, teilweise auch mit ein wenig Glück, bis jetzt jede Prüfung im Erstversuch bestanden und das mit guten Noten.

    Da ich sowieso noch einen oder zwei Master machen werde, ist es aber auch relativ egal, dass meine Note für dieses Bachelorstudium wohl nicht auf 1,x kommen wird. Das ist auch etwas, das man beachten sollte, da die TU ja jeden eigenen Bachelorstudenten in den Master aufnimmt, egal mit welchem Schnitt. Ich würde wohl auf jeden Fall wieder Informatik studieren, da mir auch in den fast 7 Semestern, in denen ich nun hier in Darmstadt bin nichts eingefallen ist, was mir mehr Spaß macht. Außerdem gibt es im Master ja noch Anwendungsfächer, sodass ich auch in andere Bereiche, wie zum Beispiel Psychologie, eintauchen kann.

}
{}

\vfill
\bildmitunterschrift{../grafik/comics/compiling}{width=9cm}{}{xkcd.org}

\newpage
