\artikel{Selbststudium}
{Der Begriff des Selbststudiums wird Dir im Studium öfters vorgehalten werden. In der Tat steht das Selbststudium sogar als Form der Lehre in der Ordnung des Studiengangs. Was aber ist damit genau gemeint? Welche Verantwortungen kommen auf mich zu?
}{Auch wenn es oft heißt, das Bachelor- und Master-System sei verschult, liegt doch deutlich mehr Lernverantwortung bei einem selbst als noch in der Schule: Nur in den wenigsten Fächern bestehen Anwesenheitspflichten, in den meisten braucht man auch weder eine Vorlesung noch eine Übung besucht zu haben, um die Klausur mitschreiben zu dürfen. Insbesondere existiert damit auch kein Mensch, der Dir vorschreibt, was Du wann zu tun hast und kontrolliert, ob Du seinen Vorgaben gefolgt bist.
    Was zunächst sehr angenehm und locker anmutet, hat aber auch seine Schattenseite. Anders ausgedrückt ist es an der Uni nämlich fast jedem egal, wann Du welche Veranstaltung belegst, ob Du sie bestehst oder in den Prüfungen durchfällst. Es kümmert auch kaum jemanden, ob Du Dein Studium überhaupt abschließt oder allein schon, ob Du in der Uni anwesend bist oder nicht. Klingt drastisch, ist aber im Großen und Ganzen so.

    Was wir Dir verdeutlichen wollen ist, dass im Studium Du selbst hauptverantwortlich dafür bist, dass Du damit vorankommst. Dazu gehört vor allem, sich selbst zu motivieren (oder zu disziplinieren) und am sprichwörtlichen Ball zu bleiben. Wie gesagt, es verlangt niemand von dir, dass Du in die Vorlesung gehst, und Du wirst auch keinen Ärger bekommen, wenn Du lieber ausschläfst, anstatt um acht Uhr morgens eine Übung zu besuchen. So lange Du den Stoff der Veranstaltung zur Klausur beherrschst, ist es gleichgültig, auf welchem Weg Du ihn Dir angeeignet hast.

    Und genau das bedeutet Selbststudium: Du bist selbst dafür verantwortlich, Dir alles für Dein Studium notwendige Wissen anzueignen. Wie Du das tust, ist theoretisch nebensächlich – Hauptsache ist, dass Du es überhaupt tust.\\

    \textbf{Noch ein paar Worte zum Lernen}

    Praktisch dürfte es aber natürlich auch in Deinem Interesse sein, Dir den Studienstoff möglichst effizient anzueignen. Leider gibt es dafür kein allgemeines Patentrezept, da jeder Mensch auf andere Art und Weise zu maximalem Lernerfolg kommt. Darum ist es umso wichtiger, den Begriff des Selbststudiums auch auf eine andere Weise auszulegen, nämlich als Studium des eigenen Selbst. Das klingt vielleicht philosophischer als man in der Informatik erwarten mag, eine Lerntechnik zu finden (und weiterzuentwickeln), mit der man Erfolg hat, ist aber etwas, das alle Studierenden beschäftigt – insbesondere zu Beginn des Studiums.

    Dabei solltest Du auch offen für unkonventionelle Ansätze sein. Wenn Du beispielsweise merkst, dass Du in Vorlesungen ohnehin kaum aufpasst, dann spar Dir eben die Zeit – in den meisten Veranstaltungen werden zumindest die Vorlesungsfolien oder ein Skript online zur Verfügung gestellt, anhand derer Du auch lernen kannst, ohne in der Vorlesung körperlich anwesend zu sein. Einige Vorlesungen, insbesondere in den Grundlagenfächern, werden auch per Video aufgezeichnet und können so auch ohne Hörsaalfeeling (dafür aber mit Pause- und Wiederholungsfunktion) nachgeholt werden. Nur solltest Du aufpassen, die Inhalte nicht zu sehr schleifen zu lassen, wozu das Fernbleiben von einzelnen Veranstaltungsteilen schnell führen kann.

    Zu guter Letzt soll auch nicht unerwähnt bleiben, dass es im Studium nicht allein ums Lernen des Stoffes irgendwelcher Veranstaltungen geht. Das Studium, ganz besonders an der Universität, soll die Fähigkeit vermitteln, eigenständig wissenschaftlich arbeiten zu können. Wissenschaftliches Arbeiten ist aber nicht nur Methodik, sondern hängt auch stark mit der geistigen Einstellung dazu zusammen. Die Wissenschaft lebt schließlich vom kritischen Hinterfragen und dementsprechend solltest auch Du den gelehrten Stoff bisweilen kritisch reflektieren. Das beinhaltet beispielsweise auch eigenständige Recherche in anderen Quellen, beispielsweise im Internet oder in der entsprechenden Literatur. Bei Verständnisproblemen hilft es auch oft, einfach mal z.B. einem*einer Übungstutor*in oder dem*der Dozierenden Fragen zu stellen.
}
{}

\vfill
\bildmitunterschrift{../grafik/comics/manuals}{width=11cm}{}{xkcd.org}

\newpage
