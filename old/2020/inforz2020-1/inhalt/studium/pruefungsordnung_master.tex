\artikel{Die Ordnung des Studiengangs - Allgemeines}
{Die Ordnung Deines Studiengangs beschreibt, wie Dein Studium organisiert ist. Sie besteht aus den Ausführungsbestimmungen (AB) zu den allgemeinen Prüfungsbestimmungen (APB), dem Studien- und Prüfungsplan sowie dem Modulhandbuch.
}{
    Die folgenden Angaben sind wie immer ohne Gewähr. Verbindlich sind nur die offiziellen Versionen der Ordnung und der Allgemeinen Prüfungsbestimmungen. Verbindliche Informationen geben außerdem die (Fach-)Studienberatung, der*die Dekan*in, die*der Studiendekan*in und das Studienbüro.\\

    \noindent\textbf{Vorbemerkung}

    Zum erfolgreichen Abschluss des Masterstudiengang musst Du mindestens 120 Credit Points (CP) gemäß der Ordnung erbringen. Nach dem Abschluss des Masters erwirbst Du dann den akademischen Grad Master of Science (M. Sc.).\\

    \noindent\textbf{Studienziele}
    Dein Studium umfasst sowohl mathematisch"=naturwissenschaftliche als auch ingenieurwissenschaftliche Aspekte. Du sollst lernen, selbständig zu arbeiten. Dazu gehört die Fähigkeit, Problemlösungen zu finden und deren Auswirkungen und Konsequenzen abschätzen zu können, ebenso die Weiterentwicklung, Anpassung oder Verwertung dieser Lösungsansätze. Des Weiteren soll Dir Dein Studium einen Einblick in die Arbeits- und Berufswelt geben und Du sollst die Verantwortung und Stellung als Informatikerin oder Informatiker in der Gesellschaft kennen lernen.

    Um das alles zu erreichen, bedarf es unter anderem:
    \begin{itemize}
        \item einer Basis an wissenschaftlichen Methoden der Informatik und der Mathematik
        \item fachübergreifendem Denken
        \item der Kenntnis und Fähigkeit, methodisch komplexe Softwaresysteme zu realisieren
        \item kritischer Reflexion und Argumentation über Inhalte und Methoden der Informatik
        \item wissenschaftlichem Arbeiten mit dazugehörigem Vertrauen und Selbstständigkeit
        \item Kooperation, Kommunikation und Kreativität sowie Abstraktions- und Ordnungsvermögen
        \item der Bereitschaft zu gesellschaftlich verantwortlichem ingenieursmäßigem Handeln.
    \end{itemize}

    \noindent\textbf{Credit Points}
    Credit Points sind eine Aufwandsbewertung, um eine einheitliche Größe zum Vergleich des zeitlichen Umfangs mit anderen Veranstaltungen zu haben. Ein Credit Point entspricht etwa 30 Stunden Arbeit im Semester. Pro Semester soll ein*e Student*in ungefähr 30 Credit Points erwerben. Dies entspricht dem Aufwand einer 40-Stundenwoche bei einem Acht-Stunden-Tag.\\


    \noindent\textbf{Prüfungsplan}

    Um Dein Studium zu planen, ist es sinnvoll, einen Prüfungsplan zu erstellen. Im Prüfungsplan listest Du  die Module auf, die Du in Deinem Studium einbringen will. Der Plan soll sicherstellen, dass Deine Kurse die Bedingungen der Ordnung des Studiengangs erfüllen. Es gibt ein Tool des Fachbereiches, das beim Erstellen des Prüfungsplanes hilft. Dieses Tool wird bald, wenn die neuen Studiengänge modelliert sind, erneut zur Verfügung stehen, um bei der Planung des Studiums zu helfen. \footnotemark[1].\\


    \noindent\textbf{Masterarbeit}
    Die Masterarbeit stellt den "`krönenden Abschluss"' Deines Studiums dar. Sie hat 30 CP. Ein Thema für Deine Masterarbeit kannst Du z.B. durch Aushänge der Fachgebiete finden. Es ist auch möglich, selbst ein Thema vorzuschlagen. Dafür musst du aber eine*n Professor*in, der*die Dich bei der Arbeit betreut. Die Masterarbeit hat eine reine Bearbeitungszeit von 900 Stunden und soll in einem Zeitraum von sechs Monaten fertiggestellt werden. Das Ziel der Masterarbeit ist zu zeigen, dass Du in der Lage bist, ein Problem aus der Informatik selbstständig in vorgegebener Zeit zu bearbeiten und die Ergebnisse verständlich darzustellen. Neben der schriftlichen Arbeit gehört dazu auch eine Präsentation dieser Ergebnisse mit anschließender Befragung und Diskussion.\\


    \noindent\textbf{Studienleistungen}
    Studienleistungen kannst Du beliebig häufig wiederholen, bis Du sie bestanden hast. Seminare und Praktika z.B. sind solche Studienleistungen.\\


    \noindent\textbf{Fachprüfungen}
    Fachprüfungen unterscheiden sich von Studienleistungen in ihrer Wiederholbarkeit. Fachprüfungen können nur zweimal wiederholt werden. Das heißt, dass Du für jede Veranstaltung, die Du mit einer Fachprüfung abschließen musst, nur drei Versuche zum Bestehen hast. Wenn Du bei zwei Versuchen durchgefallen bist, dann wirst Du von der Studienberatung zu einem Beratungsangebot eingeladen (siehe Grafik).

    \bildmitunterschrift{../grafik/artikel/pruefungsablauf_alternativ}{width=\linewidth}{}{}

    Fachprüfungen können schriftlich oder mündlich sein. Für Fachprüfungen muss man sich, genau wie für Studienleistungen, anmelden. Dies geschieht online im TUCaN-System (https://www.tucan.tu-darmstadt.de, siehe auch den letzten Abschnitt).

    Eine Fachprüfung, die einmal angetreten wurde, muss im Allgemeinen bestanden werden, um den Abschluss zu erhalten. Es ist jedoch einmalig möglich, ein ausgewähltes Modul aus dem Wahlbereich auch nach angetretener Prüfung abzumelden.

    Prüfungen finden (von wenigen Ausnahmen abgesehen) in der vorlesungsfreien Zeit statt. Spezielle Prüfungszeiträume, in denen alle Prüfungen stattfinden, gibt es hier nicht.\\


    \noindent\textbf{Auflagen}
    Wenn bestimmte inhaltliche Zulassungsvoraussetzungen, die für den Masterstudiengang notwendig sind, nicht im Bachelor erlangt wurden, so können Auflagen in Form von Veranstaltungen erteilt werden. Diese Auflagen müssen innerhalb eines Jahres in maximal zwei Versuchen erbracht werden, ansonsten kann die Zulassung wieder entzogen werden. Falls Du der Meinung bist, die Inhalte einiger Auflagen bereits erfüllt zu haben, offizielle Nachweise darüber vorhanden sind und diese vom Fachbereich akzeptiert werden, so können Auflagen auch wieder erlassen werden. Dafür muss Du Dich an die Fachstudienberatung wenden.\\


    \noindent\textbf{Pflichtveranstaltungen}
    Einige der Spezialisierungsmaster enthalten Pflichtmodule, die abgelegt werden müssen. Für Studierende, die bereits vorher an der TU Darmstadt studiert haben, gilt: Wurde eines dieser Module bereits absolviert, so müssen die entsprechenden CP im Wahlpflichtbereich erbracht werden.\\

    \noindent\textbf{Gesamtnote}
    Die Gesamtnote Deines Studiums setzt sich aus den nach CP gewichteten Noten Deiner eingebrachten Leistungen zusammen.\\


    \noindent\textbf{TUCaN: Studiengangsordnung digital}

    Seit dem Wintersemester 2010/11 gibt es an der TU Darmstadt TUCaN. Dort ist die Ordnung Deines Studiengangs digital hinterlegt, mit all ihren hier beschriebenen Regeln. Das Wichtigste für dich: In TUCaN meldest Du Dich für die Prüfungsleistungen an und ab!
}
{Ingo Reimund und Thomas Pilot, überarbeitet von Stefan Gries, Tobias Otterbein, Johannes Alef und Stefan Pilot}

\footnotetext[1]{\url{http://inferno.dekanat.informatik.tu-darmstadt.de/pp/}}
