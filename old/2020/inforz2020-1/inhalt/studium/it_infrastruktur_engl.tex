\artikel{IT Infrastructure}
{At the TU Darmstadt and the department of computer science there is a lot of technology available for students. In this article you will learn what is available and who is responsible for administration.
}{
    \subsection*{At the department}

    \noindent\textbf{ISP}

    The ISP (Infrastruktur und studentischer Poolservice) \footnotemark[1] (the former RBG) administrates most of the IT-Infrastructure of the Robert-Piloty-Building and offers several services for students and research groups. The ISP is running about 15 servers and more than 100 Pool-Computers that are regularly maintained.\\

    \noindent\textbf{Account}

    Every student that studies computer science or a study program with computer science courses (CE, iST, EtIT) can get an ISP/RGB-account to use the services of the ISP. The account can be applied for and administrated on \footnotemark[2]. The user name is the TU-ID. The account stays active for one term and has to be activated for the following term, again on \footnotemark[2]. Before your account is deactivated you get a message to your ISP/RGB-Emailaccount (see paragraph services). If your account was inactive for two terms it gets deleted.\\

    \noindent\textbf{ISP-Pools}

    The ISP runs two large PC-Pools on floor 0 of the Piloty-Building. The C-Pool with about 100 computers is in the C-wing of the building. The operating system on theses computers is Linux Mint. Additionally, there are some spots where you can work with your notebook. The E-Pool in the E-wing is mostly intended for notebook users and only has a few computers.

    The C-Pool is open from Monday to Friday from 7:30 am to 8 pm. The E-Pool can be accessed at all times if you have a transponder (see paragraph services).

    In the pools there are printers that you can access.
    Within the new system papercut you have to register your Athene card to your ISP account \footnotemark[3].
    Every student can print 50 pages each month for free.
    By the end of the month the value of your remaining pages is stored on your account until you reaches 3 \euro.
    It is possible to print either from the computers in the pool, your own computer \footnotemark[3] or via the web interface \footnotemark[4].
    The web page is only available within the university network.
    The printer will print your pages when you have enough money remaining for the job and you hold your card against the papercut reader in the pool.

    Every student has 1 GB of storage on the computers in the Pools. For more data you can use temporary folders: \footnotemark[5]. But those are deleted every night.

    You can use SSH \footnotemark[6] to access the pool-computers from any other computer you use. This way you can access data or use the software even if you can't use the pool right now. Before you can do this you must place a SSH-key on the computer in the pool. A manual (in German) can be found here: \footnotemark[6].\\

    \noindent\textbf{Services of the ISP}

    Every ISP/RGB-Account also has an email address of the following form: $<$username$>$@rbg.informatik.tu-darmstadt.de. The ISP has a Wiki with manuals for setting up Email-clients \footnotemark[7] and using the web access \footnotemark[8]. If you don't need an extra Email-Account you should forward your ISP/RGB-Emailaddress \footnotemark[7] as important information by the Dean's office, Student Advisory Service or Fachschaft may be sent to this address.

    If you want to access the E-Pool when the building is not open you need a transponder. You can use this transponder to access the E-Pool through the door of the E-wing. You can find more information on the transponder here: \footnotemark[9].\\

    \bildmitunterschrift{../grafik/artikel/poolraum}{width=\linewidth}{The C-Pool of the Piloty-Building}{Claudius Kleemann}

    \subsection*{University}

    \noindent\textbf{HRZ}

    The university IT Administration HRZ (German: Hochschulrechenzentrum (University data centre)) \footnotemark[10] offers IT-services for the entire university, similar to the ISP for the department. But it offers some more services than the ISP.\\

    \noindent\textbf{Account}

    The HRZ-account uses your TU-ID as user name. You can activate it with the password on your matriculation letter. This account can be used to access most of the IT-systems at the TU Darmstadt. The account is connected to yet another email-address that you can forward. \footnotemark[11]\\
    \\\\
    \noindent\textbf{Services by HRZ}

    One of the most important tasks of the HRZ is to provide WiFi on the entire campus. There are two WiFi networks available: the unencrypted TUDWeb (not recommended) and the encrypted eduroam. Eduroam is available at many European and international universities and can there be used with the TU-ID as well. To access this network you need your TU-ID and password \footnotemark[12]. If you want to access the network from outside the campus you have to connect via VPN. Manuals can be found here: \footnotemark[12].

    The HRZ also provides the Athene-card \footnotemark[13]. With this card you can pay in the cafeteria and it serves as an ID-Card for the library. It is planned to expand its uses in the future.

    The HRZ also offers computer pools with Windows as operating system. You can access these with your TU-ID and password. Theses pools also provide printers where you can print in different sizes for little money.

    To support teaching the HRZ offers several E-learning services \footnotemark[14], for example OpenLearnWare where video recordings of lectures can be watched.

    In the service centre of the HRZ \footnotemark[15] you can get help with problems concerning WiFi, VPN and the other services. Additionally, you can borrow hardware like projectors, screens and cameras.
}
{Tobias Otterbein\\edited and translated by Johannes Alef\\edited by Anna-Katharina Wickert}

%\bildmitunterschrift{../grafik/comics/file_extensions}{}{}{xkcd.org}


\footnotetext[1]{\url{https://www.isp.informatik.tu-darmstadt.de}}
\footnotetext[2]{\url{https://support.rbg.informatik.tu-darmstadt.de}}
\footnotetext[3]{\url{https://support.rbg.informatik.tu-darmstadt.de/wiki/de/doku/computerhilfe/drucker/papercut}}
\footnotetext[4]{\url{https://print.informatik.tu-darmstadt.de/app?service=page/UserWebPrint}}
\footnotetext[5]{\url{https://support.rbg.informatik.tu-darmstadt.de/blog/2014/03/temp-speicherplatz-fuer-24h/}}
\footnotetext[6]{\url{https://support.rbg.informatik.tu-darmstadt.de/wiki/de/doku/computerhilfe/ssh}}
\footnotetext[7]{\url{https://support.rbg.informatik.tu-darmstadt.de/wiki/de/doku/computerhilfe/mail/email}}
\footnotetext[8]{\url{https://webmail.rbg.informatik.tu-darmstadt.de/mail/}}
\footnotetext[9]{\url{https://www.informatik.tu-darmstadt.de/?id=4376}}
\footnotetext[10]{\url{https://www.hrz.tu-darmstadt.de}}
\footnotetext[11]{\url{https://www.hrz.tu-darmstadt.de/mail/e\_mail/mail\_studierende/}}
\footnotetext[12]{\url{https://www.hrz.tu-darmstadt.de/netz/netzzugang_internet/netz_datennetz_internet_vpn_1/}}
\footnotetext[13]{\url{https://www.hrz.tu-darmstadt.de/angebote_studierende/studierende_athenekarte/}}
\footnotetext[14]{\url{http://www.e-learning.tu-darmstadt.de/elearning/}}
\footnotetext[15]{\url{https://www.hrz.tu-darmstadt.de/support/hrz_service/}}

\newpage
