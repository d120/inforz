\artikel{Where and how to study?}
{The opinions of students differ in many ways, but most of all on the subject of studying. Everyone studies in a different way because everyone perceives his life differently. And that is important! The most important distinction when it comes to studying is between those who study at home and those who study on the campus.
}{
    %\noindent\textbf{Working rooms}
    %
    It doesn't matter where you study. It may be hard to believe but you don't even need to own a computer to study for your exams. If you are studying on the campus you can use one of the larger working rooms or look for a room in another building where you meet your fellow students to study with them.\\

    \noindent\textbf{Piloty-Building (S2$|$02)}

    First of all there are the PC-Pools, the C-Pool and the E-Pool. There you have computers to work with. The Pools are usually crowded but you can even go there at night if you put in a request for a Transponder from the ISP. Additionally, there is the Bistro Athene (C301) which is also often crowded because of the good atmosphere and the kiosk. Last but not least there is the learning centre of computer science (LZI) in room A020.\\

    \noindent\textbf{Mensa City Campus}

    The rooms of the cafeteria (S1$|$11) are open from 7 am to 7 pm. There is plenty of space for studying. And there is of course the Bistro that offers coffee and sandwiches. It sounds good but it really depends on how many people are already learning there because it can get quite loud.\\

    \bildmitunterschrift{../grafik/artikel/lernen_utilities}{width=\linewidth}{}{Henry Klingberg / PIXELIO}

    \noindent\textbf{Library}

    The library (S1$|$20) offers a lot of literature and long opening hours (currently January to March and June to August a 24-hour service, in the other 6 months daily from 07:00 - 01:00). But of course it is a library so you have to be quiet there. You can rent small rooms there for free, for example when working on your thesis. But this has to be planned months ahead as they are almost always occupied. If you want to discuss a small subject, you can do it during a lunch or a coffee at the Lesbar (a small bistro next to the library).\\

    \noindent\textbf{Other rooms}

    Apart from the "official" working rooms you also have the possibility to look for rooms that are not used or booked for usage and study there. A good starting point for this is the old main building (S1$|$03). \\

    \noindent\textbf{Studying at home}

    Many students also like to study at home because it is more quiet and they can concentrate better. But this may not always be a good tactic for learning. In fact, it is often better to study with other students. So if you plan to study at home, try to do it together with other students.\\

    \noindent\textbf{Workload}

    It is very important not to underestimate the workload of the courses. Your schedule may look fairly empty but that is misleading. Although for some courses it may be enough to just visit the lectures and attend the exercises, for most courses you will have to invest more time than that. You may have to use the same time you spend in the lecture or even double that time to go through the material again and really understand it. So don't underestimate any course. Take them all serious from the beginning of the term. It is better to invest more time in the beginning of a course and then see that you need less time later on than to find out that you should have done more by the end of term. This can be really important for passing exams.
}
{Andreas Marc Klingler,\\edited by Patrick Toschka\\edited and translated by Johannes Alef}
%\newpage
