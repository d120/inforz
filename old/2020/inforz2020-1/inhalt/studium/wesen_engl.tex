\artikel{Das Wesen der Informatik...}
{...is what we call the Fachschaft's mascot. You've probably already seen some variants of it scattered throughout this booklet, but its original form, depicted below, is what gave it its name in the first place. Here is a little story about what it symbolises}
{\bildmitunterschrift{../grafik/wesen/wesen_transparent}{width=\columnwidth}{}{}

    First of all, what does its name actually mean? Being a certain kind of wordplay, its meaning is naturally hard to translate: "Wesen" can be translated both as "being" or "creature" as well as "essence".
    And in a way it is supposed to represent both of these meanings: being a small child it is certainly a being -- or person, if you will.
    On the other hand, it personifies several fundamental aspects of computer science as a scientific discipline, thereby relating to its very essence.

    You might not know this, but the Wesen has been the Fachschaft's mascot for a very long time, namely since 1986, about ten years after the installment of computer science as a distinct scientific discipline in German universities.
    By the standards of most other sciences, computer science is a very young discipline, but not long after its emergence many students already felt that it might have grave impact on society -- but no one was quite sure which way that would turn out.
    That is why the image was seen as fitting: what happens if you give an assault rifle to a toddler? No one knows, but there no-one has a good feeling about it.\\

    \textbf{What is computer science?}

    Computer science (or informatics, respectively) is commonly understood as the scientific discipline of systematic processing of information, primarily regarding automated processing using digital computing devices.
    In and by itself, this does not appear particularly dangerous. After all, solving problems more efficiently in an automated fashion is convenient, what could possibly be wrong with that...?\\

    \textbf{Example: Pathfinding}

    One of the earliest problems tackled by computer science is pathfinding. The goal is getting from point A to B in the most efficient way.
    This may, depending on context, mean the shortest, fastest, or least encumbering path. Or potentially the one with most sightseeing spots between two points.
    Obviously, navigation systems for cars or other means of transportation are use cases for this technology, in order to reduce human error as well as costs.
    However, armed combat drones, dispatched to quickly and efficiently dispose of terrorists and other unpopular persons in an automated fashion, rely heavily on this kind of algorithm.\\

    \textbf{A second example: robots}

    Robots have been in industrial use for quite a while now. They are valued mainly for their abilities to take on tasks that are too strenuous or too dangerous for humans.
    An example is the automobile industry. Stationary robots move heavy parts, assemble cars and partially already perform quality checks in concert with an array of interconnected sensors.
    In this highly digitalised work environment, humans have partially even become a nuisance, as their manual labour oftentimes does not live up to the machines' standards of efficiency any more.
    To make matters even worse, humans, in contrast to robots, also have elevated monetary demands, yet demand sleep and other pesky rights...\\

    \textbf{Another example: Machine Learning and AI}

    With the continuous increase in processing power and storage capacity, whole new opportunities for scientific data analyses have opened up in recent years.
    Using machine learning and artificial intelligence, terabytes of data, as for example sensors in today's particle accelerators generate within a short amount of time, can be processed quickly and efficiently.
    These algorithms have enabled whole new scientific disciplines, among others the already mentioned particle research or in the medical neurosciences.
    However, research labs are not the only sources of large amounts of data; ever since a large percentage of the world's population accesses the internet on a regular basis, states and companies keep ever growing databases as well.
    And wouldn't it be a waste not to use learning algorithms on these datasets in order to understand your citizens or customers better and be able to deliver more personalised interaction experiences to them.
    After all, the better you know them, the easier you can use that knowledge to manipulate your customers (or citizens) for your profit (or political agenda)...\\

    \textbf{Computer science and society}

    Within merely a few decades, computer science has gained a foothold in nearly every aspect of our lives and society, as the examples above (and many others) demonstrate.
    In many cases, the effects are beneficial for us as individuals, but there are often hidden costs to these benefits -- which even we as computer scientists often cannot foresee when we develop such solutions.
    Thus, with our contributions to the field of computer science we are often the metaphorical na\"ive toddlers who have been given a gun to play with, for naturally we tend to only see the beneficial effects of our work, and not its potential for abuse.
    As such, computer science is more than just a scientific discipline, since we computer scientists are going to have a large influence of the direction in which our respective societies are going to move towards.
    This is a great opportunity, but at the same time also imposes on us a large responsibility, which we should always keep in mind.
}
{Stefan Gries}
\newpage
