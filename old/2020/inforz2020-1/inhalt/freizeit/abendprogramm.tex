\artikel{Abendprogramm}
{Heute Abend schon was vor?
}{
    \textbf{Kino}

    In Darmstadt gibt es diverse Kinos: das Kinopolis am Bahnhof und die kleineren Kinosäle Helia, Pali, Festival und Rex in der Nähe des Luisenplatzes. Das komplette Programm findest du tagesaktuell unter  \footnotemark[1].
    Als gute Alternative zum normalen Kino gibt es die Vorstellungen des studentischen Filmkreises. In der Regel finden während der Vorlesungszeit jede Woche zwei Filmvorführungen statt. Dazu gibt es vorher jeweils einen Kurzfilm, und außerdem kaum Werbung und vor allem kein Popcornmonopol – Essen und Getränke dürfen selbst mitgebracht werden.

    Eine Karte kostet 2,50\euro. Zusätzlich muss ein Mitgliedsausweis erworben werden, welcher zusammen mit dem Eintritt aber immer noch weniger als ein normaler Kinobesuch kostet und ein Jahr lang gültig ist. Er kann vor jeder Vorstellung direkt an der Kasse gekauft werden. Das aktuelle Programm findest du unter \footnotemark[2].

    Wer es lieber luftig mag, kann im Sommer im Schlosshof das Open-Air-Kino besuchen.\\

    \textbf{Theater}

    Viel Kultur bietet ein Besuch im Staatstheater Darmstadt. Studierende erhalten hier unter Vorlage des Studienausweises einen Rabatt und darüber hinaus ab drei Tage vor Veranstaltungsbeginn die Restkarten, egal welcher Preisklasse, komplett kostenlos. So kann ein Theaterbesuch deutlich günstiger sein als Kino.

    Außerdem gibt es auch das halbNeun-Theater\footnotemark[3] und das Theater Moller Haus\footnotemark[4].
    \\\\
    \textbf{Lyrik}

    Definitiv lohnenswert ist der Besuch eines Poetry Slams. Diese finden monatlich in der "`Goldenen Krone"' statt und teilweise auch in der Centralstation. Wenn du selbst einmal vor Publikum stehen willst, bietet der Krone Slam darüber hinaus auch eine offene Liste für Neulinge.
    Unter \footnotemark[5] und \footnotemark[6] gibt es jeweils aktuelle Termine. Außerdem gibt es in Seeheim noch die Open-Air-Dichterschlacht.
    \\\\
    \textbf{Musik}

    Im Schlosskeller (im Innenhof des Schlosses) gibt es je nach Wochentag verschiedene Musikrichtungen zu hören. Das Angebot ist breit gefächert und oft hört man bisher Ungehörtes. Zusätzlich finden hier in regelmäßigen Abständen Musikevents statt. Einfach mal auf \footnotemark[7] vorbeischauen.
    Im Juli neu eröffnet wurde das 806qm direkt neben der Mensa. Neben dem Caf\'ebetrieb über Tag gibt es hier abends Konzerte zum Studentenpreis. Das Programm findet man unter \footnotemark[8].
    Sowohl der Schlosskeller als auch das 806qm sind Gewerbe des AStA, sodass Einnahmen in gewissen Teilen wieder der Studierendenschaft zugute kommen.

    Musik und Kabarett gibt es in der Centralstation (im Innenhof des City-Carree). Tickets und Informationen zum aktuellen Programm gibt es unter \footnotemark[9]. Ein ähnliches Angebot gibt es im Darmstädter Kongresszentrum, dem darmstadtium. Wem die Leuchtwerbung über dem Haupteingang nicht auffällt, der kann unter \footnotemark[10] die kommenden Veranstaltungen nachschlagen.

    Freunde klassischer Musik kommen mit den Aufführungen der Philharmonie Merck im regionalen Umfeld sowie den Konzerten im Staatstheater auf ihre Kosten. Zuweilen bieten auch Hochschulgruppen wie das Orchester der TU Darmstadt oder der Chor Kostproben ihres Könnens.\\

    \textbf{Party}

    Wer's lieber laut und tanzbar mag, sollte sich einmal die Clubs in Darmstadt näher ansehen: auch hier ist für praktisch jeden Geschmack etwas vorhanden – neben der Goldenen Krone nahe des Schlosses mit sehr gemischtem Programm, dem oben erwähnten Schlosskeller und der House/Hip-Hop/Dancehall Großraum-Disco Musikpark Darmstadt ("`A5"') in Richtung Weiterstadt gibt es in Mühltal-Traisa (etwas außerhalb von Darmstadt) auch noch das Steinbruch Rock-Theater für Freundinnen und Freunde härterer Musik.
    In der Innenstadt findet ihr außerdem noch das Nova mit verschiedenster Tanzmusik.

    Ansonsten reicht, was Partys angeht, eigentlich fast schon ein Verweis auf \footnotemark[11]: So gut wie alle aktuellen Partys und Veranstaltungen sind hier eingetragen. Ansonsten findest du auch in verschiedenen Kultur-Magazinen, zum Beispiel dem P-Magazin, viele Anregungen zum Abfeiern.

    Ganz groß finden in Darmstadt außerdem jedes Jahr zwei Straßenfeste rund um das Schloss statt: das Heinerfest und das Schloßgrabenfest. Das Schloßgrabenfest zeichnet sich vor allem durch viele Bühnen aus, auf denen verschiedene Musikrichtungen gespielt werden, während das Heinerfest das größte und älteste hessische Volksfest ist.

    Drumherum in den Darmstädter Stadtteilen finden ebenfalls (wenngleich kleinere) Straßenfeste statt und die Pfalz ist mit ihren vielen Weinfesten im Spätsommer auch nicht weit. \\
    \\\\
    \textbf{Angebote der Fachschaft}

    Games no Machines (GnoM) ist der Spieleabend der Fachschaft Informatik. Hier kannst du gemütlich mit Kommilitoninnen und Kommilitonen zusammensitzen und alle erdenklichen Gesellschaftsspiele ausprobieren, genießen, perfektionieren oder wonach dir auch immer der Sinn steht. Mehr hierzu findest du im entsprechenden Artikel.\\
    Während es bei GnoM um Gesellschaftsspiele geht, geht es bei seinem kleinen Bruder, dem RPGnoM, um Pen-and-Paper Rollenspiele.
    Der RPGnoM ist der offene Rollenspielabend der Fachschaft und findet ca. ein Mal pro Monat statt. Er richtet sich an alle, egal ob erfahrene*r Rollenspieler*in oder Neuling. Neben Rollenspielrunden mit diversen Systemen gibt es regelmäßig auch Diskussionsrunden und Workshops rund um das Hobby. Alle Infos und Termine erhaltet ihr auf der Mailingliste unter \footnotemark[12].

    Die Games-Gruppe trifft sich während der Vorlesungszeit, um allen Hobby-Spieleentwicklern und Interessierten eine Austauschsplattform zu bieten und in kleineren Gruppen Spiele zu entwickeln. Auch hier gilt, bei Interesse kann man sich an \mbox{games@D120.de} wenden.

    Du möchtest auch so eine Gruppe leiten und hast eine tolle Idee dafür (z.B. Schachabend, Debattierclub, Münzfußball)? Melde dich einfach in D120. Wir helfen dir, dein Projekt bekannt zu machen und Räume dafür zu finden.
}
{Tobias Freudenreich, Martin Tschirsich, Stefan Gries,
    überarbeitet von Julian Haas, Johannes Alef}

\footnotetext[1]{\url{http://www.kinos-darmstadt.de}}
\footnotetext[2]{\url{https://www.filmkreis.de}}
\footnotetext[3]{\url{http://www.halbneuntheater.de}}
\footnotetext[4]{\url{http://www.theatermollerhaus.de}}
\footnotetext[5]{\url{http://krone-slam.de}}
\footnotetext[6]{\url{https://www.facebook.com/Dichterschlacht}}
\footnotetext[7]{\url{https://www.schlosskeller-darmstadt.de}}
\footnotetext[8]{\url{https://www.806qm.de}}
\footnotetext[9]{\url{http://www.centralstation-darmstadt.de}}
\footnotetext[10]{\url{https://www.darmstadtium.de}}
\footnotetext[11]{\url{http://www.partyamt.de}}
\footnotetext[12]{\url{https://lists.d120.de/mailman/listinfo/rpgnom}}

\newpage
