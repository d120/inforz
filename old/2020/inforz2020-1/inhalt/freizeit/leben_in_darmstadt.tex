\artikel{Leben in Darmstadt}
{Weil Lernen eben nicht alles ist: Auch Studierende sollten sich Freizeit gönnen. Und da man in Darmstadt viel unternehmen kann, findest du hier einige Anregungen.
}{
    Die vorigen Seiten haben sich mit der akademischen Seite des Studiums beschäftigt. Zum Studium gehört aber noch ein anderer, wichtiger Teil: die Freizeit. Sie dient als Ausgleich zu einem anstrengenden Tag und schenkt Erholung, um den nächsten Tag mit neuer Kraft meistern zu können und: sie lenkt uns ab und hilft so, den Kopf wieder frei zu bekommen.

    Deshalb ist es so wichtig, gerade auch in angespannten Wochen auf fest eingeplante Freizeitpausen zu achten. Lernen ist wichtig, aber mit einem freien Kopf geht es deutlich leichter. Ein Praktikum muss fertig werden, die Abgabe steht bevor - wenn du nicht erst am letzten Tag anfängst, musst du nicht bis Mitternacht daran arbeiten.

    Zur guten Freizeitgestaltung gehören gesellige Treffen (Partys, Spielabende usw.) genauso wie sportliche Aktivitäten. Die folgenden Seiten sollen dir dabei helfen, die verschiedenen Möglichkeiten in Darmstadt kennenzulernen und das für dich passende Freizeitprogramm zusammenzustellen.
}
{Tobias Freudenreich, Martin Tschirsich, Stefan Gries}
