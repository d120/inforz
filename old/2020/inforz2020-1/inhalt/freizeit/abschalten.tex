\artikel{Einfach mal abschalten}
{Eine der angenehmsten Möglichkeiten, seine Freizeit zu verbringen, ist einfach mal abzuschalten und sich zu entspannen, was besonders an wärmeren Tagen an der frischen Luft äußerst angenehm ist.
}{
    Darmstädterinnen und Darmstädter finden in ihrer Heimat eine Vielzahl schöner Orte zum Wohlfühlen und Entspannen, welche selbst von älteren Semestern unentdeckt bleiben: Im Norden der Bürgerpark direkt am Nordbad, im Süden an der Heidelberger Straße der Prinz-Emil-Garten und die Orangerie, am Ostbahnhof der Tiergarten Vivarium und die Rosenhöhe.
    Den Herrngarten, Darmstadts größte Parkanlage, können Informatikstudierende dagegen nicht übersehen, denn er befindet sich direkt auf der Rückseite des Piloty-Gebäudes. Auch die Mathildenhöhe mit dem Hochzeitsturm als Wahrzeichen Darmstadts und regelmäßigem Kunst- und Kulturprogramm darf nicht unbekannt bleiben.
    Im Sommer versprechen Freibäder und Badeseen Abkühlung: Neben den Schwimmbädern der Stadt, über die man sich am besten direkt online informiert, gibt es noch folgende Empfehlungen für Studierende: Das kleine Uni-Freibad direkt neben dem Hochschulstadion, welches durch kostenlosen Eintritt mit Studienausweis und WLAN-Versorgung auf der Liegewiese punkten kann.
    Wer lieber im See badet, der begibt sich kostenlos in das Arheilger Mühlchen oder in die Grube Prinz von Hessen. Beide liegen aber etwas außerhalb, näher an der Uni ist der Große Woog.
}
{Tobias Freudenreich, Martin Tschirsich, Stefan Gries}
\newpage
