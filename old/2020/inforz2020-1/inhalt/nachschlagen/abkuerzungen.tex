\addsec{Häufige Abkürzungen}

\textbf{Erläuterungen zu einigen beliebten und gebräuchlichen Abkürzungen an der TU Darmstadt. Für alle, die viele wichtige Sachen noch mal nachschlagen möchten.}

\begin{longtable}{p{20mm}p{85mm}}
    APB          & Allgemeine Prüfungsbestimmungen sind das Regelwerk, nach denen du deine Prüfungen schreiben darfst und musst.                                                                                                                                                           \\
    AStA         & Der Allgemeine Studierendenausschuss wird vom Studierendenparlament gewählt und hat verschiedene Referate (Soziales, Fachschaften, Ausländer, uvm.). Er macht Hochschulpolitik und ist zuständig für viele Serviceangebote und Gewerbe wie z.B. den Schlosskeller.      \\
    B.Sc.        & Bachelor of Science. Mittlerweile der erste Hochschulabschluss.                                                                                                                                                                                                         \\
    CE           & Computational Engineering. Ein Studiengang aus Informatik, Mathematik, Maschinenbau und Elektrotechnik.                                                                                                                                                                 \\
    c.t.         & cum tempore. Die berühmte akademische Viertelstunde, die man zu spät kommen darf. An der TU Darmstadt gilt aber meist s.t.                                                                                                                                              \\
    EH           & Evangelische Hochschule Darmstadt.                                                                                                                                                                                                                                      \\
    ESG          & Die Evangelische Studierendengemeinde bietet Kurse und Freizeitaktivitäten nicht nur für die Protestanten hier an der TU Darmstadt an und unterhält ein eigenes Studierendenwohnheim.                                                                                   \\
    FB           & Diese Abkürzung steht für Fachbereich. Es gibt 13 verschiedene Fachbereiche an der TU Darmstadt. Jedem Fachbereich ist hierbei eine Nummer zugeordnet. So bekommst du vom FB 4, der Mathematik, deine Mathematikvorlesung. Die Informatik hat die höchste Zahl (FB 20). \\
    FBR          & Im Fachbereichsrat bestimmen Professor*innen, Mitarbeiter*innen und Studierende über Entscheidungen sowie Orientierung des Fachbereichs.                                                                                                                                \\
    FIfF         & Forum InformatikerInnen für Frieden und gesellschaftliche Verantwortung e.V.                                                                                                                                                                                            \\
    FS           & Die Fachschaft wird meist mit den Studierenden gleichgesetzt, die sich am Fachbereich in irgendeiner Weise engagieren. Formal gehören zur Fachschaft jedoch alle Studierenden eines Fachbereichs.                                                                       \\
    FSK          & Die Fachschaftenkonferenz trifft sich einmal im Monat, um über fachbereichsübergreifende Themen zu diskutieren und zu entscheiden.                                                                                                                                      \\
    FSR          & Der Fachschaftsrat ist das von dir gewählte Organ der Fachschaft. Er tagt regelmäßig Mittwoch um 18 Uhr in D120 im Robert-Piloty-Gebäude.                                                                                                                               \\
    GnoM         & Games no Machines ist der Name des Spieleabends der Fachschaft Informatik ohne Computerspiele.                                                                                                                                                                          \\
    h\_da        & Hochschule Darmstadt, früher Fachhochschule Darmstadt.                                                                                                                                                                                                                  \\
    HDA          & Die Hochschuldidaktische Arbeitsstelle bringt studentischen Tutor*innen pädagogisches Handwerkszeug bei und berät auch bei Referaten, Bachelor- und Masterarbeiten. Unser Feedback (Evaluation der Lehrveranstaltungen) machen wir mit der HDA zusammen.                \\
    HRZ          & Das Hochschulrechenzentrum versorgt die Nichtinformatikerinnen und Nichtinformatiker mit Rechenpower und alle Angehörigen der TU mit WLAN. Es verwaltet die Athene-Karte und bindet die TU Darmstadt an das Internet an.                                                \\
    iST          & Studiengang Informationssystemtechnik, welcher aus Teilen der Informatik und Elektrotechnik besteht.                                                                                                                                                                    \\
    ISP          & Der neue Name der RBG. ISP steht für Infrastruktur und studentischer Poolservice.                                                                                                                                                                                       \\
    KIF          & Die Konferenz der Informatikfachschaften aus dem deutschsprachigen Raum findet einmal pro Semester statt.                                                                                                                                                               \\
    KHG          & Die Katholische Hochschulgemeinde unterhält ein Studierendenwohnheim und organisiert Seminare.                                                                                                                                                                          \\
    LiWi/LW      & Lichtwiese. Auf der Lichtwiese haben wir Informatikstudierenden selten etwas zu tun. Im Sommer kann man hier draußen im Biergarten sitzen, lernen und entspannen.                                                                                                       \\
    LZM          & Im Lernzentrum Mathematik gibt es Skripte, Übungen, alte Klausuren mit Musterlösung und Beratung.                                                                                                                                                                       \\
    M.Sc.        & Master of Science. Ist gleichwertig zum Diplom und berechtigt auch zur Promotion.                                                                                                                                                                                       \\
    Piloty       & Robert-Piloty-Gebäude (Gebäude S2$|$02) = Hauptquartier und Lebensraum der Informatikerinnen und Informatiker. Man beachte den guten Schutz vor Sonneneinstrahlung, 1A-Anzahl von Poolrechnern, sowie die exzellente Kaffeeversorgung.                                  \\
    PO           & Die Prüfungsordnung regelt die Inhalte des Studiums, die Studienziele und vieles mehr.                                                                                                                                                                                  \\
    RBG          & Die Rechnerbetriebsgruppe ist für die technische Infrastruktur im Fachbereich Informatik verantwortlich. Anfang 2014 wurde sie in ISP umbenannt.                                                                                                                        \\
    RMV          & Rhein-Main-Verkehrsverbund                                                                                                                                                                                                                                              \\
    SFK          & Der Studentische Filmkreis ist eine Hochschulgruppe, welche zweimal in der Woche Filme im Audimax vorführt.                                                                                                                                                             \\
    SS n/SoSe n  & Das Sommersemester des Jahres n                                                                                                                                                                                                                                         \\
    s.t.         & sine tempore. Ohne akademische Viertelstunde muss man pünktlich kommen. Gegenteil von c.t.                                                                                                                                                                              \\
    StuPa        & Studierendenparlament                                                                                                                                                                                                                                                   \\
    TUCaN        & TU-Campus-Net                                                                                                                                                                                                                                                           \\
    TUD          & Technische Universität Dresden                                                                                                                                                                                                                                          \\
    TU Darmstadt & Technische Universität Darmstadt                                                                                                                                                                                                                                        \\
    ULB          & Universitäts- und Landesbibliothek Darmstadt. Der Neubau befindet sich zwischen Mensa und altem Hauptgebäude.                                                                                                                                                           \\
    USZ          & Das Unisportzentrum ist am Campus Hochschulstadion zu finden. Hier kann man sich für die meist kostenlosen Angebote anmelden oder Karten dafür erwerben.                                                                                                                \\
    WInfe        & Wirtschaftsinformatikerinnen und -informatiker gehören dem FB 1 an.                                                                                                                                                                                                     \\
    WS m/n       & Das Wintersemester von Herbst m bis Frühjahr n.                                                                                                                                                                                                                         \\
    ZSB          & Zentrale Studienberatung. Hilft bei nicht fachspezifischen Studienfragen.                                                                                                                                                                                               \\
\end{longtable}

\newpage
