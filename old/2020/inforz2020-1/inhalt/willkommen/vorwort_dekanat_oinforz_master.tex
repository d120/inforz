\artikel{Vorwort des Dekanats}{Liebe Studierende,}{

    Herzlich willkommen am Fachbereich Informatik der TU Darmstadt!
    Sie haben eine sehr gute Wahl getroffen. Der Fachbereich Informatik ist einer der drei ältesten Informatik-Fachbereiche in Deutschland, gleichzeitig aber auch einer der modernsten. So haben wir als erster Fachbereich auf Bachelor- und Master-Studiengänge umgestellt und bieten Ihnen heute eine große Auswahl an spezialisierten Master-Studiengängen und Vertiefungsfächern an, die in dieser Form in Deutschland einzigartig ist. Doch Vielfalt ist nichts ohne Qualität. Diese wird durch die seit Jahren sehr hohe Einschätzung von Personalchefs zu den besten Informatik-Studiengängen in Deutschland belegt.


    %Die Leerzeichen sollen verhindern, dass LaTeX die Zeilenumbrüche zu lang zieht.
    \

    Der hervorragende Ruf kommt nicht von ungefähr. Das Studium der Informatik an der TU Darmstadt ist anspruchsvoll, auch wenn Ihnen das möglicherweise nicht sofort so vorkommen wird. Lassen Sie sich nicht täuschen. Anspruch und Tempo der Lehrveranstaltungen steigen danach schnell an. Nutzen Sie daher die ersten Wochen, um sich zu orientieren und gut in das neue Umfeld einzuleben. Die von der Fachschaft Informatik organisierte Orientierungsphase bietet dazu eine hervorragende Gelegenheit.

    \

    Insbesondere im Masterstudium an einer Universität sind Sie für Ihr Vorankommen selbst verantwortlich. Einer unserer ehemaligen Dekane hat den Unterschied von der Schule zur Universität mit den verschiedenen Arten von Wegen auf einem Berg verglichen. Die Schule ist ein Wanderweg auf einer Alm, der breit und gut beschildert ist. Auf dem Weg kommen Sie vielleicht manchmal in Atemnot und der Schweiß rinnt Ihnen in die Stirn, aber nachträglich können Sie sich vermutlich an wenige außerordentliche Schwierigkeiten mehr erinnern.

    \

    %\bildmitunterschrift{../grafik/willkommen/muehlhaeuser_sw}{width=35mm}{Prof.~Dr.~Max Mühlhäuser}{}

    An der Universität geht es dagegen darum, von der Alm auf den felsigen Gipfel zu klettern. Sie bietet dazu ein Gewirr von Kletterpfaden an, aus denen Sie sich einen auswählen und das Ziel, begleitet von Bergführern (Dozenten, Tutoren und Mentoren), erklimmen. Die Bergführer stellen Ihnen die notwendige Ausrüstung zur Verfügung. Sie werden Sie jedoch niemals hochziehen, sondern Ihnen nur die nächsten Griffe zeigen. Klettern müssen Sie selbst!

    \

    Sie werden sich jede Woche selbst motivieren müssen, um zu den Vorlesungs- und Übungsstunden zu gehen, die Übungsaufgaben zu bearbeiten und sich auf die Klausurprüfungen vorzubereiten. Dabei ist der Lehrstoff umfangreicher und anspruchsvoller zuvor, so dass die Bearbeitung eines einzelnen Übungsblattes leicht einen Tag oder mehr beanspruchen kann. Und eine gute Prüfungsvorbereitung erfordert Wochen sorgfältiger Planung und konsequenter Durchführung.

    Wenn Sie sich nun bei dieser Klettertour sorgen sollten, dass der derzeitige Klettersteig oder die verwendete Klettertechnik nicht zum Gipfel führen oder Ihre eigenen Kräfte übersteigen sollten, dann ist es Zeit, die Route mit den Bergführern im Detail zu studieren. Vielleicht wäre eine andere Route besser für Sie geeignet, vielleicht war ein Fehler in der Wegbeschreibung, vielleicht gab es ein Missverständnis bei der letzten Besprechung, vielleicht sollten Sie ein Trainingslager aufsuchen. Es kann viele Gründe geben, frustriert zu sein. Da hilft dann nur die Analyse: Wo stehe ich im Studium, wo will ich hin und wie kann ich meine Fähigkeiten bestmöglich einsetzen und weiterentwickeln, um dorthin zu kommen? Dabei lassen wir Sie nicht allein, sondern stehen bereit, Sie mit verschiedenen Angeboten zu unterstützen. Scheuen Sie sich daher nicht, sich an Ihre Professoren, Mentoren, Tutoren und Studienberater zu wenden, damit wir gemeinsam mit Ihnen Lösungen für etwaige Probleme erarbeiten können.

    \

    %\bildmitunterschrift{../grafik/willkommen/fischlin_studiendekan}{width=35mm}{Prof.~Dr.~Marc Fischlin}{}

    Vergessen Sie aber bei aller Anstrengung auch nicht, sich umzublicken und die Aussicht zu genießen. Ihr Studium soll für Sie interessant sein und Spaß machen. Gleichzeitig ist es der Beginn eines neuen Lebensabschnittes mit exzellenten Möglichkeiten, den eigenen Horizont enorm zu erweitern sowie neue Erfahrungen und neue Freundschaften fürs Leben zu gewinnen. Nutzen Sie die vielfältigen Möglichkeiten, um einen Ihren Interessen angepassten Weg zu finden! Bewegen Sie sich nicht immer auf ausgetretenen Trampelpfaden. Schweifen Sie gelegentlich auch einmal bewusst ab vom Klettersteig und pflücken ein paar Blumen, aber behalten Sie dabei Ihr Ziel im Auge.

    \

    Im Hinblick auf den Studienabschluss sollten Sie auch nicht vergessen, dass reine Spezialisten nur im Studium scheinbar erfolgreich sind und dass der spätere Erfolg beim Berufseinstieg und im Beruf nur zum Teil
    von fachlichen Fähigkeiten und guten Noten im Abschlusszeugnis bestimmt wird. Den anderen Teil des Erfolgs machen überfachliche Qualifikationen aus. Je nach künftiger Tätigkeit kann es mitentscheidend sein, ob man gut argumentieren und in Wort und Schrift überzeugend formulieren kann, ob man als Teamplayer und Teamleiter gleichermaßen befähigt ist, ob man mit Kritik gut umgehen kann oder manches andere mehr. Wir empfehlen Ihnen daher, diese wichtigen überfachlichen Qualifikationen im Studium bewusst weiter zu entwickeln.

    \

    Am besten stellen Sie sich dann zur Erreichung dieser Zwischenziele ein Portfolio zusammen aus speziellen Kursen, besonders förderlichen fachlichen Lehrveranstaltungen, Tutoren- oder Hilfsassistenten-Tätigkeiten und Selbststudien-Anteilen. Eine besonders erwähnenswerte Möglichkeit ist das Engagement in der Fachschaft, die nicht nur diese wichtige Orientierungsphase für Sie organisiert hat, sondern die sich auch in vielfältiger Weise wirksam und konstruktiv für die Belange der Studierenden einsetzt und damit wertvolle und wichtige Beiträge zur Entwicklung des Fachbereichs Informatik leistet. So werden Sie vom Konsumenten zum Mitgestalter des Informatikstudiums in Darmstadt.

    \

    Nun wünschen wir Ihnen einen guten Start in das Informatikstudium an der TU Darmstadt!
}
{Prof.~Stefan Roth, Ph.D.~(Dekan) \\Prof. Dr. rer. nat.~Michael Waidner (Studiendekan)}

\newpage

