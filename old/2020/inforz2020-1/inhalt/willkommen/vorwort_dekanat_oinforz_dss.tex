\artikel{Foreword by the Department Chair and Dean of Studies}{Dear Students,}{

  We would like to welcome you most cordially to the Department of
  Computer Science of Technische Universit\"at Darmstadt.

  \

  We are very happy that you have chosen our department to continue your studies and we believe that you have made an excellent decision.
  Our CS department has been one of the first three Computer Science departments in Germany and it is also one of the most forward looking.
  We were one of the first departments to offer studies according to the new international Bachelor/Master system and for several years we have been offering the Master in Distributed Software Systems taught completely in English.
  However, the most important aspect for you is the quality of the education we offer.
  Based on a poll among human-resources managers at German companies, the Computer Science education in Darmstadt has been consistently ranked among the best at German universities.
  \

  %\bildmitunterschrift{../grafik/willkommen/muehlhaeuser_sw}{width=35mm}{Prof. Dr. rer. nat. Max Mühlhäuser}{}
  %\vspace{2mm}

  You come here with high expectations about how your learning experience will look like and we want to prepare you for what lies ahead.
  Success in your studies demand commitment, discipline, and adjusting to new ways of learning.
  You are used to studying hard and you already possess a solid foundation, but at the Master's level you will encounter courses that will introduce you to the state of art in their respective areas.
  We will challenge your analytic skills and your abilities for abstraction, as well as your aptitude for design and implementation of complex software.
  You must be able to work independently and proactively but also as part of a team.

  %\bildmitunterschrift{../grafik/willkommen/goesele_studiendekan}{width=35mm}{Prof. Dr.-Ing. Michael Goesele}{}
  %\vspace{2mm}

  Many of us in the faculty have studied abroad, so we know firsthand that it may take a few months to adjust to studying and living in a country with a different culture, life style, language, and more.
  Even the way the university and the study programs are organized may differ quite a bit from what you are used to.
  At the same time, many of us know how much of an enriching experience studying abroad is.
  Therefore, we highly encourage you to make the best of your time and take advantage of the many wonderful possibilities at and around TU Darmstadt.

  \

  To get you started, here are a few important tips to keep in mind:
  \\ \\
  \begin{description}
    \item[\parbox{\textwidth}{Be focused in your studies, but not over- \\
    ambitious.}]
      In the first term we recommend choosing courses that you are really interested in and whose content you can judge.
      Please, don't try to take too many courses at once: rather, focus on a few and do your best.
      It is important to start successfully and avoid being involved in too many activities.
    \item[Don't hesitate to ask for help.]
      If you have questions about organizational issues (e.g., you failed to register or de-register for an exam), if you have personal problems, or if you have problems with your visa, don't hesitate to ask for help.
      For study-related issues, in particular, the Examination Office and the Student Advisory Service are first points of contact.
      For all non-academic questions and issues, the International Student Service (ISS) is your best partner.
    \item[Be mindful of your visa status.]
      Please make sure that your visa status or resident permit in Germany remains always valid.
      It is important to remember deadlines and appointments with the authorities such as the Foreigners Office.
      Please pay close attention to any communication from the authorities -- keep your address current and carefully study your any letters in the mail.
    \item[Don't neglect the social side.]
      Skype \& Co.\ keep you in touch with home, but it is important to have a social life here.
      This is a great opportunity to make new friends and to learn about new things.
      The university offers a lot of social activities in which you can participate: sports, games, music, crafts, etc.
      And why not help out in the ``Fachschaft'', the representatives of the students, who put together this brochure?
    \item[Learn a few words of German.]
      German is a difficult language and no one expects you to master it in the two years you spend here.
      Having said that, knowing a few words of German will go a long way in your everyday life, be it in a restaurant, in a shop, or on the street.
      Just seeing that you make an effort can make a big difference, even if the conversation switches to English after a few sentences.
  \end{description}

  We, the Chair and the Dean for Studies of the Department of Computer Science at TU Darmstadt wish you a successful, exciting, educational, and happy time here in Darmstadt.
  We hope that you will enjoy your studies and that some time in the future you will look back fondly at the time you spent with us.

}{Prof.~Stefan Roth, Ph.D.~(Department Chair)\\ Prof.~Dr. rer. nat.~Michael Waidner (Dean of Studies)}
