\artikel{Mensa - the cafeteria}
{The cafeteria at TU Darmstadt is referred to as "Mensa" in German.
    Here you can get food from 11.15 to 14.00 .
}{


    \noindent\textbf{Symbol explanation}
    Each meal is signed with symbols to show you what they contain.

    \begin{tabular}{|c|c|p{3cm}|}
        \hline
        \rule{0pt}{1cm+1ex}\includegraphics[scale=0.5]{../grafik/artikel/vegetarian}  & (V)     & Vegetarian meal. It doesen't contain meat but can contain milk products or eggs. \\
        \hline
        \rule{0pt}{1cm+1ex}\includegraphics[scale=0.5]{../grafik/artikel/vegan}       & (Vegan) & Vegan meals don't contain any animal products.                                   \\
        \hline
        \rule{0pt}{1cm+1ex}\includegraphics[scale=0.5]{../grafik/artikel/chicken}     & (G)     & These meals contain chicken.                                                     \\
        \hline
        \rule{0pt}{1cm+1ex}\includegraphics[scale=0.5]{../grafik/artikel/porkandbeef} & (S, R)  & Theses meals contain pork or beef.                                               \\
        \hline
        \rule{0pt}{1cm+1ex}\includegraphics[scale=0.5]{../grafik/artikel/fish}        & (F)     & These meals contain fish.                                                        \\
        \hline
    \end{tabular}


}{Johannes Alef}
