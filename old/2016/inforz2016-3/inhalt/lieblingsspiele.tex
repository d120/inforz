\artikel{Aus der Fachschaft: Lieblingsspiele}
{Nach dem letzten Inforz wurden Proteste laut. Fachschaftler fühlten sich nicht repräsentiert. Daher auch in diesem Inforz die (hoffentlich letzte) Sammlung von
    interessanten und empfehlenswerten Computer- und Brettspielen für jedermann*frau.}
{\textbf{Name}: Jonas \\
    \textbf{Spiel}: GO\\
    \textbf{Warum}:  Nicht die Programmiersprache ist gemeint, sondern das
    spannende Brettspiel für zwei Spieler*innen. Ein Brett, beliebig viele weiße und schwarze Steine und nur fünf Regeln. Das sind die Grundlagen für dieses über 5000 Jahre alte Strategiespiel aus dem Reich der Mitte. Trotz dieses Aufbaus ist das Spiel nicht leicht. Computer schlagen den Menschen
    erst seit diesem Jahr. Ein Muss also für alle Informatiker*innen!\\
    \textbf{Für wen}: Menschen zwischen 3-99, die offen für neue Kulturen sind.\\

    \textbf{Name}: Mark \\
    \textbf{Spiel}: Total War \\
    \textbf{Warum}:  Total War zeichnet sich vor allem durch seinen hohen
    strategischen Anspruch aus. Die verschiedenen Gefechte benötigen gute Planung
    und Kalkül oder man wird schnell von der Armee des Gegners überannt. Wer gegen
    die Computergegner zu häufig gewinnt kann sich auch am Onlinemultiplayer
    versuchen und dort seine Skills beweisen. \\
    \textbf{Für wen}: Alle, die epische Strategiespiele lieben und vor relativ
    hoher Komplexität nicht zurückschrecken.\\

    \textbf{Name}: Johannes \\
    \textbf{Spiel}: Codenames\\
    \textbf{Warum}: Codenames ist nicht nur das Spiel des Jahres 2016, sondern auch
    ein Spiel, das am meisten Spaß mit guten Freunden macht. Denn deren Gedankengänge
    meint man immer noch am besten zu kennen. \\
    \textbf{Für wen}: Perfekt für mittelgroße (eventuell leicht alkoholisierte) Gruppen\\

    \textbf{Name}: Jannis \\
    \textbf{Spiel}: Final Fantasy (IX) \\
    \textbf{Warum}: Final Fantasy ist eine Spielreihe, die durch ihre Story
    besticht. Dabei erschafft sich das Spiel fast mit jedem Teil neu, spielt mal in
    einer fantastisch angehauchten, ein andermal in ein sehr futuristischen Welt. Weshalb man mit jedem Teil wieder abwägen muss, ob einem das Spiel gefällt. Ich persönlich bin nach wie vor begeisterter Fan des 9. Teils, auch wenn dieses inzwischen seine 16 Jahre alt ist. Das Spiel sticht hier durch seinen sehr ungewöhnlichen Artstil und seine liebevoll erzählte Geschichte hervor.  \\
    \textbf{Für wen}: Für jeden, der Lust und Laune auf eine gute Geschichte hat. \\
}
{Heiko Carrasco}
