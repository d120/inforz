\definecolor{myBlue}{RGB}{188,219,238} %www.farb-tabelle.de/en/table-of-color : LightSteelBlue2


\newcommand\addHeader[3]{
    Titel & #1 \\
    \tiny Originaltitel & \tiny #2 \\
    \tiny Filmkreistermin & \tiny #3 \\
}
\newcommand\addBeschreibung[1]{\multicolumn{2}{|p{5cm}|}{#1}\\}
\newcommand\addAnmerkung[1]{\small Anmerkungen & \small \textit{#1}\\}

\newenvironment{Film}{
    \centering
    \renewcommand{\arraystretch}{1.5}

    \begin{tabular}{|lp{3cm}|} \hline}
        { \hline
    \end{tabular}
    \vspace{1.5em}
}


\artikel{Mein Filmtipp zum Filmkreisprogramm}
{ Wie jedes Semester bringt der Filmkreis auch dieses wieder einige gute Filme.
    Um euch einen Einblick zu geben, folgt nun eine Liste von empfehlenswerten
    Filmen. Ich muss sagen, dass diese Auswahl natürlich weder neutral noch objektiv ist.}
{
    Hinweis: Leider kam diese Ausgabe später als gedacht und erhofft, weshalb ich mich entschieden habe alle Filme des Programmes mit einzubeziehen und nicht nur noch kommende. Vielleicht ergibt sich ja nochmal die Gelegenheit einen der Filme zu sehen.

    \begin{Film}
        \addHeader{Ex Machina}{Ex Machina}{13.10.16}
        \addBeschreibung{Stell dir vor, ein millionenschweres Genie lässt dich für
            einen Job auf seine einsame Ranch einfliegen. Deine Arbeit besteht darin,
            dich ein paar Stunden am Tag mit einer hübschen jungen Frau zu unterhalten, den Rest verbringst du in seiner Privatresidenz...
            Etwas verstörend und erschreckend durchdacht. Meiner Meinung nach hat der Film nicht ohne Grund eine Oscarnominierung erhalten.}
        \addAnmerkung{Überaschungsfilm}
    \end{Film}

    \begin{Film}
        \addHeader{The Jungle Book}{The Jungle Book}{18.10.16}
        \addBeschreibung{Das Dschungelbuch, ist zumindest für mich immer noch ein Stück Kindheit und umso schöner finde ich, dass der Film meine Erwartungen übertroffen hat. Eine gelungene Neuauflage, die sich gerade an Leute richtet, die mit dem Dschungelbuch groß geworden sind.}
        \addAnmerkung{Originalvertonung}
    \end{Film}

    \begin{Film}
        \addHeader{Spotlight}{Spotlight}{20.10.16}
        \addBeschreibung{Ein erschreckend ehrlicher Film über ein Thema, das wir alle zumindest ansatzweise mitbekommen haben. So finde ich den Film in seiner Art nicht nur besonders gut gelungen, sondern auch herausragend, wie er mit dem Thema des investigativen Journalismus arbeitet. Zurecht Teil des Filmprogrammes, sollte man gesehen haben.}
    \end{Film}

    \begin{Film}
        \addHeader{Cafe Society}{Cafe Society}{26.10.16}
        \addBeschreibung{Woody Allens neuester Film kommt zwar nicht an seine
            Highlights wie "Blue Jasmine" heran, ist meiner Meinung nach aber immer
            noch ein gelungener Film, der sich lohnt zu sehen. Für mich stechen hier
            sowohl die Requisiten, als auch die Szenerie heraus, wenn auch die Handlung etwas schwach ist. Etwas Nostalgie zum mitnehmen.}
        \addAnmerkung{Blinddate und Originalvertonung mit Untertiteln}
    \end{Film}

    \begin{Film}
        \addHeader{Zoomania}{Zootopia}{27.10.16}
        \addBeschreibung{Wie ich finde ein Film, der zu wenig Aufmerksamkeit bekommen hat. Ein kleines Meisterwerk, sowohl was die Animationen betrifft, als auch was die Handlung betrifft. Ich kann jedem nur empfehlen sich diesen Film anzusehen. Mehr will ich dazu auch gar nicht sagen.}
        \addAnmerkung{Originalvertonung}
    \end{Film}

    \begin{Film}
        \addHeader{Star Trek Beyond}{Star Trek Beyond}{17.11.16}
        \addBeschreibung{Eine gelungene Fortsetzung, die ihre Wurzeln nicht
            vergisst, uns dennoch in ein neues Abenteuer mitnimmt. Dabei erinnert
            dieser Film wieder mehr an die alten, erzählt von der Enterprise und einem
            ihrer vielen Abenteuern. Star Trek ist eine Reihe, die ich immer wieder
            gerne sehe und ich kann jedem, der Sci-Fi mag, empfehlen mal einen Blick auf diese Filme zu werfen.}
        \addAnmerkung{Originalvertonung}
    \end{Film}

    \begin{Film}
        \addHeader{Toni Erdmann}{Toni Erdmann}{22.11.16}
        \addBeschreibung{Der wohl einzige Film dieser Liste, den ich selbst noch
            nicht gesehen habe. Für mich steht er auf dieser Liste, weil er zum einen
            sehr viele deutsche Filmpreise gewonnen hat, als auch die deutsche Einreichung für den Oscar 2017 ist.}
    \end{Film}

    \begin{Film}
        \addHeader{Colonia Dignidad - Es gibt kein zurück}{Colonia}{24.11.16}
        \addBeschreibung{Ein Film, der mich auf seine eigene Art mitgenommen hat.
            Ich muss sagen, der Film ist härter als man es erwartet. Die Thematik ist
            schwer und ja, der Film macht sprachlos. Trotzdem oder gerade deswegen ist
            der Film einen Blick wert. Er ist großartig und ein kleines Meisterwerk,
            aber trotzdem ein Film, den ich mir sicher kein weiteres Mal ansehen werde.}
    \end{Film}

    \begin{Film}
        \addHeader{Die Feuerzangenbowle}{Die Feuerzangenbowle}{1.12.16}
        \addBeschreibung{Ein altes Meisterwerk, das jeder mal gesehen haben sollte. Vielleicht auch zwei- oder dreimal.}
        \addAnmerkung{Auf ein neues, so wie jedes Jahr}
    \end{Film}

    \begin{Film}
        \addHeader{The Hateful 8}{The Hateful Eight}{19.01.16}
        \addBeschreibung{Der achte Film von Quentin Tarantino und wie immer ein
            gelungener Film. An einigen stellen etwas zu stark überspitzt oder blutig,
            doch größtenteils ein klassischer Tarantino Film. Er kommt zwar nicht an
            Django heran, das muss er aber auch gar nicht.}
        \addAnmerkung{Originalvertonung mit Untertiteln}
    \end{Film}

    \begin{Film}
        \addHeader{Raum}{Room}{26.01.16}
        \addBeschreibung{Mein Highlight dieses Jahr. Der Film, auf den ich mich wohl am meisten Freue. Das Buch ist exzellent und der Film setzt dieses genauso um. Es ist jetzt nicht der spannendste oder dramatischste Film und doch der, der mich am meisten in seinen Bann gezogen hat. Sowohl das Buch als auch der Film sind ein Erlebnis.}
    \end{Film}
}
{Jannis Blüml}
