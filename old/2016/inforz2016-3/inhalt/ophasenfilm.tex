\artikel{Behind the Scenes - Making of the Ophasenfilm}
{Die inzwischen traditionellen und legendären Filmchen zu Beginn der Ophase
    lassen sich nicht mehr wegdenken. Doch wer hätte gedacht, dass hinter diesen
    einmaligen 15 Minuten eigentlich mehrere Monate Action stecken? Aus dem
    \enquote{Gedächtnistagebuch} eines damaligen Erstis.}
{\paragraph{t minus 1 Jahr}
    Die Lichter gehen aus. Der Puls steigt. Es wird still. Film ab. Es ist nun
    endlich so weit: mein erster Tag als Student an der TU Darmstadt beginnt.

    Nach der Ausgabe der Namensschilder wurde schnell klar, dass es bei der Ophase
    mit mehr Spaß zugeht als erwartet. Dass auf ihnen jeweils ein Pokémon
    abgedruckt war ließ erahnen, dass sich Pokémon als roter Faden durch die Woche
    ziehen wird.

    Die mysteriöse Aufforderung, sich im Audimax zu sammeln, deutete darauf hin,
    dass die Begrüßung sehr groß und deshalb wahrscheinlich klischeehaft mit
    PowerPoint beladen ausfallen dürfte.

    Fehlanzeige. Die Lampen werden gedimmt und auf der riesigen weißen Wand
    erscheint Professor Eich. Hat sich jemand ernsthaft die Mühe gemacht, extra für
    uns einen kompletten Film zu machen? In der nächsten Sequenz erscheint ein
    Ersti mit roten Haaren, zunächst als Grafik, dann in Person, um uns durch die
    Highlights eines Studiums zu führen. Vom Erwerben der Klausurenzulassungen über
    Mietpreiserhöhungen zum Bestehen einer Prüfung. Ein gelegentliches \enquote{Woah,
        cool.} geht durch die Reihen und zum Schluss ertönt ein tosender Applaus. Als
    dann die Tutor*innen sich vorne versammelten, wird schnell klar, dass diese
    Ophase richtig cool werden könnte. Es stellt sich schnell heraus, dass dieser
    Film von der \enquote{Fachschaft} gemacht wurde, unserer Studierendenvertretung. Ob sie
    das nächstes Jahr toppen kann?
    \paragraph{t minus 7 Monate}
    Zum Ende des Mentorensystems wird uns erklärt, was wir im weiteren Studium zu beachten haben,
    gefolgt von einem kurzen Vortrag über die Fachschaft. Moment, von der habe ich
    doch bereits gehört!
    Tatsächlich wird erwähnt, dass die Fachschaft Leute sucht, die engagiert
    mithelfen. Auch bei den Ophasenfilmen. Interessant. Merke ich mir.
    \paragraph{t minus 4 Monate}
    \enquote{Hey, der eine kommt mir bekannt vor. Ist das nicht ein Fachschaftler?}, sage
    ich an einem Samstagnachmittag in Aschaffenburg. Tatsächlich, nach einem kurzen
    Gespräch stellt sich heraus, dass Chris als Teil der nächsten Ophasenleitung
    sogar Kontakt zum Motto-Team pflegt. Er erklärt mir, dass die Mottos jeweils
    von diesem Team vorgeschlagen, entwickelt und dann von der Leitung abgesegnet
    werden. Mit den Worten \enquote{Kann ich mithelfen?} fängt ab hier mein Mitwirken an.

    So werde ich also zum ersten Treffen eingeladen.
    Da ein Motorschaden auf dem Weg zur Uni dazu führt, dass ich die Kekse auf
    dem Treffen nicht genießen kann, beobachte ich das Brainstorming-Geschehen von
    zu Hause aus. Claas, der bereits seine Rolle als Darth Assistent im Hinterkopf
    hat, schreibt in Windeseile Zeile für Zeile des Plots nieder.


    Am Anfang steht die Grundidee da, dass ein Held ausgebildet werden und die
    Fachschaft retten soll. Schnell sammeln sich auch Anekdoten, von denen sich
    jeder sicher ist, dass sie witzig und passend sind: TUCaN als Droide,
    beziehungsweise Mülltonne auf Rollen, die aktive Fachschaft als Rat der Jedi
    und natürlich Cameo-Auftritte von Sir Bearington.


    Da ich bis zu diesem Zeitpunkt tatsächlich noch nie Star Wars gesehen habe,
    kann ich erschreckend wenig zum Drehbuch beisteuern. Aus diesem Grund habe
    ich mich für die Rolle des Ersties gemeldet, immerhin bin ich im Moment selbst
    noch einer.


    \paragraph{t minus 3 Monate}
    Die Drehwoche ist schneller als geplant da - während Simon, der Mentor, ein
    professionelles und selbstgemachtes Kostüm hat, frage ich mich einen Tag vor
    dem Dreh, was ich als sein Schüler tragen soll. Natürlich haben die Geschäfte,
    die ich abklappere, keine Bademäntel in beige oder dunkelbraun. An dieser
    Stelle helfen die Nähfähigkeiten meiner Mutter, die aus einem Spannbettlaken
    etwas zaubert, das immer noch wie ein Spannbettlaken, aber irgendwie auch wie
    eine Robe aussieht.

    %\bildmitunterschrift{grafik/robe}{width=\columnwidth}{}{}

    %\bildmitunterschrift{grafik/rtucan2}{width=\columnwidth}{}{}

    Gleich am ersten Drehtag wird RTuCan2 vorbereitet - wir nehmen eine Mülltonne,
    streuen Kopfhörer und ein Raspberry Pi darüber und kleben Kulleraugen dran.
    Hierbei stellen wir fest, dass Sir Bearington mit Kulleraugen
    sehr gruselig aussieht.

    \bildmitunterschrift{grafik/rtucan2}{width=\columnwidth}{}{}

    %\bildmitunterschrift{grafik/bearington}{width=\columnwidth}{}{}

    Die Anfangsszene ist schnell im Kasten - man nehme ein paar Fachschaftler*innen, die
    geduldig im
    Hörsaal im Piloty sitzen können und dabei auch klatschen. Die Perspektive ist
    dabei entscheidend, da es deutlich voller aussehen muss, als es ist. Abgesehen
    davon musste natürlich beachtet werden, dass der Darth Dekan als Hologramm
    eingeblendet wird, weshalb eine Referenzaufnahme ohne Menschen benötigt
    wird.


    \bildmitunterschrift{grafik/bearington}{width=\columnwidth}{}{}

    Die eigentliche Magie der Szene sollte erst viel später von Kevin hinzugefügt
    werden. Er zeigt stolz seine hollywoodreifen Prototypen von einigen Szenen und
    den Lichtschwerteffekten, während er geduldig jede einzelne Szene im
    dreidimensionalen Raum tracken muss. Dabei wird die Position des Besenstiels
    im Video ermittelt und mit ein wenig Zauber im 3D-Programm Blender durch ein
    Lichtschwert ersetzt.


    Richtig, es wird mit Besenstielen gekämpft.


    Claas, der bereits mit Schauschwertkämpfen vertraut ist, zeigt mir im
    Herrngarten mit dem Besenstiel einige Bewegungen, die für Außenstehende sehr
    beeindruckend aussehen, aber ungefährlich sind, wenn man im richtigen Moment
    die richtigen Bewegungen ausführt.


    Dass es bei diesem Dreh wortwörtlich mit Blut, Schweiß und (Freuden)tränen zugehen
    wird, kann zu diesem Zeitpunkt noch niemand ahnen.

    Als wir zwischendurch
    einige Trainingsszenen filmen, wo ich mit einer grün angemalten Simone als Yoda
    auf dem Rücken durch Gänge renne, wird unser Ordner mit den Outtakes um eine
    Datei reicher: ich nehme die Kurve zu schnell und renne direkt gegen die Kante
    einer Wand.


    Während ich mich auf dem Fachschaftssofa erhole, werden gleich die nächsten
    Szenen im Besprechungsraum gedreht, wo sich die aktiven Fachschaftler*innen
    beraten, was sie gegen die bösen Pläne des Darth Dekan machen können.


    Nach der Erholpause kann ich hinter der Kamera erneut mithelfen um bei der
    Mensaszene für gute Kameraeinstellungen zu sorgen. Das Problem mit dem Ton ist
    wegen der Hintergrundgeräusche schlimmer denn je, weshalb die Hälfte im
    Nachhinein wahrscheinlich neu synchronisiert werden muss.

    Am zweiten Drehtag wird die Sandkastenszene gefilmt, in welcher der Mentor den
    jungen Padawan rekruitiert. Gerade noch rechtzeitig vor einem starken Regen,
    der uns bei der nächsten Szene begleitet, während wir am \enquote{zerstörten
        Mathebau}
    vorbeilaufen. Die Regentropfen verleihen dieser Sequenz zwar eine nette
    Atmosphäre, aber nasse Klamotten sind nie toll.


    Auch hier stellen
    wir direkt fest, dass diese Stelle neu vertont werden muss, aber das kann auch
    später passieren.


    Wichtiger ist, dass die weiteren Szenen halbwegs sitzen: Der Kampf zwischen
    Darth Assistent, einem ehemaligen Fachschaftler, und dem Mentor, gefolgt von dessen
    Exmatrikulation. Die Klausurenszene wird traditionell im kleinen Hörsaal in der
    Nähe von D120 gedreht, gefolgt von der letzten Kampfszene zwischen Darth
    Assistent  und
    mir.


    Nach dem zweiten Drehtag sind alle Szenen abgedreht. Denken wir.
    \paragraph{t minus 2 Monate}
    Tatsächlich bemerken wir schnell, dass einige Szenen überbelichtet,
    verwackelt oder zu dunkel sind - Kevin konnte zwar die meisten Szenen mit
    3D-Effekten versehen, jedoch nicht alle. Aus diesem Grund werden die
    Exmatrikulationsszene und der komplette mittlere Kampf neu gedreht.


    Die Choreographie zwischen Claas und Simon muss hierbei genau sitzen, da die
    Szene nicht in einem Rutsch gedreht werden kann. Umso wichtiger ist es, dass
    zwischen den Schnitten keine Bewegungen verloren gehen, weshalb jedes Detail
    geplant werden muss. Matthias und ich müssen uns ebenfalls koordinieren, da
    wir zeitgleich aus verschiedenen Perspektiven drehen und eine Kamera im Bild
    nicht schön anzusehen ist.


    So füttern wir Kevins Festplatte mit noch mehr Videodateien, die er dann auf
    einem Servercluster rendern lässt, jedenfalls ist das der Plan.
    Ab hier vergeht die Zeit wie im Flug und es fühlt sich so an, als wäre ich seit
    Monaten in der Fachschaft gewesen. Alle Studierenden kommen mir fast direkt
    vertraut vor, da man, egal, wohin man schaut, nur nette Personen trifft. Auch,
    wenn es zunächst ein wenig ungewöhnlich sein mag, seinen AFE-Tutor mit einem
    Besenstiel zu bekämpfen.
    \paragraph{t minus 4 Wochen}
    Die Pipeline für die Verarbeitung der Videos wirkt inzwischen wie ein
    einstudierter Tanz - Matthias neutralisiert die Farben der Videos, Kevin fügt
    die 3D-Effekte ein, ich einige 2D-Effekte und Jannis schneidet das Ergebnis mit
    dem passenden Ton zusammen. Sein Mitbewohner steuert noch Lichtschwertgeräusche
    bei, damit sich die Kämpfe so realistisch anhören, wie sie aussehen.


    Mit ein wenig Filmtrickserei wird auch die Szene mit dem Sturz repariert und
    filmtauglich gemacht, wobei wahrscheinlich beide Versionen im finalen Film zu
    sehen sind. Immerhin kann man über einige Narben im Nachhinein lachen, wenn die
    Geschichte dahinter absurd oder witzig ist.


    \paragraph{t minus 2 Wochen}
    Die ersten 3D-Szenen stehen in der finalen Fassung bereit. Kevins Computer
    berechnet inzwischen mit Lichtgeschwindigkeit (Füsiker dürfen sich an dieser
    Stelle gerne bei mir beschweren) die Einzelbilder, die vom Servercluster nicht
    optimal erstellt werden konnten.


    Ein großes Problem ist, dass die Schwerter natürlich nicht einfach so
    überlappend eingefügt werden können, da zum Teil Hände und Gegenstände im Weg
    sind. Diese müssen per Hand und pro Einzelbild maskiert werden, um zu
    vermeiden, dass zum Beispiel in einem Schauspielenden ein Lichtschwert steckt.

    Dank Heiko, einem weiteren Fachschaftler, können wir mit professionellem
    Zubehör die Szenen nachvertonen, die eher wie ein Stummfilm wirken. Hier sehen
    wir, dass es enorm mühsam ist, zu sprechen, wenn man dabei gleichzeitig
    auf Lippenbewegungen achten muss, die über mehrere Wochen versetzt sind. Obwohl
    es nur ungefähr zwei dutzend Zeilen sind, brauchen wir knapp anderthalb
    Stunden, bis wir zufrieden sind.
    \paragraph{t minus 1 Woche}
    Während der Programmiervorkurs der Fachschaft läuft und die Tutor*innen für die
    Ophase instruiert werden, finden wir noch Zeit, um gemeinsam alle Star Wars
    Episoden bei drei gemütlichen Filmabenden zu schauen.


    Während der meisten Szenen geht mir immer wieder ein Licht auf: \enquote{Ah, das haben
        wir also gedreht!}.


    \paragraph{t minus 1 Tag}
    Während des Drehs wurde mir oft gesagt, dass so ein Ophasenfilm laut Murphy’s
    Law immer im letzten Moment fertig werden muss. \enquote{Erinnerst du dich an den
        Pokémon-Film?} - \enquote{Ja!} - \enquote{Das war eigentlich kein Film, sondern die
        Bildschirmaufnahme eines ausprogrammierten Spiels, welches extra dafür erstellt
        wurde. Das Rendern des Films erfolgte noch in der Nacht vor der Premiere.}


    Auch in diesem Fall ist die finale Version erst am Sonntag vor der Ophase
    online. Die Erleichterung ist riesig, mehrere Wochen Arbeit gehen hiermit zu
    Ende. Jedenfalls fast.
    \paragraph{t minus 1 Minute}
    Es ist Ophase. Die Namensschilder werden ausgegeben, auf ihnen sind Star Wars
    Figuren abgebildet. Die Ersties ahnen schon, dass diese Filmreihe ein roter
    Faden in dieser Woche sein wird. Es bildet sich ein Strom vom Piloty-Gebäude
    zum Audimax, während wir uns ganz hinten im Hörsaal auf unseren Einsatz bereit
    machen.


    Die Spannung ist für das Mottoteam kaum zu ertragen. Wie wird der Film bei
    den Ersties ankommen? Wird das gemeinsame Einlaufen reibungslos funktionieren?


    Die Lichter gehen aus. Der Puls steigt. Es wird still. Film ab.
}
{Jan Bambach}
\vfill
\bildmitunterschrift{grafik/titlecrawl}{width=\textwidth}{}{}
\vfill
\pagebreak
