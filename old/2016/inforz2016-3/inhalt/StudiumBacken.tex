\artikel{Wie bekomme ich mein Studium gebacken}
{Bald ist wieder Weihnachten und ich besuche meine Familie. Wie soll ich
    meiner Oma, 80 Jahre, super Köchin, Floristin, kennt sowohl die
    deutschen als auch die lateinischen Namen von gefühlt allen Blumen, hat
    noch nie einen Computer bedient, erklären was ich im Studium mache?
}
{

    \begin{itemize}
        \item \textbf{Funktionale und objektorientierte Programmierung:} Ich
            lerne, wie man einen Kuchen backt.
        \item \textbf{Algorithmen und Datenstrukturen:} Ich lerne Kochrezepte
            auswendig.
        \item \textbf{Betriebssysteme:} Ich lerne, was Schüsseln, Mixer und
            Backblech sind.
        \item \textbf{Einführung in den Compilerbau:} Ich lerne, wie man einen
            Backofen bedient.
        \item \textbf{Systemnahe und parallele Programmierung:} Ich lerne, wie
            ich auf mehreren Herdplatten gleichzeitig kochen kann.
        \item \textbf{Automaten, formale Sprachen und Entscheidbarkeit:} Ich
            lerne, was für Kochrezepte man schreiben könnte.
        \item \textbf{Aussagen- und Prädikatenlogik:} Ich lerne, was alles als
            Kuchenrezept gilt.
        \item \textbf{Formale Methoden und Softwareentwurf:} Ich lerne, wie man
            überprüfen kann, ob ein Kuchen gut schmecken wird.
        \item \textbf{Mathe 1:} Ich lerne, wie ich Zutatenmengen angeben kann.
        \item \textbf{Mathe 2:} Ich lerne, wie ich ein Rezept für 2 Personen auch
            für 6 Personen backen kann.
        \item \textbf{Mathe 3:} Ich lerne so viel Zutaten zu nehmen, dass
            ungefähr 5 Personen, ein Hund und ein Goldfisch satt werden. (Den
            Goldfisch habe ich wahrscheinlich nicht, aber das macht in der
            Zutatenmenge einen vernachlässigbar kleinen Unterschied.)
        \item \textbf{Digitaltechnik:} Ich lerne, was es für Zutaten gibt.
        \item \textbf{Rechnerorganisation:} Ich lerne, wie ich Zutaten rühren kann.
        \item \textbf{Software Engineering:} Ich lerne, wie ich einen Kuchen backen kann ohne die Küche einzusauen.
        \item \textbf{Computersystemsicherheit:} Ich lerne, wie ich einen Kuchen
            backen kann, der garantiert nicht anbrennt.
        \item \textbf{Architekturen und Entwurf von Rechnersystemen:} Ich lerne,
            wie ich eine Kuchenschüssel backe. (Ja, ich kann die tatsächlich im Ofen
            backen!)
        \item \textbf{Modellierung, Spezifikation und Semantik:} Ich lerne, wie
            man überprüfen kann, ob ein Kuchen gut schmecken wird, indem ich einen
            einfacheren Kuchen mit weniger Zutaten und kürzerer Backzeit backe, der
            aber genau so schmeckt. (Dabei brauche ich extrem lange um den
            einfacheren Kuchen zu backen, dass ich eigentlich gleich den richtigen
            Kuchen hätte backen können, aber für große Kuchen macht das bestimmt
            irgendwann Sinn.)
        \item \textbf{Computational Engineering \& Robotik:} Ich lerne das
            Volumen eines Kuchens anhand der Backpulvermenge zu berechnen.
        \item \textbf{Informationsmanagement:} Ich lerne, wie ich im Kühlschrank
            nachschauen kann, ob ich noch genügend Zutaten habe.
        \item \textbf{Computernetzwerke und verteilte Systeme:}  Ich lerne, wie
            ich einen Kuchen mit mehreren Schichten backen kann.
        \item \textbf{Visual Computing:} Ich übe die Kuchenglasur.
        \item \textbf{Bachelorpraktikum:} Ich lade ein paar Freunde ein und
            backe einen Kuchen. (Weil wir zu wenig Mehl haben, benutzen wir ultra
            viel Backpulver, damit er trotzdem groß genug wird.)
        \item \textbf{Bachelorarbeit:} Ich entwerfe einen neuartigen Kuchen.
            (Den eh niemand essen will, weil ich eigentlich nur zwei Kuchen genommen
            habe und willkürlich zusammen gemischt habe.)
    \end{itemize}
}
{Jonas Kapitzke}
