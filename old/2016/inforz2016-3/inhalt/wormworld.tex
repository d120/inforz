\artikel{Wormworld Saga: \\ Ein Comic nicht nur für Kinder}
{Bei der Wormworld Saga handelt es sich um eine Comicreihe des deutschen Zeichners Daniel Lieske. Das erste Kapitel wurde bereits von Seiten der Community in über 25 Sprachen übersetzt. Und das beste an der Reihe, sie ist kostenlos.}
{
	Auch wenn Daniel Lieske in Deutschland recht unbekannt ist, so werden seine Comics doch weltweit gelesen und geliebt. Sein Comic "`Wormworld Saga"', der über seine Seite \url{https://wormworldsaga.com/} frei verfügbar ist umfasst inzwischen 9 Kapitel, welche er im Laufe der letzten 6 Jahren zeichnete. Dabei lebt der Autor nur von den Spenden und Einnahmen des dazugehörigen Shops. Die Idee zum Comic entstand dabei durch ein einzelnes Bild, dass er damals im Rahmen eines Wettbewerbes malte. Heute zählt "`The Journey Begins"' zu seinen bekanntesten Zeichnungen. \\

	Die Handlung des Comics dreht sich dabei um Jonas, einen Jungen der mit seinem Vater, wie jedes Jahr, über die Sommerferien zu seiner Großmutter aufs Land fährt. Für Jonas das Highlight des Sommers, ist es für ihn doch immer eine Reise mitten ins Abenteuer. Irgendwann entdeckt Jonas auf dem Dachboden des Hauses ein magisches Bild, das mitten in die Wormworld führt. Ohne zu ahnen, welche Folgen es für Jonas hat, betritt er diese Welt und findet nicht mehr zurück. Gestrandet in dieser für ihn vollkommen fremdartigen Welt, trifft er auf die Jägerin Rahja. Zusammen erfahren sie immer mehr über die magische Welt "Wormworld". So treffen sie auf Propheten, die in Jonas den Auserwählten sehen, lernen mehr über den König und sein Königreich kennen. Doch erfahren sie auch, dass die Welt vor einer Katastrophe steht und Jonas diese nur retten kann, wenn er sich seiner größten Angst stellt.	Für Jonas beginnt ein Abenteuer, dass er so nicht erwartet hätte.  \\

	Ich muss zugeben, die Handlung klingt jetzt erstmal etwas kindisch und einfach gehalten, das täuscht meiner Meinung nach gewaltig. Die Handlung, so kindisch sie auch ist, zeichnet eine der schönsten Geschichten über einen jungen Helden und wie er sich seiner Angst stellen muss. Außerdem geht es bei Comics um so vieles mehr. Für mich sticht der Comic durch seine Bilder und den Stil hervor. Es gibt immer wieder Bilder, die zum Staunen einladen und mich verzaubern. Allein die Arbeit, die in jedem einzelnen Bild steckt, macht den Comic für mich sehenswert. Dazu kommt die großartige Welt, die der Zeichner geschaffen hat. Alles kommt einem so fremd und doch wieder bekannt vor. So findet mach beispielsweise auch in dieser Welt eine Art von Industrialisierung. Diese beiden Faktoren tragen für mich diesen Comic und machen ihn so einzigartig. Ich sehe keinen Grund, warum er nicht einen Blick wert sein sollte, gerade, wenn er frei verfügbar ist. \\
}
{Jannis Blüml}
