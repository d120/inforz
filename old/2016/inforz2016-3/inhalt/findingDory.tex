\artikel{Filmrezension: Findet Dorie \\  \small(Originaltitel: Finding Dory) }
{Ich, als bekennender Findet Nemo Fan, kann nur sagen "Danke Pixar!". Nach eher mäßigen zweiten Teilen wie Cars 2 oder Planes 2, schafft es Pixar nach nur 4698 Tagen auch Findet Nemo eine schöne Fortsetzung zu geben…
}
{ Nun stellen wir uns natürlich alle die Frage, warum wir solange warten mussten. Ist der Film etwa so aufwendig gewesen, dass das Studio 13 Jahre daran gerechnet hat oder hatten die Synchronstimmen keine Lust mehr oder war der Findet Nemo einfach so gut, dass sie sich nicht an die Fortsetzung trauten? \\

    Ich kann nur sagen, egal was der Grund war, das Warten hat sich gelohnt. Findet Dorie ist ein durchaus gelungener Film, auch wenn er meiner Meinung nach nicht mit dem erstem mithalten kann. Dies liegt aber auch daran, dass Findet Nemo ein Teil meiner Kindheit ist und ich nicht weiß, wie oft ich ihn gesehen habe. Aber nicht so schnell, kommen wir erstmal zum Inhalt des Filmes. \\

    Wie schon der erste Teil, dreht sich auch im zweiten Teil wieder alles um die drei Freunde, Nemo, Marvin und Dorie. Wir begleiten dabei Dorie auf ihrer Suche nach ihren Eltern.  Hier treffen wir neue Freunde, wie Hank, den Septupus, Destiny und Becky, die uns aufs neue begeistern. Wir lernen, warum Dorie walisch spricht und wieso wir einfach schwimmen sollten. Und wer sich dann noch Zeit nimmt und den Abspann verfolgt, der bekommt heraus, wie es mit Kahn und seinen Freunden weiter geht. \\

    Der Film hat mich begeistert, ich fühlte mich wieder ein wenig in meine Jugend versetzt. Die Art, wie Pixar diesen Film umgesetzt hat, finde ich mehr als gelungen. Er spielt sehr schön auf seinen Vorgänger an, ohne ihn für die neue Jugend unattraktiv zu machen. Dabei erinnert mich der Film in keinster Weise an ein Remake, die Handlung ist neu und lädt zum Staunen ein. Ich musste lachen, fühlte mich ergriffen und habe mich mit den Charakteren gefreut. Ich bestaune Pixar immer wieder, wie toll sie es schaffen, Emotionen zu transportieren. Es ist jetzt kein Film, den man auf der großen Leinwand sehen muss, aber sehenswert ist er alle mal. Am besten schnappt man sich seine Freunde und freut sich wieder einmal in die Welt von Nemo einzutauchen.   \\

    Und zum Schluss kann ich nur sagen: \\
    "`Just keep swimming, swimming, ..."'
}
{Jannis Blüml}
