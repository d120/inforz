\section*{Liebe Leser*innen,}

hiermit veröffentlichen wir die erste Ausgabe des \textit{Inforz} im Wintersemester 2016/17. An alle, die zum ersten Mal ein \textit{Inforz} in Händen halten: Schön, dass ihr Interesse an unserer und eurer Zeitschrift habt. An alle anderen: Willkommen zurück.
\vspace{2mm}

\textit{Endlich geschafft!} Man könnte sagen, dass dies das Motto dieser Inforzausgabe ist. Nicht nur das dieses Inforz knapp 2 Monate nach seinem eigentlichen Erscheinungsdatum herauskommt, sondern war es bei jedem Text wieder aufs neue schön wenn er fertig war. \\

Was haben wir für euch in dieser Ausgabe? \\
Unschwer am Cover zu erkennen, haben wir dieses Mal einiges zum Thema Film. So könnt ihr euch über einen Artikel zu unserem Ophasenfilm, sowie Filmtipps zum Filmkreisprogramm freuen. Vor dreizehn Jahren kam \textit{Findet Nemo} raus. Inzwischen ist schon der Nachfolger da (ja, fühlt euch alt), daher findet ihr zum Jubiläum einen Artikel zu Findet Dory in dieser Ausgabe.\\
Wir berichten von der Buchaktion und den Ablaufänderungen, die wir vornehmen mussten. Ihr erhaltet wieder einen Einblick in die Lieblings[pc,brett]spiele der Fachschaft. Wir lösen auf, wie viele Gummibärchen in den VW Käfer passen und geben euch ein neues Rätsel mit auf den Weg. Das Ganze wird mit der Vorstellung von Arch und Fedora, zwei Linuxdistributionen und ein paar Rezepten aus unserer Weihnachtsbäckerei abgerundet. Unter anderem auch eines für das Bachelorstudium Informatik. \\

Wir danken allen, die Texte zu dieser Ausgabe beigetragen haben und wünschen euch jetzt viel Spaß mit diesem \textit{Inforz}.
Feedback und Texte für das nächste Inforz sind wie immer willkommen.

\vspace{5mm}
Heiko Carrasco \& Jannis Blüml



\vfill


\bildmitunterschrift{grafik/wesen/wesen_inforz}{width=4cm}{}{}
