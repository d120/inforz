%Comic auf Seite 2 und/oder Seite 3
\artikel{Arbeiten neben dem Studium}{BaFöG, Stipendien, Eltern, Verwandte... Alle Studierenden benötigen für ihren Lebensunterhalt Geld und die meisten von uns haben eine Vielzahl von Quellen, aus denen sie ihre monatlichen Einkünfte beziehen. Immer wieder kommt es jedoch vor, dass das Geld, welches man bekommt, nicht ausreicht. Vielleicht reicht das Gehalt der Eltern nicht aus, um einen zu unterstützen, oder der Bafög Antrag wurde abgelehnt. Dann heißt es für viele: Arbeiten gehen.}
{   30 CP im Semester entsprechen in etwa einer Arbeitswoche von 40 Stunden. Da
    bleibt für viele gar nicht mal viel Zeit, auch noch einer geregelten
    Beschäftigung nachzugehen. Das erste, was ihr für euch entscheiden solltet,
    ist, wie viel Zeit ihr für eine Nebenbeschäftigung aufwenden könnt und
    wollt. Für die Uni, und ganz wichtig auch für den Staat, ist eure
    Hauptbeschäftigung das Studium, was rechtliche Konsequenzen hat. Mehr als
    20 Stunden die Woche während der Vorlesungszeit solltet ihr auf keinen Fall
    arbeiten, da ihr sonst sozialversicherungspflichtig werdet. Auch die
    Freibetragsgrenze (2016 waren es 8.472 Euro) solltet ihr im Auge behalten,
    da ihr sonst Steuern zahlen müsst.\\ \vfill \columnbreak Gut zusammengefasst bekommt ihr diese
    Informationen unter \url{https://www.studentenwerke.de/de/werkstudentenprivileg}. \\

    Prinzipiell sind eurer Kreativität keine Grenzen gesetzt, welchen Beschäftigungen ihr nachgehen wollt. Ob Nachtschicht an der Tankstelle, Werkstudent*in bei einem Softwareunternehmen oder selbstständiger Web-Designer, solange ihr die oben genannten Rahmenbedingungen einhaltet, könnt ihr arbeiten, wo ihr wollt\footnote{An Gesetze müsst ihr euch natürlich trotzdem halten, Drogen dealen ist nach wie vor verboten ;-)}. Allerdings unterscheiden sich unterschiedliche Formen der Arbeit voneinander! Die erste und wichtigste Unterscheidung ist, ob ihr angestellt oder selbstständig arbeiten wollt. \\

    Als Selbstständige seid ihr ziemlich frei, was die Gestaltung eurer
    Tätigkeit angeht, aber auf euch selbst gestellt. Als unabhängige
    Dienstleister könnt ihr eure eigene Firma gründen\footnote{Die  Gründung
        einer eigenen Firma ist ein Unterfangen, welches locker einen eigenen
        Artikel wert wäre. Um hier Platz zu sparen, verweise ich auf\\
        \url{https://www.existenzgruender-jungunternehmer.de/p/gruendung/studenten.html}
        für weitere Informationen.} und selbstständig eure Dienstleistung oder
    Produkte anbieten. Diese Freiheit heißt aber auch, dass ihr eine große
    Unsicherheit habt und viel Arbeit selbst machen müsst, denn Kunden
    anzuwerben und Preise zu verhandeln ist dann eure
    Aufgabe.\\\vfill\columnbreak Als Informatiker*innen gibt es aber sehr viele Nischen, in denen ihr euch sehr gut selbstständig machen könnt. Ob als Web Developer oder als Programmierer, die Grenzen setzen hier nur eure Zeit und euer unternehmerisches Talent. \\

    Als Angestellte seid ihr deutlich weniger frei, da ihr Arbeitnehmer*innen bei einem Arbeitgeber seid. Dafür könnt ihr dann auf in festes, vorhersehbares Gehalt freuen und müsst euch nicht darum kümmern, dass ihr Kunden bekommt\footnote{Es sei denn, ihr werdet zur Kundenwerbung eingestellt ;-)}. Außerdem seid ihr vor plötzlicher Kündigung geschützt und habt ein Anrecht auf Urlaub und Pausen. Euer größtes ''Problem'' dabei ist, einen passenden Arbeitsplatz zu finden. Dabei könnt ihr auf Ressourcen im Internet zurückgreifen, wie z.B. die Webseiten der Uni und das D120-Forum. Darüber hinaus lohnt es sich die Augen und Ohren offen zu halten. Oft suchen Freunde, Verwandte und Bekannte nach Mitarbeiter*innen oder wissen von interessanten Stellen. Und egal wie man zum berüchtigten ''Vitamin B'' steht, oft ist dies der beste Weg zu einem lukrativen und interessanten Job. \\

    Für Studierende steht ein Arbeitgeber natürlich immer offen: die Uni.
    Stellen als Hilfswissenschaftliche Mitarbeiter*innen (HiWis) sind
    Arbeitnehmerstellen. Eure Arbeitgeber*innen sind dann die verschiedenen
    Bereiche an der Uni, oft einzelne Fachgruppen oder die Fachbereiche.
    HiWi-Jobs sind zwar meist weniger gut bezahlt, als vergleichbare Stellen in
    der Wirtschaft (9,50\,\EURtm ~für Bachelor-, 11,50\,\EURtm ~für Master-Studierende), aber oftmals werden deutlich mehr Stunden bezahlt als man eigentlich leistet. Der größte Vorteil ist, dass die Arbeitgeber*innen an der Universität oft Verständnis für die Eigenheiten des Studiums haben. Hier erhaltet ihr oft mehr Nachsicht, wenn ihr wegen einer anstehenden Klausur weniger arbeitet, als in der freien Wirtschaft, wo die Produktivität des Unternehmens eindeutig Vorrang vor eurem Studium hat. \\

    Wichtig bei der Arbeitgeberwahl ist es, dass ihr euch darüber im Klaren seid, wer an eurer Arbeit mit verdient. An der Uni werben oft Vermittlungsfirmen, die euch an ihre eigenen Kunden weiterverleihen. Die Verträge mögen gut aussehen, aber oft fällt ein großer Teil eures eigentlichen Lohns an die Vermittlung. Sobald mehrere dieser Firmen hintereinander hängen, ihr also nur über mehrere Ecken weiter geleitet werdet, solltet ihr aufpassen. Es ist fast immer besser, direkt bei eurem entgültigen Arbeitgeber angestellt zu sein, als über Vermittlungsfirmen verliehen zu werden.Eine Außnahme bilden hier einige Consulting-Firmen, die euch ermöglichen, an interessante Projekte zu kommen, die ihr sonst niemals gesehen hättet. Die Entscheidung, ob die Vermittlungs- oder Consulting-Firma ihr Geld wert ist, müsst ihr je nach Fall selbst treffen. Wie an so vielen Stellen ist hier ein gesunder Menschenverstand gefragt. \\

    Die nächste wichtige Frage ist, wann ihr arbeiten wollt. Hier macht es vor allem einen Unterschied, ob ihr in den Semesterferien eher am Stück oder während der Vorlesungszeit nebenbei arbeiten wollt. Beides hat Vor- und Nachteile, aus persöhnlicher Erfahrung kann ich allerdings sagen, dass man in der Vorlesungsfreien Zeit zwar an sich viel Freiraum hat, aber ein großer Job das Lernen schnell in den Hintergrund rücken lässt. Seid also vorsichtig, wenn ihr euch für ein größeres Praktikum verpflichtet, bevor alle Klausuren geschrieben sind. Während der Vorlesungszeit habt ihr dafür natürlich auch andere Verpflichtungen, Vorlesungen, Übungen und andere Abgaben. \\

    Diese Übersicht soll nur einen schnellen Überblick über die wichtigsten
    Themen bei der Arbeitswahl bieten. Sollten euch bestimmte Themen genauer
    interessieren, schreibt uns eine kurze Mail an
    \texttt{\href{mailto:inforz@d120.de}{inforz@d120.de}}.

}
{Claas Voelcker}
\vfill
\bildmitunterschrift{grafik/comics/wizards.png}{width=11cm}{}{http://abstrusegoose.com/253}
