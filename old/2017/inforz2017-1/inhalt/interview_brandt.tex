\artikel{Interview mit Ulrike Brandt}
{Ulrike Brandt war lange eine Konstante am Fachbereich Informatik. Viele Studierende kennen sie und haben bei ihr Antworten auf viele Fragen über die Formalia des Studiums gefunden. Wir wollten von Frau Brandt wissen, was sie aus den Jahren an der Uni mitnimmt und was ihre Pläne für den Ruhestand sind.
}
{
    \subsection*{Sie sind ja inzwischen eine ganze Weile hier, aber war denn die Informatik immer Ihr Ziel?}
    Nein, eigentlich wollte ich in die Naturwissenschaften und so was wie Mathematik oder Chemie studieren.
    Als ich mir hier in Darmstadt einen Studiengang in Richtung chemische Verfahrenstechnik angeschaut habe, bin ich quasi über die Informatik gestolpert. Da habe ich mir gedacht, ich probiere das mal aus. Es klang halt ganz interessant.
    \subsection*{Wie war denn das Informatikstudium zu ihrer Zeit?}
    Bei mir war es so, dass ich nach meinem Vordiplom nicht wusste was die Informatik eigentlich ist. Damals gab es noch keinen roten Faden. Es gab zwar sehr interessante Vorlesungen, von zum Beispiel Robert Piloty, Hans-Jürgen Hoffman und Hermann Walter, aber die Zusammenhänge sind mir nicht immer ganz klar geworden. Wäre nicht die theoretische Informatik gewesen, die mir von Anfang an sofort Spaß gemacht hat, hätte ich hier wahrscheinlich die Informatik an den Nagel gehängt.
    \subsection*{Sie sind nach dem Studium beziehungsweise nach Ihrer Habilitation in die Industrie gegangen. Was haben Sie da getan?}
    Ich muss sagen, ich war nur ungefähr drei Jahre weg. In denen habe ich für die Höchst AG mathematisch-technische Assistenten ausgebildet. Danach habe ich kurz in der Firma meines Vaters als Geschäftsführerin gearbeitet, bis ich glücklicherweise wieder hierher zurückkommen konnte.
    \subsection*{Wollten sie denn schon immer an der Uni arbeiten oder ergab sich das einfach so?}
    Ja, da der Bereich, welcher mich am meisten interessierte, schon immer die theoretische Informatik war und dies auch der Bereich ist, in dem ich damals mehr machen wollte. Daher war die Universität eigentlich schon immer mein Ziel. Hier ist man dafür am besten aufgehoben. Es war Glück, dass das alles so gut geklappt hat und ich hierher zurückkehren konnte, ansonsten wäre ich halt in der Industrie geblieben.
    \subsection*{Sie haben ja damals noch Diplom studiert und kennen deswegen sowohl dieses als auch das neuerer Bachelor/Master System. Was ist denn ihre Meinung zum neuen System bzw. zu dem Wechsel weg vom Diplom?}
    Aus meiner ganz persönlichen Sicht, wenn ich das entscheiden könnte, wäre ich wahrscheinlich beim Diplom geblieben. Ich bin der Meinung, dass das Bachelor/Master System stärker reguliert ist und als solches einige Freiheiten verloren gehen, die es damals gab. Auch wenn der Master bei uns schon sehr viele Wahlmöglichkeiten hat und damit in die richtige Richtung geht. Ebenso gab es im Diplom weniger Prüfungen, zum Beispiel vor dem Vordiplom nur 5 Prüfung und einige "`Scheine"', die man erbringen musste. Vom Umfang her würde ich sagen, dass eine solche Prüfung von damals den Aufwand von zwei Fächern heute widerspiegelt, aber trotzdem ist das dann immer noch sehr viel weniger als dies heute der Fall ist. Gleiches gilt natürlich auch für das Hauptdiplom.
    \subsubsection*{Was würden Sie sagen, waren ihre schönsten Erlebnisse in knapp 25 Jahren an der TU?}
    Der Abschied war sehr nett, den fand ich sehr herzlich.
    \subsubsection*{Und zur Gegenfrage, was würden Sie denn am liebsten wieder vergessen?}
    Also so schlecht, dass ich es vergessen möchte ist es nicht, aber was ich sehr bedauert habe, ist die Abschaffung der Theorie. Sie ist zwar nicht komplett abgeschafft und in vielen Veranstaltungen integriert, allerdings wurde sie schon sehr stark zusammengekürzt. Dadurch das Aussagen und Prädikatenlogik (APL) und Automaten, formale Sprachen und Entscheidbarkeit (AFSE) sehr früh im Studium kommen, sind diese Veranstaltungen nicht besonders tief. Viele Bereiche, die ich als sehr wichtig einschätze, bleiben leider etwas auf der Strecke. Das bedauere ich doch schon sehr.
    \subsubsection*{Wir haben Sie bereits vor 9neun Jahren im Rahmen der Fachstudienberatung interviewt (Ausgabe: Oktober 08). Was hat sich an Ihrer Arbeit in dieser Zeit verändert?}
    Ich mache inzwischen weniger Studienberatung, das hat Tim [Neubacher, Anm. der Redaktion] zu großen Teilen übernommen. Ich habe mich seitdem verstärkt um die Weiterentwicklung des Fachbereichs und der Studiengänge gekümmert. So war ich beispielsweise ziemlich stark in der Reakkreditierung vor zwei Jahren involviert.
    \subsubsection*{Damals (vor neun Jahren) sagte man uns: "`Jeder Student sollte seine Prüfungs- und Studienordnung mal gelesen haben!"'. Hat sich daran etwas geändert oder vertreten sie immer noch diese Meinung?}
    Ja, auf jeden Fall! Man sollte sie zumindest mal gelesen haben und die rudimentären Sachen mitgenommen haben. Man sollte auch immer wissen, an wen man sich wenden kann, sollte man mal einen Rat brauchen.
    \subsubsection*{Was würden sie neuen Studierenden an dieser Uni raten?}
    Wichtig ist, in einer Gruppe zu arbeiten und sich nicht zu isolieren. Auch sollte man sich immer über sein Studium informieren und sich notfalls helfen lassen. Einfach "`vor sich hin zu studieren"', ohne sich damit auseinandergesetzt zu haben ist einfach nicht gut. Nehmt Angebote wie die Ophase war und geht lieber einmal zu oft, als einmal zu wenig, zur Studienberatung.
    \subsubsection*{Sie saßen ziemlich lange in Gremien wie dem LuSt. Wie fanden sie ihre Arbeit dort, konnten Sie so mitwirken wie sie es sich gewünscht hätten?}
    Teils, teils. Das ist halt so. Im Ganzen bin ich aber recht zufrieden. Während der Reakkreditierung 2003 war der Kommission der Anteil der Mathematik in unserem Studiengang zu viel, weshalb wir die Mathematik  mit der Elektrotechnik Mathematik zusammengelegt haben. Das fand ich damals nicht so gut, hier hätte ich mir gewünscht, dass es anders gelaufen wäre. Aber alles in allem konnte ich vieles einbringen und bin sehr zufrieden damit.
    \subsubsection*{Sie haben sehr viel Zeit nicht nur in unserem Fachbereich, sondern insgesamt an dieser Uni verbracht. Wie hat sich denn die Uni im Laufe ihrer Zeit hier entwickelt?}
    Also ich glaube, dass die Professoren mehr unter Stress stehen, weil das Einwerben von Drittmitteln hier inzwischen eine große Rolle eingenommen hat. Was sich dadurch zum Beispiel geändert hat, ist, dass früher die Professoren, und zwar alle, im Fachbereichsrat saßen und sich mehr in die Lehre als Ganzes eingebracht haben. Damals waren wir auch nicht so viele Professoren. Ich war bereits in der Fachschaft, damals nannten wir uns noch Basisgruppe, und saß als Vertreterin lange und auch oft im Fachbereichsrat. Und das Engagement jedes einzelnen Professors sich einzubringen war viel mehr vorhanden. Heute ist es größtenteils das (Studien-) Dekanat und einige wenige Professoren, die sich als Gruppe versuchen einzubringen. Ich wünsche mir, das sich dies irgendwann wieder ändert und mehr zurück zum Zustand wie es früher einmal war zurückgeht.
    \subsubsection*{Wenn Sie irgendwas an diesem Fachbereich ändern könnten, egal was, was wäre das?}
    Die stärkere Beteiligung und mehr Interesse an Lehre, denn nur so können wir den Fachbereich besser gestalten.
    \subsubsection*{Haben sie eine konkrete Idee, wie man die Professoren mehr einbeziehen könnte?}
    Ich glaube, auch wenn das jetzt etwas pessimistisch klingt, es ist aktuell hoffnungslos. Ich sehe auch die Problematik nicht am mangelnden Interesse mehr an der Lehre mitzuwirken, sondern mehr darin, dass sie keine Zeit haben sich mehr einzubringen.
    \subsubsection*{Könnten sie sich vorstellen, dass eine Art "`Vollversammlung aller Professoren, Mitarbeitern und der Fachschaft"' helfen würde? }
    Man könnte das mal versuchen, aber ob das zum gewünschten Ergebnis führt weiß ich nicht. Vielleicht wäre es aber eine Idee, so etwas mal zu machen und dann zu evaluieren.
    \subsubsection*{Sie haben ja demnächst etwas mehr Zeit als vorher. Wissen sie schon, was sie mit Ihrer neuen Freizeit anfangen werden?}
    Also teils, teils. Erstmal mache ich ein bisschen Urlaub. Ich habe von meinen Kollegen zum Abschied einen Gutschein für einen Garten geschenkt bekommen und werde diesem wohl demnächst etwas Zeit zu Gute kommen lassen. Ich mach erstmal das, was mir Spaß macht. Dann sind in letzter Zeit einige Dinge liegen geblieben, denen ich mich wieder etwas mehr widmen möchte. Zum Beispiel kam meine wissenschaftliche Arbeit etwas zu kurz. Ich habe ein fast fertiges Paper, dass ich demnächst angehen möchte. Vielleicht will ich auch noch eine Ausbildung in Richtung Psychologie machen, dass hat mich schon immer interessiert. Ich hatte sogar überlegt es zu studieren.
    \subsubsection*{Was nehmen sie aus 25 Jahren Uni mit und wie hat es sie verändert?}
    Man braucht eine ganze Menge Ausdauer und Zähigkeit, wenn man hier arbeitet. Manchmal laufen die Dinge nicht immer einfach oder so wie man es sich vorstellt. Das braucht Kraft. Ich habe hier gelernt, dass man an bestimmten Stellen halt nicht einfach mit dem Kopf durch die Wand kann.
    \subsubsection*{Windows, Mac oder Linux}
    Ich habe mich an Windows gewöhnt und nie etwas anderes gemacht, also bleibe ich erstmal dabei.
    \subsubsection*{Was ist für sie die Zahl 42}
    Das ist das Geburtsjahr von meinem Ehemann.2

}{Heiko Carrasco \& Jannis Blüml}

\vfill
\bildmitunterschrift{grafik/comics/fun_with_statistics2.png}{width=11cm}{}{http://abstrusegoose.com/55}
