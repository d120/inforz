%Comic am Ende oder Anfang von anderem Artikel auf gleicher Seite
\artikelohnevorspann{Podcasts für Informatiker*innen}
{Die meisten Menschen stecken, wenn sie nach der Arbeit die Uni verlassen, sich erstmal Kophörer rein. Ich bin da keine Ausnahme. Aber während die meisten Anderen dann direkt in den
    Musikabspieler ihrer Wahl wechseln, schalte ich meine Podcastapp ein und verbringe die nächsten Minuten oder auch Stunden damit, dem Geplapper anderer Leute zuzuhören.
    Aber was ist denn ein Podcast? Im Prinzip ist es meist eine oder mehrere Personen, die sich über ein Thema unterhalten. Beispielsweise gibt es Podcasts über Filme und Serien oder welche, die sich mit Technologie und Wissenschaft beschäftigen. Dabei gibt es innerhalb der Casts mehrere Episoden, ähnlich einer Serie. Diese sind nicht unbedingt zusammenhängend, man kann also auch in der Mitte einsteigen. Die Länge variiert, liegt aber oft im Bereich von einer Stunde und mehr.
    Mit der Zeit haben sich im Internet eine Fülle an Podcasts angesammelt, sodass die Auswahl am Anfang recht schwer fällt. Daher möchte ich hier eine kleine Auswahl meiner abonnierten Podcasts vorstellen. Die Liste ist vor allem wissenschaftlich/technisch orientiert. Alle vorgestellten Casts sind kostenlos, über Spenden (z.B. mit flattr)
    freuen sich die meisten. Sie können über einen beliebigen Podcatcher (Podcastapp) bezogen werden. Unter Android ist zum Beispiel PodcastAddict sehr zu empfehlen.
    \vfill\columnbreak
    ~\vfill
    \textbf{Name: }Binärgewitter\\
    \textbf{Sprache: }Deutsch\\
    \textbf{Thema: }OpenSource- und Techniknews\\
    \textbf{Beschreibung: }In Binärgewitter treffen sich ein paar alte Hasen der Podcastszene und quatschen über das aktuelle Geschehen in der Welt der Technik.
    Aktuelle News werden vorgestellt und kommentiert. Dabei nehmen sich die Macher selbst nicht unbedingt ernst und oft enden Diskussionen in dem ein oder anderen
    Schlagabtausch.
    \vfill\vfill
    \textbf{Name: }Chaosradio\\
    \textbf{Sprache: }Deutsch\\
    \textbf{Thema: }Computer und Gesellschaft, Technik, CCC\\
    \textbf{Beschreibung: }Das Chaosradio kommt jeden Monat raus. Vom CCC Berlin organisiert, werden vor allem Themen welche die Hackerszene betreffen besprochen. Das kann zum Beispiel Netzpolitik, Biohacking oder digitales Geld sein. Dazu werden oft Experten eingeladen, zum Beispiel Leute von Netzpolitik.org oder der FSFE\footnote{\textit{Free Software Foundation Europe}}. Abgerundet wird die Sache von der guten Moderation, durch einen professionellen Radiomoderatoren.
    \vfill~\\\columnbreak\\
    \textbf{Name: }Serial\\
    \textbf{Sprache: }Englisch\\
    \textbf{Thema: }Die Geschichte eines Mordes\\
    \textbf{Beschreibung: }In Serial erzählt Sarah Koenig die wahre Geschichte über den Mord an Hae Min Lee im Jahre 1999 in Baltimore, USA. Im Laufe der Geschichte deckt die Reporterin Ungereimtheiten und
    Fehler während der Aufklärung und Verhandlung des Falls auf. Jede Episode befasst sich mit einem anderen Teilaspekt des Falls, wodurch sich nach und nach ein Gesamtbild zusammenpuzzelt.
    \vfill\columnbreak
    \textbf{Name: }Robot Congress\\
    \textbf{Sprache: }Englisch\\
    \textbf{Thema: }Technologie und Recht\\
    \textbf{Beschreibung: } Dieser Podcast hat sich der juristischen Einschätzung von technischen Themen verschrieben. Mit Ryan Morrison, besser bekannt als der Video Game Attorney, ist auch jemand vom Fach an Bord. Trockene juristischen Texte muss man aber nicht fürchten, die Themen werden mit Humor angegangen. Generell bekommt man aber leider nur die amerikanische Seite des Rechts zu sehen.}{Heiko Carrasco}
