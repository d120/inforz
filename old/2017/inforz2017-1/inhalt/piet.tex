\artikel{Esoterische Programmiersprachen - Teil 1: Piet}
{Dies ist der erste Artikel einer Reihe über esoterische Programmiersprachen.\\
    Aber was ist das eigentlich?\\
    Esoterische
    Programmiersprachen sind nicht zum praktischen Gebrauch bestimmt, die Ziele
    mit denen diese entwickelt werden sind aber recht unterschiedlich.\\
    Bei der
    Programmiersprache \glqq{}Brainfuck\grqq{}, die vermutlich recht viele kennen, war die Prämisse
    eine turingvollständige Sprache zu entwickeln, die mit einem möglichst kleinen
    Compiler auskommt. \glqq{}Whitespace\grqq{} degegen soll beim Ausdruck des
    Sourcecodes möglichst wenig Tinte bzw. Toner verbrauchen.}
{Die Programmiersprache, die ich heute vorstelle heißt \glqq{}Piet\grqq{}, benannt nach dem
    Künstler Piet Mondrian. Ihr Ziel ist es gängigen Mondrian Bildern (s.
    Abb 1) möglichst nahe zu kommen.
    \bildmitunterschrift{grafik/piet_bild.jpg}{width=.75\columnwidth}{\small{}Composition
        with Red, Yellow and Blue}{Piet Mondrian}
    Ihr fragt euch vielleicht, wie eine Programmiersprache (bzw. ein in ihr geschriebenes Programm) einem abstrakten
    Gemälde ähneln soll.
    Nun, in Piet geschriebene Programme sind ebenfalls Bilder.
    Abbildung 2 zeigt ein Piet Programm, welches das Wort \glqq{}Piet\grqq{} ausgibt und auf
    dem Umschlag ist ein Programm abgebildet, welches das Wort \glqq{}Inforz\grqq{}
    erzeugt.
    \bildmitunterschrift{grafik/piet-program.png}{width=.75\columnwidth}{\glqq{}Piet\grqq}{Thomas
        Schoch}~\\

    Im nachfolgenden werde ich die grundlegenden Konzepte erläutern:\\

    Ein Programm wird aus den Farben Rot, Gelb, Grün, Cyan, Blau und Magenta
    in jeweils drei Helligkeitsstufen Hell, Normal und Dunkel, sowie aus den
    Farben Weiß und Schwarz zusammengesetzt.\\

    Ein zusammenhängender Block von Pixeln einer Farbe wird \glqq{}Colour Block\grqq{}
    genannt.\\

    Zur Programm Ausführung werden zwei Zeiger verwendet: Der \glqq{}Direction Pointer\grqq{}
    (DP)
    zeigt nach oben, unten, links oder rechts, während der \glqq{}Codel Chooser\grqq{} (CC) nach
    links oder nach rechts zeigt.
    Die Ausführung beginnt im Block in der oberen linken Ecke mit DP=rechts und
    CC=links
    Für einen Ausführungsschritt wird nun zunächst die Kante des aktuellen
    Blocks gewählt, welche sich am weitesten in Richtung des DP befindet.
    Nun wird das Pixel ausgewählt, dass sich an dieser Kante am weitesten in
    richtung CC (\grqq{}aus sicht des DP\grqq{}) befindet von dort wählt der Interpreter den
    anliegenden Block, führt die Instruktion dieses Überganges aus und führt den
    nächsten Schritt aus.\\

    Instruktionen sind die Übergänge zwischen zwei verschiedenfarbigen Blöcken.
    So wird zum Beispiel bei einem Übergang zu einer um eine Stufe dunkleren
    Farbe (im Kontext von Piet ist \glqq{}Hell\grqq{} eine Stufe dunkler als \glqq{}Dunkel\grqq{}, so dass
    ein Zyklus entsteht) die Größe des Blockes von dem der Interpreter kommt auf den
    Stack gepusht.
    Bei einem Übergang zur nächsten Farbe (Reihenfolge wie oben, nach Magenta
    kommt wieder Rot) werden die letzten beiden Werte vom Stack genommen addiert
    und die Summe wieder auf den Stack gepusht.\vspace{5em}\\

    Wenn der Interpreter bei der Ausführung einen schwarzen Block betreten
    müsste, oder wenn das Bild an dieser Seite zu Ende ist wird statt den neuen
    Block auszuwählen zunächst der CC geändert und vom aktuellen Block aus erneut
    versucht einen Schritt auszuführen. Falls dies auch schief geht, wird der DP
    im Urzeigersinn gedreht. Diese beiden Schritte werden so oft wiederholt, bis
    die Programmusführung fortgesetzt werden kann, oder bis der Interpreter
    wieder im Ursprungszustand ist (nach insgesammt 8 Versuchen). In diesem Fall
    wird das Programm beendet.\\

    Wenn der Interpreter einen weißen Block betritt, so wird dieser einfach
    passiert und die Ausführung in gerader Linie am nächsten farbigen Block
    fortgesetzt. Weder das Betreten noch das Verlassen des weißen Blockes führt
    irgendeine Instruktion aus.\\

    ~\\~\\
    Insgesammt ist Piet eine recht amüsante Abwechslung. Allerdings braucht man
    schon für die Entwicklung kleiner Programme unverhältnissmäßig viel Zeit
    weshalb die Programmiersprache für den alltäglichen Gebrauch eher ungeeignet
    ist.\\
    Wer sich dennoch mal an der Programmierung in Piet versuchen möchte dem
    empfehle ich sich auf \url{http://www.dangermouse.net/esoteric/piet.html}
    umzuschauen.

}
{Fabian Franke}
