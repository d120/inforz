\artikelohnevorspann{Warum die 42}
{
    \begin{minipage}[t]{\textwidth}
        Weit draußen in den unerforschten Einöden eines total aus der Mode gekommenen Ausläufers des westlichen Spiralarms der Galaxis leuchtet unbeachtet eine kleine gelbe Sonne. Um sie kreist in einer Entfernung von ungefähr achtundneunzig Millionen Meilen ein absolut unbedeutender, kleiner blaugrüner Planet, dessen vom Affen stammende Bioformen so erstaunlich primitiv sind, daß sie Digitaluhren noch immer für eine unwahrscheinlich tolle Erfindung halten. \\
        Dieser Planet hat – oder besser gesagt, hatte – ein Problem: die meisten seiner Bewohner waren fast immer unglücklich. Zur Lösung dieses Problems wurden viele Vorschläge gemacht, aber die drehten sich meistens um das Hin und Her kleiner bedruckter Papierscheinchen, und das ist einfach drollig, weil es im großen und ganzen ja nicht die kleinen bedruckten Papierscheinchen waren, die sich unglücklich fühlten. \\
        Und so blieb das Problem bestehen. Vielen Leuten ging es schlecht, den meisten sogar miserabel, selbst denen mit Digitaluhren.
        Viele kamen allmählich zu der Überzeugung, einen großen Fehler gemacht zu haben, als sie von den Bäumen heruntergekommen waren. Und einige sagten, schon die Bäume seien ein Holzweg gewesen, die Ozeane hätte man niemals verlassen dürfen.
        Und eines Donnerstags dann, fast zweitausend Jahre, nachdem ein Mann an einen Baumstamm genagelt worden war, weil er gesagt hatte, wie phantastisch er sich das vorstelle, wenn die Leute zur Abwechslung mal nett zueinander wären, kam ein Mädchen, das ganz allein in einem kleinen Café in Rickmansworth saß, plötzlich auf den Trichter, was die ganze Zeit so schiefgelaufen war, und sie wußte endlich, wie die Welt gut und glücklich werden könnte. Diesmal hatte sie sich nicht getäuscht, es würde funktionieren, und niemand würde dafür an irgendwas genagelt werden. \\
        Nur brach traurigerweise, ehe sie ans Telefon gehen und jemandem davon erzählen konnte, eine furchtbar dumme Katastrophe herein, und ihre Idee ging für immer verloren. Das hier ist nicht die Geschichte dieses Mädchens. Es ist die Geschichte dieser furchtbar dummen Katastrophe und einiger ihrer Folgen. \\

        Zitat der ersten Seiten des Werkes „Per Anhalter durch die Galaxis“
    \end{minipage}



    \clearpage


    Diese Worte waren für viele von uns der erste Berührpunkt mit einer Buchreihe, die meiner Meinung nach zu den besten Science Fiction Reihen gehört. Und dies verdankt er nicht gerade seinem wissenschaftlichen Bezug, sondern seinem Humor, der oft schon ins Groteske ausschlägt. Auch spiegelt es in vielen Aspekten den Menschen und sein Handeln wieder beziehungsweise parodiert diese. Dabei geht der Autor auf den Sinn des Lebens ein und stellt feste Prinzipien und Motive in Frage. Der Autor oder viel mehr sein Werk bleibt allerdings immer fantastisch kuntabunt, was für mich schlussendlich die Reihe ausmacht. \\

    Für die meisten Informatiker gehört die Buchreihe zu den Klassikern und gehören gelesen, warum die Reihe gerade bei uns so beliebt ist, kann ich nicht beantworten. Vielleicht einfach ein Beispiel für Zur richtigen Zeit am richtigen Ort, vielleicht auch mehr. \\

    Nach dem Anhalter ist die Zahl 42 die Antwort auf die Frage nach nach dem Leben, dem Universum und dem ganzen Rest. Ich meine so hat es ja ein sehr sehr großer Rechner in sehr sehr langer Zeit berechnet.
    Zugegebenermaßen kann man damit nicht unbedingt viel anfangen, aber das ist sicherlich nicht die Schuld des Computers, sondern nur die des Nutzers, der seine Frage zu schwammig formuliert hat.
    Für uns Informatiker gehört sie jedenfalls zu den magischen Zahlen, denen man einfach nicht aus dem Weg gehen kann. Ich meine „101010“ ist halt auch was besonderes. Auch wissen wir ja alle, dass 9*6 im 13er-Stellenwertsystem 42 ergibt.

}
{Jannis Blüml}
