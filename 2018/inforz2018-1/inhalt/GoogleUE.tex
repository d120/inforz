\artikelohnevorspann{Der Zauberlehrling + Google Übersetzter }
{
    {\large \textbf{Der Zauberlehrling} \\
        Johann Wolfgang von Goethe} \\

    Hat der alte Hexenmeister \\
    Sich doch einmal wegbegeben!  \\
    Und nun sollen seine Geister \\
    Auch nach meinem Willen leben. \\
    Seine Wort' und Werke \\
    Merkt ich und den Brauch, \\
    Und mit Geistesstärke \\
    Tu ich Wunder auch. \\

    Walle! walle \\
    Manche Strecke, \\
    Daß, zum Zwecke, \\
    Wasser fließe \\
    Und mit reichem, vollem Schwalle \\
    Zu dem Bade sich ergieße. \\

    Und nun komm, du alter Besen! \\
    Nimm die schlechten Lumpenhüllen; \\
    Bist schon lange Knecht gewesen: \\
    Nun erfülle meinen Willen! \\
    Auf zwei Beinen stehe, \\
    Oben sei ein Kopf, \\
    Eile nun und gehe \\
    Mit dem Wassertopf! \\

    Walle! walle \\
    Manche Strecke, \\
    Daß, zum Zwecke, \\
    Wasser fließe \\
    Und mit reichem, vollem Schwalle \\
    Zu dem Bade sich ergieße. \\

    Seht, er läuft zum Ufer nieder, \\
    Wahrlich! ist schon an dem Flusse, \\
    Und mit Blitzesschnelle wieder \\
    Ist er hier mit raschem Gusse. \\
    Schon zum zweiten Male! \\
    Wie das Becken schwillt! \\
    Wie sich jede Schale \\
    Voll mit Wasser füllt! \\

    Stehe! stehe! \\
    Denn wir haben \\
    Deiner Gaben \\
    Vollgemessen! – \\
    Ach, ich merk es! Wehe! wehe! \\
    Hab ich doch das Wort vergessen! \\

    Ach, das Wort, worauf am Ende \\
    Er das wird, was er gewesen. \\
    Ach, er läuft und bringt behende! \\
    Wärst du doch der alte Besen! \\
    Immer neue Güsse \\
    Bringt er schnell herein, \\
    Ach! und hundert Flüsse \\
    Stürzen auf mich ein.\\

    Nein, nicht länger \\
    Kann ichs lassen; \\
    Will ihn fassen. \\
    Das ist Tücke! \\
    Ach! nun wird mir immer bänger! \\
    Welche Miene! welche Blicke! \\

    O, du Ausgeburt der Hölle! \\
    Soll das ganze Haus ersaufen? \\
    Seh ich über jede Schwelle \\
    Doch schon Wasserströme laufen. \\
    Ein verruchter Besen, \\
    Der nicht hören will! \\
    Stock, der du gewesen, \\
    Steh doch wieder still! \\

    \vfill
    \columnbreak

    Willsts am Ende \\
    Gar nicht lassen? \\
    Will dich fassen, \\
    Will dich halten \\
    Und das alte Holz behende \\
    Mit dem scharfen Beile spalten. \\

    Seht, da kommt er schleppend wieder! \\
    Wie ich mich nur auf dich werfe, \\
    Gleich, o Kobold, liegst du nieder; \\
    Krachend trifft die glatte Schärfe. \\
    Wahrlich! brav getroffen! \\
    Seht, er ist entzwei! \\
    Und nun kann ich hoffen, \\
    Und ich atme frei! \\

    Wehe! wehe! \\
    Beide Teile \\
    Stehn in Eile \\
    Schon als Knechte \\
    Völlig fertig in die Höhe! \\
    Helft mir, ach! ihr hohen Mächte! \\

    Und sie laufen! Naß und nässer. \\
    Wirds im Saal und auf den Stufen. \\
    Welch entsetzliches Gewässer! \\
    Herr und Meister! hör mich rufen! – \\
    Ach, da kommt der Meister! \\
    Herr, die Not ist groß! \\
    Die ich rief, die Geister \\
    Werd ich nun nicht los.\\

    »In die Ecke, \\
    Besen! Besen! \\
    Seids gewesen. \\
    Denn als Geister \\
    Ruft euch nur, zu seinem Zwecke, \\
    Erst hervor der alte Meister.« \\

    (1827) \\
}
{}
\clearpage
\artikelohnevorspann{}
{

{\large \textbf{Studentischer Assistent} \\
    Johann Wolfgang von Goethe} \\
{\small in der verbesserten internationalen Version nach einer Übersetzung aus dem türkisch-französisch-chinesisch-russisch-japanisch-schwedisch-armenischem \\}

Gibt es ein altes Fenster? \\
VA-T! \\
Jetzt seine Gedanken \\
Ich lebe auch in meiner Zeit. \\
Sein Versprechen und seine Arbeit \\
Ich kenne die Tradition \\
Mentale Kraft \\
Ich spiele Wunder. \\

Walla! Walle \\
Weg \\
Zu diesem Zweck \\
Wasserdurchfluss \\
Und reicher, perfekter Sattel \\
Im Badezimmer. \\

Lass jetzt den alten Räuber. \\
Werfen Sie schlechte Dinge! \\
Ein langer Diener. \\
Bitte füllen Sie meine Bewerbung gleich aus! \\
Auf beiden Beinen stehend, \\
Overhead \\
schneller \\
In einem Wasser, im Wasser. \\

Walla! Walle \\
Weg \\
Zu diesem Zweck \\
Wasserdurchflussrate \\
Und reicher, perfekter Sattel \\
Im Badezimmer. \\

Schau, es ist an der Küste. \\
In der Tat. Schon am Fluss, \\
Aber Blitzgeschwindigkeit \\
Das ist ein schneller Schuss. \\
Zweites Mal: \\
Wie sprengt man den Pool? \\
Wie alle Bomben \\
Sie ist voll Wasser. \\

Auf! Meine Güte. \\
Deshalb \\
Ihr Geschenk \\
Vollständige Messung: \\
Ich habe bemerkt: Oh! Oh! \\
Ich habe Kelly vergessen. \\

Oh, was ist das Ende dieses Wortes? \\
Er wird es tun. \\
Leider lief er und ließ sich nieder. \\
Wenn du nur altmodisch bist. \\
Immer neue Schriften \\
Moo bald verlassen \\
Oh! Zum Fluss \\
Lass mich zusammenbrechen \\

Nein, mehr \\
Ich bin gegangen \\
Er akzeptiert es.\\
Das ist ein Problem. \\
Oh! Jetzt mache ich mir mehr Sorgen \\
Mal sehen, wie gut es ist. Wie ist es? \\

Oh, geboren aus der Hölle. \\
Das Haus ist düster. \\
Ich sehe mich selbst \\
Aber das Wasser fließt schon. \\
Blot ist schlecht \\
Er will nichts davon hören. \\
Leinwand \\
Wieder!

Möchtest du das beenden? \\
Weiter \\
Ich will dich nehmen \\
Willst du dich beschränken? \\
Und altes empfindliches Holz \\
Neben dem scharfen Hut. \\

Hör zu, er rennt. \\
Wie bin ich zu dir gekommen? \\
Klicke einfach darauf, du bist ein Lügner. \\
Es wird bald zusammenbrechen. \\
In der Tat. Ich fühle mich gut. \\
Hör zu, das ist eine doppelte Wahrheit. \\
Jetzt hoffe ich \\
Ich atme frei. \\

Oh! Oh! \\
Beide Seiten auch \\
in Eile \\
Als Diener \\
In der Luft: \\
Bitte helfen Sie. Diese Kraft. \\

Sie rannten. Es ist feucht und feucht. \\
Die Hallen und Treppen sind zugänglich. \\
Fantastisches Wasser. \\
Mein Gott, hör mir zu. - \\
Oh, mein Herr ist hier. \\
Mein Gott, du musst eine wunderbare Person sein. \\
Ich suche nach einem Geist \\
Ich werde dir nicht erlauben, jetzt zu gehen. \\

"In der Kirche, \\
Bloom! Bloom! \\
Inhalt löschen? \\
Atmen \\
Ich bin für dich da, für deine Zwecke, \\
Der erste alte Besitzer. « \\



}
{Jannis Blüml}
