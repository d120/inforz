\artikel{Was kostet ein Studium?}
{Ein Studium ist, im Gegensatz zu vielen anderen Berufsausbildungen, ein Kostenfaktor. Darum sollte schon im Voraus geklärt sein, wie man sein Studium finanziert.
}
{Während des Studiums kommt einiges an Kosten auf Dich zu. Zunächst einmal fällt vor dem Beginn eines jeden Semesters der Semesterbeitrag an. Der genaue Betrag ändert sich übrigens nahezu jedes Semester (da das Semesterticket immer teurer wird), der aktuelle Semesterbeitrag wird aber zu Beginn des Rückmeldungszeitraumes bekannt gegeben. In jedem Fall kannst Du Dich darüber auch unter \footnote{\url{https://www.tu-darmstadt.de/studieren/studienorganisation/semesterbeitrag.de.jsp}} informieren.

    Für das Studium an sich war's das aber schon fast mit den Kosten. Im Informatikstudium fallen, von Stiften und Papier abgesehen, kaum weitere Materialkosten an. Einen wissenschaftlichen Taschenrechner sollte man haben und gelegentlich kann es nützlich sein, sich ein Buch zuzulegen. Das sind aber seltene Posten, weshalb sich deren Kosten (auch wenn Lehrbücher recht teuer sein können) in einem überschaubaren Rahmen halten.

    Was für einen guten Teil der Studierendenschaft jedoch noch hinzukommt, sind Wohn- und Lebenserhaltungskosten. Wahrscheinlich gehörst auch Du eher nicht zu den Glücklichen, die in Darmstadt oder Umgebung aufgewachsen sind und für die Dauer ihres Studiums im elterlichen Haushalt wohnhaft bleiben können. Falls Du also von weiter her kommst, wirst Du nur schwer darum herumkommen, Dir im Darmstädter Raum eine Bleibe zu suchen (mehr dazu im folgenden Artikel). Leider ist Darmstadt ein sehr teures Pflaster, daher solltest Du damit rechnen, für die monatliche Miete allein bereits über 300 Euro zahlen zu müssen. Für Essen und andere lebensnotwendige Anschaffungen kommen auch gerne nochmal deutlich über 100 Euro dazu.

    Wie also soll man das alles bezahlen? Schließlich ist ein Studium ja bereits eine Vollzeitbeschäftigung, nebenbei noch so viel Geld zu verdienen, dass es zum Decken der eigenen Kosten reicht, ist sehr schwierig. Ist das Studium Deine erste berufsqualifizierende Ausbildung, sind Deine Eltern eigentlich gesetzlich verpflichtet, Dich dabei finanziell zu unterstützen. Aber auch von staatlicher Seite kannst Du in der Regel (insbesondere wenn Dir Deine Eltern nicht allzu viel Finanzhilfe bieten können oder wollen) auf Hilfe hoffen.

    Die beliebteste Studienfinanzierungs\-möglichkeit ist hierbei das Bundesausbildungsförderungsgeld (BAföG \footnote{\url{http://www.bafoeg.bmbf.de}}), ein unverzinstes Darlehen von monatlich bis zu 670 Euro. Die Hälfte davon ist tatsächlich "`geschenkt"', die andere Hälfte ist zurückzuzahlen, sobald Du (nach dem Studium) genügend Geld verdienst, wobei auch dieser Betrag nicht über 10.000 Euro hinausgehen darf. Die Förderdauer läuft allerdings nur so lange, wie die Regelstudienzeit Deines Studienganges beträgt (also drei Jahre). Zum BAföG persönlich beraten werden kannst Du unter anderem beim Studentenwerk Darmstadt \footnote{\url{https://www.studentenwerkdarmstadt.de/index.php/de/studienfinanzierung}} oder auch bei der Sozial- und BAföG-Beratung des AStA \footnote{\url{https://www.asta.tu-darmstadt.de/asta/de/angebote}}.

    Ergänzend oder alternativ zum BAföG ist ein Stipendium eine weitere Möglichkeit, an Geld zur Studienfinanzierung zu kommen. Es gibt jede Menge Stiftungen von staatlichen Institutionen, Firmen oder auch Privatpersonen, die Stipendien anbieten, eine Übersicht darüber bietet u.a. \footnote{\url{https://stipendienlotse.de}}, aber auch beim Amt für Ausbildungsföderung kannst Du Dich über Stipendien informieren. Du solltest Dich allerdings rechtzeitig bewerben, da für Stipendien oftmals speziellere Auswahlverfahren durchgeführt werden. Insbesondere bei bekannteren Stiftungen musst Du z.B. oftmals ziviles Engagement (z.B. auf sozialer bzw. politischer Ebene) oder besondere Qualifikationen nachweisen können. Da deren Anforderungen (aber auch Förderungen) meist etwas geringer ausfallen, lohnt sich aber oft auch eine Anfrage bei weniger bekannten Stiftungen – zudem ist der Andrang auf diese Institutionen meist geringer.

    Falls das Geld nach Ausschöpfen dieser Optionen immer noch nicht (oder nicht mehr) reicht, führt wohl kaum noch ein Weg um die Suche nach einem Job herum. Aber auch das ist kein Grund zum Verzweifeln – sobald Du die Veranstaltungen der ersten Semester Deines Studiums abgeschlossen hast, bieten sich allein an der Uni schon zahllose Arbeitsmöglichkeiten, zum Beispiel als Übungstutor*in einer Lehrveranstaltung oder Hilfswissenschaftler*in (HiWi) in einem der Informatik-Fachgebiete. Besonders praktisch sind diese Stellen deshalb, weil Du eben direkt an der Uni arbeitest und somit nicht auch noch zwischen Arbeitgeber und Uni pendeln musst. Außerdem bieten HiWi-Jobs viele Möglichkeiten, das im Studium Gelernte zu festigen bzw. zu vertiefen, sowie auch Einblick in den universitären Lehrbetrieb zu erhalten. Aber auch andere Firmen suchen oft studentische Aushilfskräfte (Werkstudierende), insbesondere in der IT-Branche sind derartige Stellen keine Rarität. Diese haben den Vorzug, dass man darüber sehr gut Kontakte in die Industrie knüpfen kann und auch sein Wissen auch mal an sehr konkreten und realistischen Problemen testen kann. Wie bei HiWi-Stellen gilt aber auch hier, dass solche Stellen üblicherweise erst nach zwei bis drei Semestern Studium offen stehen.
}
{}

