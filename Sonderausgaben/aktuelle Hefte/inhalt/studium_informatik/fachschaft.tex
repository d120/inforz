\artikel{Die Fachschaft}
{An der einen oder anderen Stelle in diesem Heft wirst Du Dich möglicherweise gefragt haben, wer oder was denn diese Fachschaft eigentlich sein soll. Eine kurze Erklärung bekommst Du hier.%
}
{An sich steht der Begriff der Fachschaft für alle in einem Fach (in Deinem Fall also der Informatik) eingeschriebenen Studierenden. Üblicherweise ist aber nur eine deutlich kleinere Gruppe daraus gemeint, wenn von der Fachschaft gesprochen wird – korrekterweise müsste man hier von der aktiven Fachschaft sprechen. Die aktive Fachschaft besteht aus Studierenden, die sich neben dem Studium noch ehrenamtlich am Fachbereich engagieren und nach Möglichkeit die Interessen aller Informatik-Studierenden gegenüber Professoren und anderen Mitarbeitern vertreten. Tatsächlich sitzen gewählte Fachschaftler*innen sogar in einigen Gremien am Fachbereich und haben somit hochschulpolitisches Stimmgewicht.

Die aktive Fachschaft ist aber keine rein politische Institution. Auch jede Menge anderer Freiwilligenarbeit wird von ihr organisiert und durchgeführt – darunter die Orientierungsphase für Erstsemester, die Universitätserfahrung für Schüler*innen, das jährliche Sommerfest der Informatik und auch verschiedene kleinere Aktivitäten wie z.B. die GnoM-Gesellschaftsspieleabende oder die Games-Gruppe. Auch die Informatikerzeitung Inforz (wie auch diese Sonderausgabe, in der Du gerade liest) wird von Fachschaftler*innen geschrieben, gesetzt und herausgegeben.

Wenn Du mehr über die Fachschaft wissen willst, kannst Du Dich auch vor dem Studium schon informieren, zum Beispiel über die Website \url{www.D120.de}, benannt nach der Nummer des Fachschaftsraumes im Informatik-Gebäude. Falls Du vorm Studienbeginn bereits einmal in Darmstadt sein solltest und Fragen zum Studium hast (oder einfach mal mit Leuten quatschen willst, die schon eine Weile dort studieren), kannst Du entsprechend auch gerne im Raum D120 selbst vorbeischauen. Wo Du diesen auf dem Uni-Campus findest, kannst Du der Karte auf der Rückseite dieses Heftes entnehmen.
}
{}

\vfill
\bildmitunterschrift{../grafik/artikel/d120}{width=\textwidth}{}{}
\vfill

\newpage