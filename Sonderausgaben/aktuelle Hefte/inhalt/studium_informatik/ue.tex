\artikel{Einen Tag lang Student(in) sein}
{Sie gehen in Vorlesungen. Sie sitzen in Übungen. Sie halten Seminare. Sie erfüllen Praktika. Sie schreiben Hausaufgaben und Klausuren. Man bekommt doch so einiges mit, wenn man als Schüler mal kurz mit Studierenden spricht. Doch was ist das eigentlich alles? Von was reden sie genau? Und ist es das Richtige für mich?
}
{"`Bei mir in der Schule gibt es doch gar nicht so viele Sachen. Okay, von Hausaufgaben kann auch ich ein Lied singen. Aber was ist mit Vorlesungen? Ist es so etwas wie unser Unterricht? Oder doch etwas komplett anderes? Und was ist mit dem Informatikstudium? Ist es das, was ich machen möchte? Nur zu blöd, dass ich so etwas nicht vorher entscheiden kann."'



    Fühlst Du Dich genauso? Möchtest Du gerne wissen, ob die Universität, besonders natürlich auch das Informatikstudium das Richtige für Dich ist?

    Dann haben wir etwas für Dich, was Dir hoffentlich bei Deiner Entscheidung hilft. Eine Universität ist eine Welt für sich und unterscheidet sich in vielen Sachen von der Schule.

    Um einen Eindruck davon zu bekommen, haben wir eine Möglichkeit für interessierte Schülerinnen und Schüler eingerichtet, für einen Tag an die Universität zu kommen und sich hier die Abläufe genauer anzuschauen. Zusammen mit einer Studentin oder einem Studenten gehst Du in seine Vorlesungen, Übungen, Seminare und natürlich auch in die Mensa. Du lernst alles kennen, kannst Löcher in den Bauch fragen und Dir ein Bild machen, wie die Universität und das Informatikstudium so ist.

    Die Termine sind während der Vorlesungszeiten, denn nur wenn Vorlesungen stattfinden, kannst Du auch welche besuchen. Die Vorlesungszeit geht im Sommersemester in der Regel von Mitte April bis Mitte Juli und im Wintersemester von Mitte Oktober bis Anfang Februar.

    Falls Du Interesse bekommen hast, dann melde Dich unter \url{www.d120.de/ue} an.
    \bildmitunterschrift{../grafik/artikel/qr_ue}{width=3cm}{}{}
}{}

\bildmitunterschrift{../grafik/wesen/wesen_ue}{width=6cm}{}{}

\newpage
