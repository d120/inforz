\artikel{Vorwort des Dekans}{Liebe Schülerinnen und Schüler,}{
    zunächst einmal freue ich mich über Ihr Interesse an einem Informatik-Studium, weil ich mich über jedes Interesse junger Menschen an diesem Fach freue. Sind Sie noch unsicher, ob das für Sie das Richtige ist? Dann kann ich vielleicht mit diesem einleitenden Text ein klein wenig dazu beitragen, dass Sie eine informierte Entscheidung treffen können, die sich nicht von den kursierenden Halbwahrheiten leiten lässt. Dabei hoffe ich, dass Sie sich für uns entscheiden, denn ich selbst habe meine Entscheidung für ein Uni-Studium der Informatik und für dieses Fach seit meinem Studienbeginn 1976 -- ja, vor 40 Jahren! -- nie bereut. Ich will Sie aber nicht überreden, sondern meine ehrliche Meinung mit Ihnen teilen.

    Eines wissen Sie sicherlich: die Informatik erfasst endgültig alle Lebensbereiche, sie ist in jeder Hinsicht in der Mitte der Gesellschaft angekommen. 2020 wird so gut wie jeder Weltbürger, von Peru bis zur Mongolei, per Mobilgerät online sein. In der Dritten Welt werden noch schneller als bei uns Apps genutzt, die zentrale Aspekte des Lebens betreffen: in etlichen afrikanischen und asiatischen Ländern hat das Handy bspw. die Geldbörse ersetzt -- mit zwei Effekten: (a) die Rand- und Landbevölkerung kann viel besser am Wirtschaftsgeschehen teilnehmen, und zwar nicht nur als dummer Konsument, sondern um sich ein Einkommen zu sichern, und (b) Raubüberfälle haben erheblich abgenommen. Nun glaube ich aber so wenig wie Sie, dass diese Entwicklung nur Gutes bringt! Aber Informatiker können eben auch daran arbeiten, die negativen Begleiterscheinungen zu bekämpfen; in Darmstadt tun wir das unter anderem in einem interdisziplinären Graduiertenkolleg, wo wir an Privatheitsschutz und an der automatischen Bewertung digitaler Gefahren für mobile Nutzer forschen. Mobile Computing ist natürlich nur eine der vielen Facetten der Informatik. Eine andere nehmen Sie sicherlich auch wahr: immer mehr Gegenstände und Räume werden computerbewehrt und vernetzt und so \glqq smarter\grqq~ gemacht. Auch hier liegen Vor- und Nachteile eng beieinander; wir wissen aber: diese Entwicklung ist unaufhaltsam, Fitness-Bänder werden getragen und Städte werden \glqq vermessen\grqq~, die Energiewende ist ohne computergestützte Stromnetze nicht zu stemmen und exakt bedarfsgerecht gesteuerte Medizintechnik die Lebensqualität. Segen oder Fluch der allgegenwärtigen Computer? Wer Informatik studiert, kann entscheidend an der Antwort mitgestalten!

    Viele Bereiche der Informatik sind weit weniger zweischneidig, viele Entwicklungen verlaufen mit weniger Hype, verändern aber dennoch fast jeden Bereich von Wirtschaft und Alltag. Und vieles davon ist unerlässlich, wenn Deutschland und Europa im globalen Wettbewerb den eigenen Wohlstand sichern und denjenigen anderer Regionen fördern wollen. Hochgradig personalisierte Produkte werden künftig von \glqq 3D-Druckern\grqq~ ganz verschiedener Bauart in einem volldigitalen Prozess entstehen, Finanzwesen und Logistik stehen vor \glqq digitalen\grqq~ Umbrüchen, kein Sektor bleibt unberührt. Kommunikations- und Medienindustrie wachsen künftig noch viel stärker mit der IT zusammen, und nicht zuletzt die Informatik selbst definiert sich immer wieder neu: moderne Konzepte der Softwareentwicklung, der Mensch-Computer-Interaktion, der \glqq Künstlichen Intelligenz\grqq~, der \glqq Software-definierten\grqq~ Computer- und Netzwerk-Architekturen haben nur noch wenig zu tun mit der Informatik von vor fünfzehn Jahren -- vielleicht nicht einmal mit dem einen und anderen, was Sie im Schulfach Informatik gelernt haben. Sie merken: Ein Informatik-Studium ist eine der bestmöglichen Versicherungen, um nicht in einem öden, immer gleichen Berufsalltag zu enden. Und dass Informatiker oft -- und gute Informatiker sogar bei Wirtschaftskrisen! -- sehr gute Berufschancen haben, wissen Sie längst. Darmstadt liegt im Herzen der Region \glqq Software Südwest\grqq~, welcher erst kürzlich wieder eine Studie eine Spitzenposition unter allen Informatik-Regionen Europas bescheinigt hat. Und es ist umgeben von exzellenten Wirtschaftsstandorten, in denen Informatik angewandt und als Innovationsmotor dringend gebraucht wird. Sie erlauben mir eine kleine Träne beim Gedanken an die Gehälter ehemaliger Mitarbeiterinnen(!) und Mitarbeiter, die im Frankfurter Bankensektor arbeiten. Viele andere hat das Geld nicht so sehr gelockt, sie sind aber ebenso in ihrer Traumposition angekommen und in der Kerninformatik oder in einem der fast beliebig vielen Anwendungsfelder aktiv. Sie arbeiten als Entwickler oder Systemarchitektinnen, technikversierte Berater oder Projektleiterinnen, Spezialisten oder Managerinnen, in Entwicklung oder Forschung, akademischer Welt oder Industrie, in Kalifornien oder England, Malaysia oder Good Old Germany -- kaum etwas, wofür ich nicht Beispiele hätte. Und es ist wie bei der Liebe: wer wählen kann, ist fast immer besser dran. Als Informatikerin oder Informatiker können Sie Ihre Tätigkeit danach auswählen, wo Ihre Berufsliebe hinfällt!

    \bildmitunterschrift{../grafik/willkommen/muehlhaeuser_sw}{width=\columnwidth}{Prof. Dr. Max M\"uhlh\"auser}{}

    Eine verbreitete Meinung ist ja diese: Wer Kerninformatik machen will, soll Informatik studieren; wer aber Informatik anwenden will, ist besser dran, ein anderes Fach zu wählen und Informatik im Nebenfach zu studieren! Ohne der Nebenfach-Informatik nahe treten zu wollen, kann ich sagen: Das ist Blödsinn! Lassen Sie mich das anhand einer Frage erklären: Was ist der Unterschied zwischen einem Programmierer und einem Informatiker? Der Programmierer programmiert, der Informatiker modelliert und abstrahiert! Wie bitte? Nun, die Informatik ist ja in der Tat eine Disziplin, die ganz oft im Dienste eines anderen Faches steht. Der Automobilindustrie hilft sie beim autonomen Fahren, den Meteorologen bei der Wettervorhersage, der Ölindustrie beim Steuern und Regeln gigantischer Raffinerien, den Büros bei der Verwaltung von Abläufen, bei den Printmedien hat sie Beruf und Aufgabe des Schriftsetzers völlig ersetzt durch ihre Textverarbeitung usw. Dabei geht es immer um dasselbe: zu unterstützende Systeme und Prozesse müssen zuerst in Software modelliert d.h. \glqq nachgebildet\grqq~ werden, um sie digital unterstützen, verbessern oder gar revolutionieren zu können. Dabei besteht die große Kunst im Weglassen: was keine Rolle spielt, muss als solches erkannt (!) und weggelassen werden, was eine kleine Rolle spielt, muss stark vereinfacht modelliert werden, vieles vom Rest muss auf geschickte neue Art interpretiert werden, um es überhaupt in Software fassen zu können: dies ist die hohe Kunst des Abstrahierens und Modellierens, und das lernt man in einem Kerninformatik-Studium an der Universität wie nirgends sonst -- ich gebe zu: nur wenn man will und wenn man eine gewisse Portion Köpfchen hat! In meinen 40 Informatiker-Jahren habe ich so manche Branche gesehen, in der jahrelang die Alteingesessenen über die Informatiker gelächelt haben nach dem Motto \glqq WIR wissen, was in unserer Branche abgeht\grqq~. Ohne Überheblichkeit kann ich sagen: nicht selten haben die \glqq Davids\grqq~ aus der Informatik die \glqq Goliaths\grqq~ dieser Branchen am Ende überholt und mit teilweise radikalen neuen Ansätzen quasi ein neues Zeitalter eingeläutet. Man muss dazu aber Lust und geistige Flexibilität haben, um sich in andere und anderes hineindenken können und um das zu unterstützende Fach wirklich verstehen zu können, sonst kann man nichts verbessern und man wird zum \glqq verhassten ITler\grqq~.

    Was sollte man nun mitbringen für ein Informatik-Studium? Die Vielfalt dessen, was ich bisher beschrieben habe, soll Ihnen zeigen: alles was Sie brauchen, ist Interesse und Begeisterungsfähigkeit. Klar, wir haben viel Platz für analytische Superhirne und Mathe-Asse, aber das ist keine Voraussetzung. Wenn Sie im analytischen Denken nicht ganz talentfrei sind, warten ganz viele Bereiche der Informatik, die nicht \glqq sehr theorie-lastig\grqq~ sind, genau auf ihre besonderen Begabungen.

    In der Informatik finden sich auch immer noch Nerds und Nachtarbeiter, keine Frage. Aber wir haben inzwischen jedem Nerd und Technik-Freak schon etliche Computer verkauft, was wir jetzt brauchen ist IT für normale Menschen! Und deshalb rufe ich alle \glqq normalen\grqq~ Schulabgänger und ganz, ganz besonders alle Frauen auf: kommt und helft uns, damit mehr normale Menschen an den Computern für die normalen Menschen forschen und entwickeln! Lasst die Computer verschwinden und unmerklich ihren Dienst tun, lasst sie im Hintergrund unseren Alltag, unser normales Leben verbessern! Der Mensch gehört ins Zentrum, nicht der Computer. Ich verspreche Ihnen: es sind schon viele bei uns, die so denken und fühlen, und es werden immer mehr. Und immer mehr computerbasierte Geräte und Systeme machen Spaß statt Stress!
    Vielleicht habe ich Sie ein wenig Neugierig gemacht. Ich selbst habe auch aus Neugier Informatik studiert: keiner konnte mir recht erklären, was das ist -- da wollte ich es selbst herausfinden. Ich habe es wie eingangs geschrieben nie bereut und würde mich freuen, auch Sie bald als neugierige Studierende in unserem Fachbereich begrüßen zu dürfen!
}{Prof. Dr. Max Mühlhäuser, Dekan}
