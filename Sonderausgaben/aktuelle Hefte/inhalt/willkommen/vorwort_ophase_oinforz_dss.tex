\artikel{Welcome to the Ophase}
{As you are holding this booklet in your hands and are reading this article, chances are that your orientation phase (or Ophase, in short) has just begun or you are right in the middle of it.}
{
The orientation phase at the department of computer science sports a long history: it has existed for almost as long as the department of computer science itself has by now.
Its aim is for you to get to know the university, especially the department of computer science, and to help you connect with your future fellow students.

Every part of this event is organized and made possible by student volunteers eager to help new students like you to start their studies well prepared.
The tutors taking care of you throughout the Ophase are students like you, who have also participated in an orientation phase in the beginning of their studies, and a few semesters later decided to pass on their experiences.

This booklet, the Ophasen-Inforz (or OInforz, in short), is a special issue of the Inforz, a college magazine dealing with all things computer science (mainly with respect to what's happening at TU Darmstadt), which is published more or less regularly by the Fachschaft Informatik.
This Ophasen-Inforz contains all the information you are going to be taught in the orientation phase, and potentially some more, in written form.
Nevertheless, it can obviously not substitute the Ophase on its own, as this booklet won't answer questions beyond what has been written and also won't introduce you to your fellow students.
Thus, it is still highly recommended to attend the Ophase and primarily use the Inforz as a source of information to come back to later if and when you need to look something up that you were told during the Ophase but might not remember in full detail.

That being said, we hope you find this booklet helpful during the orientation phase as well as your studies at TU Darmstadt and wish you a pleasant and informative orientation phase!
}
{OInforz editorial staff}

%\newpage
