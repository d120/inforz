\artikel{Kommentierter Stundenplan}
{Auf der zweiten Seite dieses Inforz findest du deinen Stundenplan für die erste Woche. Hier stellen wir die einzelnen Punkte etwas genauer vor.}
{\textbf{Frühstück}

    Von Montag bis Freitag bieten wir morgens im Lernzentrum Informatik (LZI, S2$|$02 A020) ein leckeres Frühstück an. Montags starten wir gemütlich um 9:30 Uhr. Den Rest der Woche sind wir schon um 8:45 Uhr auf.  Abgesehen von kostenlosem Frühstück ist es eine schöne Gelegenheit, um in Ruhe mit anderen Ersties und natürlich auch Tutor*innen zu plaudern und den Tag locker beginnen zu lassen.\\

    \noindent\textbf{Begrüßung}

    Jetzt geht's endlich los. Mit vielen weiteren Ersties, die du noch nie vorher gesehen hast, aber ab sofort jeden Tag sehen wirst – zumindest die meisten –, sitzt du in einem Hörsaal und weißt nicht, was los ist. Kein Problem, denn in dieser Einführung bekommst du den Ablauf der nächsten Tage erklärt und findest zu deiner Kleingruppe.

    \bildmitunterschrift{../grafik/willkommen/Begruessung_Robert_sw}{width=\columnwidth}{Während einer Begrüßung im Wintersemester}{Robert Rehner}

    \noindent\textbf{Kleingruppe}

    In der Kleingruppe wird es übersichtlicher: hier sind nur noch 15-20 Leute zusammen, die man recht schnell kennen lernt. Es gibt jede Menge Infos von den Ophasentutor*innen für dich. Das sind ältere Studierende, die auch mal da gesessen haben, wo du jetzt sitzt, und genauso ratlos waren, wie du es im Moment vielleicht noch bist.

    Neben der Zusammenstellung des Studienplans und einer Uniführung haben sie eine ganze Menge Geschichten zu erzählen: wie sie ihr Studium bisher verbracht haben, zu welcher Zeit man am besten in die Mensa geht, in welchen Räumen man gut lernen kann und bei welchen Vorlesungen man nicht mal in der letzten Reihe schwätzen sollte.\\

    \noindent\textbf{Mensa}

    Für einige Studierende der einzige Grund, in die Uni zu gehen, zumindest für die, die nicht mehr zu Hause bekocht werden.

    In der Darmstädter Innenstadt gibt es vier Mensen, die du ausprobieren kannst. Für Informatikstudierende, die sich die ersten Semester komplett in der Stadtmitte aufhalten, ist die Mensa Stadtmitte (am Audimax) die Mensa der Wahl.

    Von 11 bis 14 Uhr gibt es für durchschnittlich zwei bis vier Euro eine warme Mahlzeit. Solange du noch keine Athenekarte hast, solltest du dich allerdings darauf einstellen, dass du 0,30\euro~ mehr bezahlen musst als der Preis in der Essensauslage besagt.

    Das Bistro hat von 8 bis 16 Uhr offen. Dort gibt es morgens Frühstück und den ganzen Tag über Kaffee und Kuchen, Gebäck und Süßigkeiten sowie Getränke.\\

    \noindent\textbf{Caf\'e und GnoM}

    Beim Games-no-Machines Spielenachmittag kannst du in gemütlicher Atmosphäre andere Studierende kennen lernen. Beim Caf\'e gibt es in gemütlicher Runde Kaffee und Kuchen um sich besser kennen zu lernen.\\

    \noindent\textbf{Kneipenabend}

    Alles was du bisher gemacht hast, war sehr uninah, doch jetzt geht es ins richtige studentische Leben.
    Erkunde 3 Kneipen in Darmstadts ausgeprägter Pub Landschaft und genieße die Zeit mit deinen kommenden Kommilitonen und lerne sie besser kennen. \\
    \newpage
    \noindent\textbf{Studienorganisation}

    Wie geht mein Studium nach dem Grundstudium weiter? Was hat es mit dem Mentorensystem auf sich; wofür ist die Studienberatung zuständig? Was ist die ISP/RBG und warum  brauche ich ein HRZ-Konto? Antworten auf diese und weitere Fragen zu deinem Studium geben wir dir in dieser Veranstaltung.\\

    \noindent\textbf{Fachvorträge}\\
    Warum eigentlich Informatik? Was kann ich damit später mal machen? Was gibt es so alles zu erlernen und erforschen? Solche und andere Fragen werden in dieser Veranstaltung behandelt. Du erhältst einen Einblick in das Fach, das du die nächsten Jahre studierst und bekommst eine Übersicht, was so alles möglich ist. Außerdem stellen sich noch einige Gruppen in den Kurzvorträgen vor.\\

    \noindent\textbf{Unirallye}\\
    Hast du bei der Uniführung am Montag gut aufgepasst? Dann weißt du ja noch, wo die ganzen Gebäude und Räume sind, die die Ophasentutor*innen dir gezeigt haben. Die gilt es jetzt nämlich zu finden und zwischendurch kleine Aufgaben wie Kistenstapeln zu bewältigen. Auf die Gewinnerinnen und Gewinner warten attraktive Preise wie beispielsweise Freikarten fürs Kino des Studentischen Filmkreises!\\

    \noindent\textbf{Das Geländespiel der Fachschaften}
    Auch in diesem Jahr wird wieder das legendäre Geländespiel zwischen den Mathematikern (Matikern), Physikern (Füsikern) und uns, den Informatikern, ausgetragen. An verschiedensten Stationen kämpft ihr um Punkte, damit ihr im großen Endspiel den entscheidenden Vorteil habt, um den Fachschaftenpokal nach Hause zu bringen. Gebt euer Bestes und zeigt den anderen, dass wir Informatiker auch ohne Rechner glänzen!\\

    \noindent\textbf{Spieleabend}\\
    Nach einem anstrengenden Geländespiel mit einem erhofften Pokal in der Tasche geht es anschließend zu unserem großen gemeinsamen Spieleabend mit Mathematikern und Physikern, wo ihr allerlei Brettspiele ausprobieren, eine Runde Jeopardy mitspielen und wie immer eure Kommilitonen besser kennen lernen könnt, in dem ihr mit ihnen zusammen einen Mord aufklärt oder sie bei Munchkin fertig macht. \\


    \noindent\textbf{Klausur}\\
    Oh Schreck! Die erste Klausur! Natürlich ist sie nicht so ernst gemeint wie die Klausuren, die noch kommen werden, aber sie ist eine gute Vorbereitung darauf. Denn wusstest du, dass du deinen Studienausweis und einen Personalausweis oder Reisepass benötigst, um mitschreiben zu dürfen? Oder wie die genaue Sitzordnung ist und wie oft man auf die Toilette gehen darf? Hier kannst du das alles lernen und außerdem dein Wissen testen.\\

    %\bildmitunterschrift{../grafik/willkommen/ostundenplan_unirallye}{width=\linewidth}{Kistenstapeln bei der Unirallye}{Benedikt Bicker}

    \noindent\textbf{Workshops}\\
    In den Workshops am Donnerstag und Freitag kannst du verschiedene Dinge tun, beispielsweise die Arbeit mit dem Textsatzprogramm \LaTeX~erlernen, deinen Körper beim Jugger in Form bringen oder die Angst vorm Beweisen hinter dir lassen.\\

    \noindent\textbf{Abschluss}\\
    Hiermit klingt die Ophase auch schon wieder aus: Es gibt noch ein paar Ankündigungen, Siegerehrungen und ein kleines Rahmenprogramm. Danach seid ihr herzlich eingeladen beim Caf\'e mit einem Stück Kuchen, einer Tasse Kaffee und ein paar Spielen zusammen zu sitzen.
}{}

\newpage
