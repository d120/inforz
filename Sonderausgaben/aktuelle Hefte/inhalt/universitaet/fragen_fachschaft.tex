\artikel{Fragen und Antworten rund ums Thema Fachschaft}
{Im vorherigen Artikel wurde auch die Rolle der Fachschaften grob erläutert. In diesem Artikel erfährst du, was deine Fachschaft eigentlich genau tut, mit welchen Anliegen du dich an sie wenden und auch, wie du dich selbst einbringen kannst.
}{
    \textbf{Wofür steht eigentlich "`die Fachschaft"'?}

    Der Begriff der Fachschaft beschreibt eigentlich die Gesamtheit der Studierenden eines Fachbereiches, in deinem Fall also der Informatik. Dementsprechend gehörst also auch du selbst zur Fachschaft der Informatik dazu, sofern du Informatik studierst.\\

    \textbf{Warum also nur "`eigentlich"'?}

    Das liegt daran, dass im alltäglichen Sprachgebrauch der Informatikstudierenden  unter dem Begriff "`die Fachschaft"' nur ein paar wenige Studierende verstanden werden, diejenigen nämlich, die man streng genommen als "`aktive Fachschaft"' bezeichnet.\\

    \textbf{Was unterscheidet nun diese Studierenden vom Rest der eigentlichen Fachschaft?}

    In erster Linie ihr freiwilliges, ehrenamtliches Engagement für die restlichen Studierenden, um deren Studiensituation zu verbessern oder zumindest zu verhindern, dass diese schlechter wird.\\

    \textbf{Und was kann die Fachschaft dann konkret bewirken?}

    Um diese Frage zu beantworten, solltest du dich an das zurückerinnern, was im vorherigen Artikel über die selbstbestimmende Studierendenschaft steht: Ähnlich wie an der Schule eine Schülervertretung existiert, gibt es an der Uni den AStA, der sich um die Belange aller Studierenden an der Uni kümmert. Wenn wir nun bei dieser Metapher bleiben, kannst du die (aktiven) Fachschaften der verschiedenen Fachbereiche grob mit den Stufensprechern und Stufensprecherinnen in der Schule vergleichen: Die aktive Fachschaft ist im Regelfall dein erster Ansprechpartner, wenn am Fachbereich etwas passiert, was deine Studiensituation negativ beeinflusst oder wenn du Vorschläge hast, wie man die Studiensituation, zumindest was deinen Studiengang betrifft, verbessern könnte. Nicht selten kommt es zum Beispiel vor, dass in einer Veranstaltung die Inhalte der Vorlesung nicht gut auf die entsprechenden Übungen abgestimmt sind, was in ohnehin schwierigeren Fächern für viel Frustration sorgen kann. In so einem Fall kannst du dich (am besten gemeinsam mit ein paar anderen betroffenen Kommilitoninnen und Kommilitonen) an die Fachschaft wenden, sodass diese deine Beschwerde dann an den entsprechenden Veranstalter bzw. die entsprechende Veranstalterin weitergeben können. Das mag zwar länger dauern, als dich direkt an den*die Dozent*in zu wenden, hat aber zwei große Vorteile: zum einen bleibst du dadurch im Regelfall anonym und zum anderen genießt die Fachschaft auch unter den Dozent*innen ein recht hohes Ansehen, wodurch so eine Beschwerde mehr Gewicht bekommt, wenn sie durch die Fachschaft vorgetragen wird. Und nicht zuletzt haben Vertreterinnen und Vertreter der Fachschaft auch Einfluss auf die Fachbereichspolitik und können derartige Konflikte somit weiter nach oben tragen, sodass im Extremfall sogar der Dekan oder die Dekanin selbst ein Machtwort sprechen muss.\\

    \textbf{Muss ich mich also mit allem, was mich an einer Veranstaltung stört, an die Fachschaft wenden?}

    Natürlich nicht. Die Fachschaft nimmt sich hauptsächlich gravierender Probleme an, von denen viele Studierende betroffen sind oder die zu beheben größere "`Lobbyarbeit"' erfordert. Wenn du ein persönliches Problem mit einem bzw. einer Dozent*in hast oder mit Kleinigkeiten, was eine Veranstaltung angeht, dann ist es meist zeitsparender und unkomplizierter, sich mal direkt an die betreffende Veranstalterin bzw. den betreffenden Veranstalter zu wenden und auf die von dir festgestellten Mängel hinzuweisen. In der Regel sind sie sogar dankbar, wenn sie darauf hingewiesen werden, weil ihnen selbst die studentische Perspektive fehlt.\\

    \textbf{Wie kann ich denn die Fachschaft erreichen, sollte ich nun wirklich mal ein schwerwiegenderes studientechnisches Problem haben?}

    Da gibt es verschiedene Wege: Meist kannst du im Fachschaftsraum S2$|$02 D120 irgendjemanden von der aktiven Fachschaft antreffen und Fragen stellen oder Probleme ansprechen. Wenn dir der oder die Angesprochene auch manchmal nicht direkt helfen kann, kann er oder sie aber normalerweise zumindest sagen, an wen du dich sonst noch wenden könntest, oder Fragen und Probleme an den Rest der Fachschaft weiterreichen. Falls du nicht persönlich nach D120 kommen kannst oder willst, oder falls mal niemand da sein sollte, kannst du immer noch an die Fachschafts Mailingliste wir@D120.de schreiben.\\

    \textbf{Nun wird schon die ganze Zeit mit dem Begriff der "`Studiensituation"' um sich geworfen – um welche Aspekte des Studiums kümmert sich die Fachschaft denn nun genau?}

    Die Fachschaft ist natürlich auch hochschulpolitisch aktiv, hauptsächlich (aber nicht ausschließlich) auf Fachbereichsebene: hier werden drei studentische Vertreterinnen und Vertreter in den Fachbereichsrat gewählt und haben somit Mitspracherecht, was einige Entscheidungen am Fachbereich betrifft. Ebenso arbeiten einige Fachschaftler*innen auch in verschiedenen Ausschüssen und Kommissionen am Fachbereich mit, zum Beispiel im QSL-Ausschuss (Qualitätssicherung von Studium und Lehre), der sich mit der Investition der QSL-Fördermittel des Landes Hessen befasst oder dem Lehr- und Studienausschuss (LuSt), der sich mit allen lehrbezogenen Themen auseinandersetzt. Da eine Aufzählung aller Gremienarbeiten hier den Rahmen sprengen würde, sei an dieser Stelle auf \footnotemark[1] verwiesen, wo du eine ausführlichere Auflistung aller hochschulpolitischen Tätigkeiten der Fachschaft findest.\\

    \textbf{Das klingt ja nun sehr nach serious business – macht die Fachschaft denn nur Politik?!}

    Hochschulpolitisches Engagement ist ein wichtiger Teil der Fachschaftsarbeit, aber bei weitem nicht der Einzige. Neben den oben Genannten gibt es noch jede Menge anderer Aktivitäten, für die die Fachschaft ganz oder zumindest zu einem guten Teil verantwortlich ist. Die offensichtlichste Fachschaftsaktivität erlebst du gerade mit: Die gesamte Ophase wird von der aktiven Fachschaft organisiert und viele der Tutor*innen sind auch anderweitig in der Fachschaft aktiv. Das Ophasen-Inforz, in dem du gerade liest, ist eine Sonderausgabe des Inforz, der Zeitschrift der Studierenden der Informatik, welches auch von einer Redaktion, die überwiegend aus Fachschaftler*innen besteht, geführt, organisiert und auch während des Semesters veröffentlicht wird. Im Moment erscheint es allerdings eher unregelmäßig, da vor allem engagierte Leute fehlen, die gerne Artikel für das Inforz schreiben würden. Auch nicht-Fachschaftler sind dafür gerne gesehen.\\
    Des weiteren organisiert die Fachschaft auch die am Ende jedes Semesters durchgeführte Evaluation der Lehrveranstaltungen und wertet deren Ergebnisse aus. Aber auch für einige Freizeitaktivitäten am Fachbereich ist die Fachschaft verantwortlich: Jedes Jahr werden von der Fachschaft das Sommerfest und die Nikolausfeier organisiert und auch der monatlich stattfindende Gesellschafts-spieleabend Games no Machines (GnoM), zu dem du mehr Informationen im Freizeitteil finden kannst, wird von der Fachschaft veranstaltet. Neben diesen gibt es noch eine ganze Menge Aktivitäten mehr, eine genauere Übersicht findest du unter \footnotemark[2].\\

    \textbf{Himmel, man liest ja in fast jedem Satz das Wort "`organisiert"'! Braucht man bei dieser Menge an Tätigkeiten nicht Unmengen von Leuten?}

    Das tut man in der Tat und leider mangelt es auch genau an diesen. Die gesamte Fachschaftsarbeit ist nun mal ehrenamtlich, weshalb viele Studierende wenig motiviert sind, neben dem zeitaufwändigen Studium auch noch Zeit in etwas zu investieren, wovon sie selbst am Ende noch mit am wenigsten haben. Nun stell dir aber mal vor, wie dein Studium laufen würde, wenn es die Ophase nicht gäbe oder sich niemand mal organisiert dafür einsetzen würde, dass Dozent*innen, die miserable Vorlesungen halten, ebendies nach Möglichkeit nicht wieder tun. Hauptsächlich der Arbeit der Fachschaft in den letzten Jahren und Jahrzehnten ist es zu verdanken, dass die Studienbedingungen heutzutage gar nicht mal so schlecht sind, auch wenn es an vielen Stellen immer noch Verbesserungsbedarf gibt. Ein Problem, das von vielen Studierenden auch noch übersehen wird ist, dass auch Fachschaftler*innen Studierende sind, irgendwann mal ihr Studium abschließen und die Uni danach verlassen. Daher ist die aktive Fachschaft kontinuierlich auf Nachwuchs angewiesen.\\

    \textbf{Bei der Menge an Arbeit, die die Fachschaft hat, muss man sich aber auch nicht wundern, dass das von Mitarbeit abschreckt.}

    Das ist eine sehr einseitige Sichtweise. Natürlich gibt es in der Fachschaft viel zu tun, aber auf der anderen Seite würden sich diese Arbeiten viel besser in kleine Portionen verteilen lassen, wenn mehr Leute mithelfen. Insbesondere gibt es auch jede Menge kleinere oder einmalige Aktionen, bei denen mal Helfer*innen gebraucht werden, ohne dass man sich wirklich tief in fachschaftsspezifische Themen reinknien muss. Das OInforz, das du in der Hand hältst, wurde für diese Ophase von jemandem überarbeitet, der nicht in der Fachschaft ist - du siehst also, das ist durchaus möglich.\\

    \textbf{Echt? Wie erfahre ich denn von so was, damit ich mich hin und wieder mal einbringen kann?}

    Über das, was in der Fachschaft ansteht, kannst du dich auf viele Arten und Weisen informieren. Immer eine gute Anlaufstelle ist der Fachschaftsraum S2$|$02 D120, in dem sich mit hoher Wahrscheinlichkeit Fachschaftler*innen aufhalten, die dir zumindest erzählen können, was gerade aktuelle Themen sind. Wenn du aber einen tieferen Einblick in die Arbeit der Fachschaft bekommen möchtest, solltest du eine Fachschaftssitzung besuchen, die jeden Mittwoch um 18:00 Uhr stattfindet. Hauptsächlich zu diesen Sitzungen werden alle gerade wichtigen Themen angesprochen und diskutiert, während der Fachschaftsraum zu anderen Zeiten hauptsächlich als offener Aufenthaltsraum, aber auch für Fachschaftsarbeiten genutzt wird. Andere Möglichkeiten, dich über die Fachschaft und deren Tätigkeiten zu informieren, sind die Fachschaftswebsite \footnotemark[3], der Fachschafts-Blog das Wesentliche \footnotemark[4], die Facebook-Seite der Fachschaft \footnotemark[5] oder der Twitter-Account @d120de \footnotemark[6]. Falls du dich nur hin und wieder oder nur in einem bestimmten Bereich beteiligen möchtest, kannst du dich auch auf einer der offenen Mailinglisten eintragen, eine Auflistung davon findest du unter \footnotemark[7].
}
{Stefan Gries,\\
    überarbeitet von Julian Haas}


\footnotetext[1]{\url{http://www.d120.de/gremien/}}
\footnotetext[2]{\url{http://www.d120.de/de/studierende/}}
\footnotetext[3]{\url{http://www.d120.de}}
\footnotetext[4]{\url{http://daswesentliche.d120.de}}
\footnotetext[5]{\url{https://www.facebook.com/d120.de}}
\footnotetext[6]{\url{https://twitter.com/d120de}}
\footnotetext[7]{\url{http://www.d120.de/mailman/listinfo}}

\newpage
