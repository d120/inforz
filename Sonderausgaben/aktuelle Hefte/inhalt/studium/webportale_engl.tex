\artikel{Web portals at TU Darmstadt}
{TUCaN, Moodle, etc. During the first weeks of your studies you will have to learn how to use the web portals at the TU Darmstadt and the computer science department. In this article you will learn what they are for and how to get started using them.
}{
    \noindent\textbf{TUCaN}
    Let's begin with the most important one: the TU CampusNet (or just TUCaN, \footnotemark[1]) is the web portal which administrates the progress of your studies. This also implies what it should be to you: an administrative tool. In TUCaN every examination regulation that exists at the TU Darmstadt is stored (for you this is most likely the examination regulation M.Sc. Distributed Software Systems 2015 or one of the other computer science master regulations). All the data stored here, which includes the courses and exams you take during your studies can be viewed and modified by the Examination Office.

    This may sound quite abstract but the way it works is actually quite simple: your studies are organized into courses. These are mostly lectures, sometimes with exercises. Courses are the top level of organization for your studies. To take an exam, and then get the credits and your grade, you have to register for the course under courses $\rightarrow$ registration. Only then you can register for the corresponding lecture (and exercise) the same way you registered for the course. Only when you have done this you can register for an exam during the exam registration period (normally from the middle of November to the middle of December for winter terms and during June for summer terms). You register for exams via exams $\rightarrow$ my exams $\rightarrow$ exams offered for registration.

    Only if you are registered for an exam will it be categorized as an examination attempt. If you don't take the exam (and you fail to provide a sick note) although you were registered in TUCaN, you fail the exam. You don't get any credits until you finally pass the exam. For a failed exam you get the grade 5.0. If you know that you don't want to take the exam eight days or earlier before the exam date you can deregister yourself via exams $\rightarrow$ my exams.
    \end{multicols}

    \bildmitunterschrift{../grafik/artikel/Anmeldungspipeline}{width=\textwidth}{}{Thomas Pilot, translated by Johannes Alef}

    \begin{multicols}{2}
    This article can only give you a rough overview of TUCaN but there are tutorials and other means of help on the start page of TUCaN should you need help with a feature.

    Registration and deregistration are the most important functions of TUCaN but not the only important ones. Lecturers may send you messages via TUCaN or provide material. You can also see your current progress under exams $\rightarrow$ performance record. The integrated course catalogue is also very helpful. Here you can find all courses offered during the current semester along with some information and important dates. TUCaN also offers a schedule for your studies that can be of (limited) help for you. But more on that in the last paragraph.\\

    \noindent\textbf{Moodle}
    In contrast to TUCaN, Moodle isn't an administrative tool but a learning management system. It is used for providing material and making the lectures a bit more interactive aside from the lectures and exercises for example by providing forums. At the TU Darmstadt there is more than one Moodle. Several departments have their own Moodles and there is a general TU-Moodle for the university. The Moodle of the computer science department \footnotemark[2] is officially called "Lernportal Informatik".

    Of course it's not much of a surprise that not everyone uses the same Moodle. You may have courses that use the computer science Moodle, but you may just as well have courses that use the TU-Moodle \footnotemark[3].

    Yet the general appearance of the different Moodles is the same: once logged in (normally using your TU-ID with the corresponding password), you can register for courses. Some courses may require a password to register. You get these passwords in the lecture of the corresponding course. If the TU-Moodle is used you will most of the time automatically be registered for this course after you registered for this course in TUCaN.

    A Moodle course provides a time line showing the weeks of the semester and the associated materials. Sometimes you can also upload your solutions to exercises to get them evaluated by a tutor. Some courses also offer forums where you can exchange thoughts with other students and some lecturers also run surveys via Moodle. What the course looks like in Moodle and what is offered entirely depends on the lecturer.\\

    \noindent\textbf{Other web portals}
    Now you know the two most important web portals for your studies. There are, of course, more web portals that are used by other departments or by specific research groups, that can be helpful but are mostly used only for one specific course.

    The HRZ administrates the web portal OpenLearnWare \footnotemark[4]. Here you can find free learning materials and video recordings of lectures (mostly in German).

    The Fachschaft offers a subforum \footnotemark[5] for almost every lecture where you can discuss the lecture and its content with other students. In this forum you can discuss anonymously but often lecturers will also participate in these forums.\\

    \noindent\textbf{Important}
    In general, Moodle is mostly used for exercises. Lecturers may administrate the assignment of students to different exercise groups via Moodle. If this is the case, the date you choose for this exercise group doesn't have to be the same as the one you chose in TUCaN (Where you normally have to register for exercise groups as well). That's why often the schedule shown in TUCaN is not very useful since it only shows the dates registered in TUCaN but often the Moodle dates take precedence.
}
{Stefan Gries, edited by \\ Tobias Otterbein\\ edited and translated by Johannes Alef}

\footnotetext[1]{\url{https://www.tucan.tu-darmstadt.de}}
\footnotetext[2]{\url{https://moodle.informatik.tu-darmstadt.de}}
\footnotetext[3]{\url{https://moodle.tu-darmstadt.de}}
\footnotetext[4]{\url{https://openlearnware.hrz.tu-darmstadt.de}}
\footnotetext[5]{\url{https://d120.de/forum/}}

%\vfill
%\bildmitunterschrift{../grafik/comics/porn_folder}{width=\textwidth}{}{xkcd.com}
\newpage
