\artikel{Die Ordnung des Studiengangs IT-Sicherheit}
{Dieser Abschnitt beschäftigt sich speziell mit der Ordnung des Studiengangs IT-Sicherheit.
}{
    Die folgenden Angaben sind wie immer ohne Gewähr. Verbindlich sind nur die offiziellen Versionen der Ordnung und der Allgemeinen Prüfungsbestimmungen. Verbindliche Informationen geben außerdem die (Fach-)Studienberatung, der*die Dekan*in, die*der Studiendekan*in und das Studienbüro.\\

    \noindent\textbf{IT-Sicherheit}

    Der Studiengang IT-Sicherheit befasst sich speziell mit der Sicherheit und Zuverlässigkeit von Softwaresystemen und -technologien.

    Der Koordinator des Studiengangs IT-Sicherheit ist Prof. Stefan Katzenbeisser.\\

    \noindent\textbf{Pflichtbereich}

    Es müssen drei Pflichtveranstaltungen absolviert werden: \textit{Introduction to Cryptography}, \textit{Embedded System Security} und \textit{Introduction to IT Security}.\\

    \noindent\textbf{Wahlbereiche}

    Der Studiengang umfasst fünf Wahlbereiche. In jedem Bereich sind mindestens 6 und maximal 42 CP zu erbringen, in den ersten vier Bereichen zusammen insgesamt 57 bis 60 CP. Diese Bereiche sind \textit{Cryptography}, \textit{System Security}, \textit{Software Security} und \textit{Selected Complementary Topics}. Der fünfte Wahlbereich, \textit{Studienbegleitende Leistungen}, umfasst Seminar, Praktikum, Projektpraktikum, Projekt, Studienarbeit und Praktika in der Lehre. In diesem Bereich müssen 12 bis 15 CP erbracht werden. Es muss mindestens eins, maximal zwei Seminare besucht werden. Außerdem muss mindestens eine der Formen Praktika, Projektpraktika und ähnlicher Veranstaltungen besucht werden. Es kann höchstens ein Praktikum in der Lehre eingebracht werden.\\

}
{Johannes Alef}
