\artikel{Imposed conditions}
{For your studies at TU Darmstadt you may have received some imposed conditions. Here you will learn what they are and what their influence is.
}{\noindent\textbf{What are imposed conditions?}
Te begin a Master study program of computer science at TU Darmstadt you are required to already have knowledge on all major areas of computer science.
When you apply with your Bachelor Degree then the contents of your courses are checked if you have all the knowledge that you need for our study program.
If you are missing knowledge of certain areas we require you to pass certain courses after which you will have this knowledge.
Those courses are the imposed conditions.

\noindent\textbf{What is their relevance to me?}
All imposed conditions have to be passed in your first year of studies.
If you don't pass all of these courses you get de-registered from the study program.
So try to pass these courses as soon as possible.

\noindent\textbf{I think I already have that knowledge.}
If you think that you have been given an imposed condition although you acquired that knowledge in your previous study program then the talk to the Student Advisory Service for further information.

\noindent\textbf{How to pass these courses.}
First of all you must register for theses courses in TUCaN.
There you can find theses courses under "zusätzliche Leistungen"$\rightarrow$"Gesamtkatalog aller Module an der TU Darmstadt" $\rightarrow$ "Gesamtkatalog aller Module FB 20 Informatik".
The modules should then be under "Grundlagenveranstaltungen" or "Kanonische Einführungsveranstaltungen".
If you have problems with registering for these courses please contact the Examination Office.
Some of these courses are only available in German.
The exams will be in English.
You can get learning material and more information here: \footnotemark[1].



}{Johannes Alef}

\footnotetext[1]{\url{https://www.informatik.tu-darmstadt.de/en/international/applicants-for-a-university-place-from-abroad/application-and-admission/imposed-conditionsconditional-acceptance/}}
