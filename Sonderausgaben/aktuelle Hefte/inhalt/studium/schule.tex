\artikel{Von der Schule an die Uni}
{Egal ob Du direkt aus der Schule kommst, danach erst noch eine Ausbildung gemacht oder Freiwilligendienst geleistet hast, oder gar aus der Arbeitswelt ins Studium einsteigst: An der Uni läuft vieles etwas anders.}
{Insbesondere wenn man bisher nur die Schule als Lerninstitution kennen gelernt hat, unterliegt man anfangs schnell Irrtümern über die Universität. Die (teilweise enormen) Unterschiede liegen hauptsächlich in veränderten Einstellungen zur Lehre begründet und sollen im Folgenden erläutert werden, um Dir schon einmal einen groben Eindruck zu vermitteln, was mit Deinem Studium eigentlich auf Dich zukommt.

    Das vielleicht Wichtigste, das Du Dir klar machen solltest ist, dass an der Universität eine andere Erwartungshaltung an Dich existiert als in der Schule: wurde in Letzterer noch viel Wert daraufgelegt, dass Du jeden Tag zu allen Fächern erscheinst, regelmäßig Deine Aufgaben bearbeitest und die Klausuren mitschreibst, damit Du das Schuljahr nicht wiederholen musst, existieren in der Universität nahezu keine derartigen Zwänge. An der Uni wird Dir nicht vorgeschrieben, wann Du welches Fach zu belegen hast und wie lange Du für Dein Studium brauchen sollst – die so genannte "`Regelstudienzeit"' ist eine Maßgabe, Du kannst Dein Studium auch in kürzerer (unwahrscheinlich) oder längerer Zeit (eher die Regel als die Regelstudienzeit) absolvieren. Dabei ist es übrigens nicht, wie in der Schule, möglich "`sitzen zu bleiben"', auch wenn es mit Pech durchaus vorkommen kann, dass man in zwei Semestern die gleichen Veranstaltungen belegt. Es bestehen nur in den wenigsten Fächern tatsächliche Anwesenheitspflichten.

    Auch bei der Auswahl der Fächer gibt es keine Vorgaben wie in der Schule – es gibt nur Regelungen, welche Fachprüfungen Du für den Studienabschluss irgendwann einmal bestanden haben musst. Wann und in welcher Reihenfolge Du das tust, bleibt aber Dir überlassen. Für Deinen Lernfortschritt bist Du also vollständig selbst verantwortlich. Damit geht auch einher, dass Du selbst dafür zuständig bist, Dir zu Beginn jedes Semesters einen Stundenplan für das Semester zusammenzustellen und Dich rechtzeitig zu den Prüfungen, an denen Du teilnehmen möchtest, anzumelden.

    Ein weiterer enormer Unterschied besteht in den Veranstaltungsformen. Während es in der Schule nur "`Unterricht"' gibt, existieren davon an der Uni viele verschiedene Formen, unter anderem Vorlesungen, Übungen, Seminare und Projektarbeiten (so genannte "`Praktika"', die nicht mit Betriebspraktika verwechselt werden sollten). Mehr Details zu den verschiedenen Veranstaltungsformen findest Du im folgenden Artikel.

    In der Art und Weise, wie gelehrt und gelernt wird, bestehen ebenfalls große Unterschiede. Du wirst schnell feststellen, dass das Tempo, mit dem an der Uni Wissen vermittelt wird, deutlich über dem der Schule liegt. Außerdem wird auch nicht sämtliches Wissen auf dem Silberteller präsentiert – in vielen Fällen wird auch erwartet, dass Du Dir selbstständig noch zusätzliches bzw. vertiefendes Wissen aneignest.

    Zuletzt solltest Du Dich auch schon mal darauf einstellen, dass Dir kaum jemand auf die Finger hauen wird (auch nicht im übertragenen Sinne), wenn Du im Studium nicht vorankommst. Das mag zwar zunächst positiv klingen, praktisch fällt es aber oftmals schwer, sich selbst zu motivieren, um mit dem Lernen voranzukommen.

    Die oben genannten sind noch bei weitem nicht alle Unterschiede zur Schule. Neben diesen gibt es natürlich auch in der Struktur große Unterschiede (z.B. durch die Unterteilung in Fachbereiche), spezielle Systeme, die die Verwaltung oder die Lehre unterstützen und neben all dem auch noch interne Politik, an der auch einige studentische Organisationen teilhaben und die z.T. sogar starkes Mitspracherecht haben (z.B. der AStA und die Fachschaften).

    Zu guter Letzt solltest Du Dich wegen all dieser Unterschiede aber nicht verrückt machen. Es ist zwar in der Tat so, dass man an der Uni viel mehr aus eigener Initiative angehen muss, hängen gelassen wird man allerdings selten. In vielen Fällen gibt es Anlaufstellen, die Unterstützung bei Deinen Problemen bieten.
}
{}
