\artikel{Webportale an der TU Darmstadt}
{TUCaN, Moodle… in den ersten Wochen des Studiums weiß man oft mit den ganzen Webportalen an der TU und am Fachbereich Informatik nicht ein noch aus. Hier wird dir erklärt, welches Portal was kann und wie du dich zumindest für den Anfang auf ihnen zurechtfindest.
}{\noindent\textbf{TUCaN}

    Fangen wir mit dem Wichtigsten zuerst an: das TU CampusNet (kurz TUCaN, \footnotemark[1]) ist dasjenige Webportal, über welches der größte Teil deines Studienfortschrittes verwaltet wird. Das impliziert im Prinzip auch, als was du dieses Portal hauptsächlich sehen solltest: als Verwaltungswerkzeug. TUCaN basiert darauf, dass darin die Ordnung jedes einzelnen Studiengangs, die an der Uni existiert (in deinem Fall also Informatik 2015, siehe den entsprechenden Artikel), eingespeichert ist und aus diesen Daten die Module und Prüfungsleistungen, die du im Laufe deines Studiums erbringen musst, ausgelesen bzw. von den Prüfungssekretariaten auch modifiziert werden können.

    Das klingt zunächst sehr abstrakt, im Grunde ist das Funktionsprinzip aber recht einfach: Dein Studium ist in so genannten Modulen organisiert, die in der Informatik hauptsächlich aus einer Vorlesung oder integrierten Lehrveranstaltung und einer Prüfungsleistung bestehen, oftmals zusätzlich auch noch mit einer Übung oder einer Studienleistung. Theoretisch könnten Module auch aus mehreren, thematisch zusammenhängenden Vorlesungen und Übungen bestehen, die sich über mehrere Semester erstrecken. Praktisch wird das aber in deinem Studiengang nicht genutzt. Nichtsdestotrotz sind Module die oberste Ebene, die für deine Studienorganisation, also auch für TUCaN, relevant ist. Um eine Prüfung in einem Modul ablegen zu können, also um letztendlich für eine Veranstaltung eine Note und Credit Points angerechnet zu bekommen, musst du dich in TUCaN zunächst unter Veranstaltungen $\rightarrow$ Anmeldung für das entsprechende Modul anmelden. Danach erst kannst du dich auf dem gleichen Weg für die in dem Modul enthaltenen Vorlesungen und Übungen anmelden. Wenn das geschehen ist und der Prüfungsanmeldezeitraum (normalerweise von Mitte November bis Mitte Dezember im Wintersemester, im Juni im Sommersemester) läuft, kannst (und solltest) du dich auch für die Studien- und Prüfungsleistungen des entsprechenden Moduls anmelden – das allerdings über das Menü Prüfungen $\rightarrow$ Meine Prüfungen $\rightarrow$ Anmeldung zu Prüfungen.

    Erst wenn du zu einer Prüfung angemeldet bist, gilt eine in diesem Fach geschriebene Klausur als Prüfungsversuch und wird entsprechend gewertet. Bist du bei der Klausur nicht anwesend (und lieferst kein ärztliches Attest nach, falls du krank warst), obwohl du in TUCaN angemeldet warst, wird dir zunächst eine 5,0 eingetragen und du bekommst keine Credit Points angerechnet, bis du das Modul bestanden hast. Wenn bis 7 Tage vor einer Klausur absehbar ist, dass du sie nicht mitschreiben kannst oder willst, kannst du dich unter Prüfungen $\rightarrow$ Meine Prüfungen auch wieder davon abmelden.

    An dieser Stelle kann nur ein recht grober Überblick gegeben werden, es gibt auf der TUCaN-Startseite aber einige Links zu Tutorials und ähnlichen Bedienungshilfen, falls du mit einer Funktion nicht auf Anhieb zurechtkommen solltest.

    Die Veranstaltungsan- und abmeldungen sind zwar die wichtigsten Funktionen von TUCaN, aber nicht die einzigen. Über TUCaN können Veranstalterinnen und Veranstalter auch Nachrichten verschicken oder Materialien zur Verfügung stellen und du kannst unter Prüfungen $\rightarrow$ Leistungsspiegel deine bisherigen Leistungen im Studium einsehen. Sehr nützlich ist auch das in TUCaN integrierte Vorlesungsverzeichnis (VV), in dem du alle im gegenwärtigen Semester angebotenen Veranstaltungen mit vielen Details und den meisten relevanten Terminen dazu findest. Das VV ist auf jeden Fall das wichtigste Werkzeug, wenn es darum geht, seinen Stundenplan zusammenzustellen. A propos Stundenplan: auch den bietet TUCaN an, für Informatik-Studierende eignet dieser aber teilweise nur begrenzt – warum, wird im letzten Abschnitt erklärt.

    Für Studierende des normalen M.Sc. Informatik sei noch erwähnt, dass man untere Veranstaltungen $\rightarrow$ Meine Wahlbereiche noch die 4 bis 5 Wahlpflichtbereiche sowie das Nebenfach (siehe auch Abschnitt Ordnung des Studiengangs) auswählen muss, bevor man sich zu den entsprechenden Veranstaltungen anmelden kann.\\

    \noindent\textbf{Moodle}

    Im Gegensatz zu TUCaN ist Moodle keine Verwaltungs- sondern eine Lernplattform. Der Sinn von Moodle ist weniger, das Studium zu verwalten, als vielmehr, Lehrenden eine Möglichkeit zu bieten, Inhalte für Studierende zur Verfügung zu stellen und die Lehre auch abseits der eigentlichen Vorlesungen und Übungen (z.B. durch integrierte Foren) interaktiver zu gestalten. An der TU Darmstadt gibt es allerdings nicht DAS Moodle. Einige Fachbereiche besitzen ein eigenes Moodle und es gibt zudem noch das uniweite TU-Moodle. Das Moodle am Fachbereich Informatik \footnotemark[2] heißt offiziell "`Lernportal Informatik"'.

    Wenig überraschend ist es natürlich nicht einheitlich, welche Moodle-Instanz genutzt wird. Es kann also durchaus passieren, dass die Übungen einer Veranstaltung über das Lernportal Informatik organisiert werden, während eine andere Veranstaltung ihre Materialien im TU-Moodle \footnotemark[3] zur Verfügung stellt.

    Der grundsätzliche Aufbau dieser Portale ist jedoch gleich: wenn man sich angemeldet hat (im Regelfall mit deiner TU-ID und dem zugehörigen Passwort), kann man sich für Kurse anmelden, wobei manche davon noch durch Passwörter gesichert sein können, die man zumeist in der ersten Vorlesung des entsprechenden Faches erfährt. Hat ein*e Dozent*in einen Kurs im TU-Moodle angelegt, wird man automatisch in diesen Kurs eingetragen, wenn man für die zugehörige Veranstaltung in TUCaN angemeldet ist.
    \end{multicols}

    \bildmitunterschrift{../grafik/artikel/Anmeldungspipeline}{width=\textwidth}{}{Thomas Pilot}

    \begin{multicols}{2}
    Innerhalb eines Kurses gibt es dann eine Timeline über die einzelnen Wochen des Semesters und die ihnen zugeordnete Lerninhalte. Über die Moodle-Plattformen können auch Dateien (zum Beispiel Hausübungen) hochgeladen werden, deren Bewertung dann an gleicher Stelle eingesehen werden können. In vielen Veranstaltungen werden auch Moodle-interne Foren zur Verfügung gestellt oder hin und wieder Umfragen durchgeführt. Letztendlich ist aber der genaue Aufbau des Kurses von Veranstaltung zu Veranstaltung unterschiedlich, da die Veranstalterinnen und Veranstalter selbst auswählen können, welche Moodle-Komponenten sie für den angebotenen Kurs verwenden.\\

    \noindent\textbf{Andere Webportale}

    Mit TUCaN und Moodle kennst du nun die wichtigsten Webportale, die dich über dein gesamtes Studium begleiten werden. Allerdings bieten viele Fachbereiche und teilweise auch einzelne Fachgebiete noch viele andere Portale an, die recht nützlich sein können, oft aber nur für einzelne Veranstaltungen verwendet werden.

    Das HRZ bietet mit OpenLearnWare \footnotemark[4] eine Plattform, auf der viele freie Lernmaterialien und auch Vorlesungsaufzeichnungen gesammelt werden. Aus der Informatik sind zwar erst recht wenige Aufzeichnungen vorhanden, aber mit Funktionalen \& Objektorientierten Programmierkonzepten, Algorithmen \& Datenstrukturen und Mathematik 1 und 2 sind einige der wichtigsten Fächer abgedeckt.

    Für alle Vorlesungen bietet die Fachschaft Informatik ein eigenes Unterforum an, in dem über die jeweilige Veranstaltung und deren Inhalt diskutiert werden kann. Das Forum ist im Gegensatz zu den Moodle-Foren anonym, wird jedoch nicht immer auch vom Veranstalter bzw. der Veranstalterin der Vorlesung betreut.\\

    \noindent\textbf{Wichtig}

    Allgemein wird Moodle nur für Übungen verwendet, das allerdings von vielen Veranstalterinnen und Veranstaltern in der Informatik. Oft führen die Veranstalterinnen und Veranstalter die Zuteilung zu einzelnen Übungsgruppen dann über Moodle durch. Im Regelfall musst du dich dafür bei oder kurz nach dem Anmelden für einen Moodle-Kurs für einen Übungstermin (oder mehrere Favoriten) entscheiden, der nicht notwendigerweise der ist, für den du in TUCaN (wo du meist ebenso einen Termin auswählen musst, wenn du dich für eine Übung anmeldest) eingetragen bist.

    Da der TUCaN-Stundenplan allerdings nur erfasst, für welche Übung du in TUCaN angemeldet bist, werden durch Änderungen dank Moodle oft inkorrekte Termine im TUCaN-Stundenplan angezeigt, da in diesem Fall letztendlich die Moodle-Termine verbindlich sind. Dennoch musst du dich auch in TUCaN für eine Übung anmelden, ansonsten kann die Prüfungsanmeldung in diesem Fach unmöglich werden. Meist empfiehlt es sich, mit der Anmeldung zur Übungsgruppe in TUCaN solange zu warten, bis der Termin durch Moodle festliegt und sich dann zu dem zugewiesenen Termin in TUCaN anzumelden.
}
{Stefan Gries, überarbeitet von \\ Tobias Otterbein und Johannes Alef}

\footnotetext[1]{\url{https://www.tucan.tu-darmstadt.de}}
\footnotetext[2]{\url{https://moodle.informatik.tu-darmstadt.de}}
\footnotetext[3]{\url{https://moodle.tu-darmstadt.de}}
\footnotetext[4]{\url{https://openlearnware.hrz.tu-darmstadt.de}}

%\vfill
%\bildmitunterschrift{../grafik/comics/porn_folder}{width=\textwidth}{}{xkcd.com}
\newpage
