\artikel{Methods of Teaching and Learning}
{The methods of teaching at TU Darmstadt may differ from those you already know. In this article you get to know the most used methods for teaching.
}{
A course may consist of several teaching methods, for example a lecture and an exercise.
In Darmstadt a course starts at the time mentioned in the schedule. All our courses will start s.t (sine tempore) - expect something else is mentioned explicitly.

}{edited and translated by Johannes Alef, edited by Anna-Katharina Wickert}

\noindent\textbf{Lecture}
\begin{multicols}{2}
\bildmitunterschrift{../grafik/artikel/lul_vorlesung}{width=\linewidth}{}{Andreas Marc Klingler (4)}
\begin{itemize}
	\item Most used method of teaching at the department of computer science.
	\item A Lecturer (a professor or an assistant) stands in front of the students while they listen.
	\item Most lecturers use slide-presentations.
\end{itemize}
\end{multicols}


\noindent\textbf{Exercise}
\begin{multicols}{2}
\bildmitunterschrift{../grafik/artikel/lul_uebung}{width=\linewidth}{}{}
\begin{itemize}
	\item Used for getting experience and a deeper understanding of the subject.
	\item Exercises are worked on by small groups of students supervised by a tutor.
	\item Apply the knowledge gained from the lecture.
\end{itemize}
\end{multicols}

\newpage

\noindent\textbf{Practical Lab}
\begin{multicols}{2}
\bildmitunterschrift{../grafik/artikel/lul_praktikum}{width=\linewidth}{}{}
\begin{itemize}
	\item Used for getting "practical" abilities.
	\item Alone or in groups.
	\item Often your results may be tested by a tutor.
\end{itemize}
\end{multicols}


\noindent\textbf{Office Hours for consultation}
\begin{multicols}{2}
\bildmitunterschrift{../grafik/artikel/lul_sprechstunde}{width=\linewidth}{}{}
\begin{itemize}
	\item Office hours are offered for most lectures.
	\item Don't be afraid to ask "stupid" questions!
	\item But prepare for the office hours so you know what to ask.
	\item Often no prior registration is necessary.
\end{itemize}

\end{multicols}
