\artikel{Die Ordnung des Studiengangs Autonome Systeme}
{Dieser Abschnitt beschäftigt sich speziell mit der Ordnung des Studiengangs Autonome Systeme.
}{
Die folgenden Angaben sind wie immer ohne Gewähr. Verbindlich sind nur die offiziellen Versionen der Ordnung und der Allgemeinen Prüfungsbestimmungen. Verbindliche Informationen geben außerdem die (Fach-)Studienberatung, der*die Dekan*in, die*der Studiendekan*in und das Studienbüro.\\

\noindent\textbf{Autonome Systeme}

Der Studiengang Autonome Systeme wird vom Fachbereich Informatik angeboten. Die Fachbereiche Maschinenbau und Elektro- und Informationstechnik beteiligen sich am Studiengang. Der Fokus des Studiengangs liegt auf der Entwicklung und Erforschung intelligenter und autonom agierender Systeme.

Der Koordinator des Studiengangs Autonome Systeme ist Prof. Oskar von Stryk.\\

\noindent\textbf{Wahlbereiche}

Der Studiengang Autonome Systeme umfasst fünf Wahlbereiche. In jedem Bereich sind mindestens 12 und maximal 37 CP zu erbringen, in allen fünf Bereichen zusammen 70 - 73 CP. Diese Bereiche sind Sense (Sensortechnologie und Sensordatenverarbeitung), Act (Mechatronik, Robotik), Plan (Lernverfahren, KI) und Basis Technologie (Eingebette System, Software Engineering, ...). Der fünfte Wahlbereich E,Studienbegleitende Leistungen umfasst Seminar, Praktikum, Projektpraktikum, Projekt, Studienarbeit und Praktika in der Lehre. In diesem Bereich müssen 17 bis 20 CP erbracht werden. Es muss mindestens eins, maximal zwei Seminare besucht werden. Außerdem muss mindestens eine der Formen Praktika, Projektpraktika und ähnlicher Veranstaltungen besucht werden. Es kann höchstens ein Praktikum in der Lehre eingebracht werden.\\

}
{Johannes Alef}
