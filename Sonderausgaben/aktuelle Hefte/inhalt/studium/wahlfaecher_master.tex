\artikel{Wahlfach-Optionen in diesem Semester}
{Hier findest du eine Liste der Fächer, die für die einzelnen Spezialisierungs-Master in diesem Semester angeboten werden.}
{
    Alle Angaben in diesem Artikel ohne Gewähr und ohne Anspruch auf Vollständigkeit. Dieser Artikel ist hauptsächlich als Übersicht für diejenigen Studierenden gedacht, die zum Zeitpunkt der Ophase noch keinen TUCaN-Zugang haben.
    \\\\
    \textbf{Autonome Systeme}

    \textit{Wahlbereich Act:}
    \begin{itemize}[noitemsep]
        \item Flugmechanik II: Flugdynamik
        \item Mechatronische Systemtechnik II
        \item Technische Mechanik für Elektrotechniker
        \item Aktorwerkstoffe und -prinzipien
        \item Mechatronik und Assistenzsysteme im Automobil
        \item Systemdynamik und Regelungstechnik II
        \item Digitale Regelungssysteme I
        \item Digitale Regelungssysteme II
    \end{itemize}
    \textit{Wahlbereich Basis Technologies:}
    \begin{itemize}[noitemsep]
        \item Microprocessor Systems
        \item Management für Ingenieure
        \item Echtzeitsysteme
        \item Virtuelle und Erweiterte Realität
        \item Netzsicherheit
    \end{itemize}
    \textit{Wahlbereich Plan:}
    \begin{itemize}[noitemsep]
        \item Grundlagen der Navigation I
        \item Optimierung statischer und dynamischer Systeme
        \item Einführung in die Künstliche Intelligenz
        \item Statistisches Maschinelles Lernen
        \item Serious Games
    \end{itemize}
    \textit{Wahlbereich Sense:}
    \begin{itemize}[noitemsep]
        \item Grundlagen der Messtechnik und Datenerfassung mit LabVIEW
        \item Elektrische Messtechnik
        \item Elektronische Sensoren
        \item Bildverarbeitung
        \item Computer Vision II
        \item Capturing Reality
    \end{itemize}
    \textit{Studienbegleitende Leistungen:}
    \begin{itemize}[noitemsep]
        \item Projektseminar Robotik und Computational Intelligence
        \item Robotik-Projektpraktikum
        \item Integriertes Robotik-Projekt 2
        \item Praktikum aus Künstliche Intelligenz
        \item Lernende Roboter: Integriertes Projekt - Teil 1
        \item Lernende Roboter: Integriertes Projekt - Teil 2
        \item Bewegungswissenschaft Proseminar
        \item Seminar aus Data Mining und Maschinellem Lernen
        \item Seminar Fortgeschrittene Themen in Computer Vision und Maschinellem Lernen
    \end{itemize}

    \noindent\textbf{Internet- und Web-basierte Systeme}

    \textit{Pflichtbereiche:}
    \begin{itemize}[noitemsep]
        \item Kommunikationsnetze I
        \item Web Mining
        \item TK3: Ubiquitous / Mobile Computing
        \item Natural Language Processing and eLearning
    \end{itemize}
    \textit{Wahlbereich Internet-basierte Systeme:}
    \begin{itemize}[noitemsep]
        \item Serious Games
        \item Sichere Mobile Systeme
        \item Concepts and Technologies for Distributed Systems and Big Data Processing
    \end{itemize}
    \textit{Wahlbereich Web-basierte Systeme:}
    \begin{itemize}[noitemsep]
        \item Algorithmische Modellierung / Grundlagen des Operations Research
        \item Virtuelle und Erweiterte Realität
        \item Einführung in die Künstliche Intelligenz
        \item Statistisches Maschinelles Lernen
        \item TK2: Human Computer Interaction
        \item Deep Learning für Natural Language Processing
    \end{itemize}
    \textit{Studienbegleitende Leistungen:}
    \begin{itemize}[noitemsep]
        \item Praktikum Multimedia Kommunikation II
        \item Projektpraktikum Multimedia Kommunikation II
        \item Internet - Praktikum Telekooperation
        \item Praktikum Algorithmen
        \item Praktikum Algorithmen II (Vertiefung)
        \item Praktikum aus Künstliche Intelligenz
        \item Projektpraktikum Telekooperation
        \item Projektpraktikum Sichere Mobile Netze
        \item Advanced User Interfaces
        \item Seminar Multimedia Kommunikation II
        \item Seminar aus Data Mining und Maschinellem Lernen
        \item Seminar Telekooperation
        \item Seminar Visual Analytics: Interaktive Visualisierung sehr großer Datenmengen
        \item Forschungsseminar zu Netzen, Sicherheit, Mobilität und Drahtloser Kommunikation
        \item Text Analytics
        \item Seminar Multimedia-Technologie
        \item UKP Oberseminar
    \end{itemize}

    \noindent\textbf{IT-Sicherheit}

    \textit{Pflichtbereich:}
    \begin{itemize}[noitemsep]
        \item IT Sicherheit
        \item Embedded System Security
    \end{itemize}
    \textit{Wahlbereich Cryptography:}
    \begin{itemize}[noitemsep]
        \item Public Key Infrastrukturen
        \item Kryptoplexität
        \item Post-Quantum Kryptographie
    \end{itemize}
    \textit{Wahlbereich Selected Complementary Topics:}
    \begin{itemize}[noitemsep]
        \item Bioinformatik BB 36 VL+Ü
        \item Kommunikationsnetze I
        \item Echtzeitsysteme
        \item Software-Produktlinien – Konzepte, Analyse und Implementierung
        \item Graphische Datenverarbeitung II
        \item Web Mining
        \item Algorithmische Modellierung / Grundlagen des Operations Research
        \item TK3: Ubiquitous / Mobile Computing
        \item Bildverarbeitung
        \item Virtuelle und Erweiterte Realität
        \item Optimierung statischer und dynamischer Systeme
        \item Einführung in die Künstliche Intelligenz
        \item Statistisches Maschinelles Lernen
        \item Serious Games
        \item Computer Vision II
        \item Natural Language Processing and eLearning
        \item Medizinische Visualisierung
        \item Capturing Reality
        \item TK2: Human Computer Interaction
        \item Fortgeschrittener Compilerbau
        \item An introduction to TLA+
        \item Multithreading in C++
    \end{itemize}
    \textit{Wahlbereich Software Security:}
    \begin{itemize}[noitemsep]
        \item Sicherheit in Multimedia Systemen und Anwendungen
        \item Formale Methoden der Informationssicherheit
        \item Formale Spezifikation und Verifikation von Software
    \end{itemize}
    \textit{Wahlbereich System Security:}
    \begin{itemize}[noitemsep]
        \item Seitenkanalangriffe gegen IT-Systeme
        \item Netzsicherheit
        \item Sichere Mobile Systeme
        \item Sicherheitskonzepte im Eisenbahnbetrieb 2
    \end{itemize}
    \textit{Studienbegleitende Leistungen:}
    \begin{itemize}[noitemsep]
        \item Kryptographie
        \item Implementierung von Programmiersprachen
        \item Projektpraktikum Sichere Mobile Netze
        \item Implementierung in Forensik und Mediensicherheit
        \item Usable Security und Usable Privacy
        \item Zuverlässige Softwaresicherheit für mobile Endgeräte
        \item Langzeitsicherheit
        \item Software Dependability Praktikum
        \item Forschungsseminar zu Netzen, Sicherheit, Mobilität und Drahtloser Kommunikation
        \item Seminar zu Netzen, Sicherheit, Mobilität und Drahtloser Kommunikation
        \item Seminar Privacy by Design
        \item Seminar Mobile Security
        \item Seminar Sicherheit von SDN und NFV
        \item Cyber Security Seminar
        \item Seitenkanalangriffe gegen Software
        \item Privatheit und Anonymität in einer vernetzten Welt
        \item Seminar Zivile Sicherheit
    \end{itemize}

    \noindent\textbf{Visual Computing}

    \textit{Pflichtbereich:}
    \begin{itemize}[noitemsep]
        \item Graphische Datenverarbeitung II
        \item Statistisches Maschinelles Lernen
    \end{itemize}
    \textit{Wahlbereich Anwendungen:}
    \begin{itemize}[noitemsep]
        \item Numerische Lineare Algebra
        \item Nichtglatte Optimierung
        \item Angewandte Geometrie
        \item Bioinformatik BB 36 VL+Ü
        \item Fernerkundung I
        \item Fernerkundung II
        \item Microprocessor Systems
        \item Echtzeitsysteme
        \item Sicherheit in Multimedia Systemen und Anwendungen
        \item Optimierung statischer und dynamischer Systeme
        \item Einführung in die Künstliche Intelligenz
        \item Serious Games
        \item TK2: Human Computer Interaction
    \end{itemize}
    \textit{Wahlbereich Computer Vision und Maschinelles Denken:}
    \begin{itemize}[noitemsep]
        \item Bildverarbeitung
        \item Computer Vision II
    \end{itemize}
    \textit{Wahlbereich Integrierte Methoden von Vision und Graphik:}
    \begin{itemize}[noitemsep]
        \item Virtuelle und Erweiterte Realität
        \item Medizinische Visualisierung
        \item Capturing Reality
        \item User-Centered Design in Visual Computing
    \end{itemize}
    \textit{Studienbegleitende Leistungen:}
    \begin{itemize}[noitemsep]
        \item Serious Games Praktikum
        \item Praktikum Visual Computing
        \item Fortgeschrittenes Praktikum Visual Computing
        \item Serious Games Projektpraktikum
        \item Lernende Roboter: Integriertes Projekt - Teil 1
        \item Projektpraktikum Programmierung Massiv-Paralleler Systeme
        \item Projektpraktikum Capturing Reality
        \item Seminar aus Data Mining und Maschinellem Lernen
        \item Visual Analytics: Interaktive Visualisierung sehr großer Datenmengen
        \item Serious Games Seminar
        \item Aktuelle Trends in Medical Computing
        \item Skalenraum- und PDE-Methoden in der Bildanalyse und -verarbeitung
        \item Fortgeschrittene Themen in der Computergraphik
        \item Fortgeschrittene Themen in Computer Vision und Maschinellem Lernen
        \item Seminar Multimedia-Technologie
        \item Angewandte Themen der Computergraphik
    \end{itemize}
}
{Stefan Gries}
