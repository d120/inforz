\artikel{Das Wesen der Informatik...}{
    ...ist seit vielen Jahren das Maskottchen der Fachschaft Informatik. Es ist ein kleines Baby, das mit einem unschuldigen Grinsen auf einem Hocker sitzt -- mit einem Sturmgewehr in der Hand. Da kommt bei manchen wohl gleich der Gedanke an die \glqq bösen Killerspiele\grqq.  Doch weit gefehlt: Das symbolische Wesen der Informatik fand sich zum ersten Mal im Jahr 1986 auf der Zeitschrift Inforz, also zu einer Zeit, als der Begriff des Ego-Shooters noch gar nicht existierte.}
{Die Intention damals war eine andere: Was passiert, wenn man einem Baby eine Schusswaffe in die Hand drückt? Keiner weiß es genau, aber jeder hat ein ungutes Gefühl dabei.
    Die Informatik ist jung, vor allem im Vergleich zu anderen Wissenschaften. Physik, Mathematik, Biologie und Philosophie sind seit mehreren hundert bis tausenden von Jahren als eigenständige Disziplinen anerkannt.  Im Gegensatz dazu existiert allein der Begriff der \glqq Informatik\grqq~erst seit Mitte des 20. Jahrhunderts und als eigenständige wissenschaftliche Disziplin wurde sie erst in den späten 1960er-Jahren etabliert.
    Wer konnte also 1986, gerade mal etwas über ein Jahrzehnt nach der Entstehung des Fachbereichs Informatik an deutschen Universitäten, schon die Konsequenzen abschätzen, die diese neue Wissenschaft mit sich bringen würde?\\

    \textbf{Was ist Informatik?}

    Laut Wikipedia ist die Informatik die \glqq Wissenschaft der systematischen Verarbeitung von Informationen, insbesondere der automatischen Verarbeitung mit Hilfe von Digitalrechnern\grqq.
    Das klingt erstmal nicht besonders spannend oder gefährlich. Das Ganze dient dazu, Probleme einfacher und effizienter  zu lösen als es bisher möglich war - und das so automatisiert wie möglich. Einfach, effizient, automatisiert - das sind schon ganz nette Eigenschaften, die einem Menschen das Leben leichter machen können. Da kann doch eigentlich nichts schiefgehen.\\

    \columnbreak

    \textbf{Beispiel: Wegfindung}

    Eines der bekanntesten und ältesten Probleme in der Informatik ist die Wegfindung. Ziel ist es, von einem Startpunkt A möglichst günstig zum Ziel B zu gelangen. \glqq Möglichst günstig\grqq kann in diesem Fall einiges sein: kürzester Weg, schnellster Weg, Weg mit den angenehmsten Bedingungen oder auch Weg mit den meisten Sehenswürdigkeiten auf der Strecke. Eine Anwendung sind ganz offensichtlich Navigationssysteme für Autos oder andere Transportmittel wie Schiffe und Flugzeuge, um dem Menschen, der diese Geräte steuert, Zeit- und Geldeinsparungen zu ermöglichen. Aber auch schwer bewaffnete Kampfdrohnen verlassen sich auf Wegfindungsalgorithmen, um einfach, effizient und automatisiert gesuchte Terroristen oder andere unliebsame Personen liquidieren zu können.\\

    \bildmitunterschrift{../grafik/wesen/wesen_transparent}{width=\columnwidth}{}{}

    \textbf{Ein 2. Beispiel: Roboter}

    Roboter werden schon sehr lange in der Industrie eingesetz, um Tätigkeiten zu übernehmen, die für Menschen zu schwer oder gefährlich sind. Die Automobilindustrie ist hier ein gutes Beispiel. Stationäre Roboterarme wuchten hier schwere Teile durch die Gegend, setzen sie zusammen und überprüfen bisweilen auch schon in Zusammenarbeit mit vernetzten Sensoren verschiedene Qualitätseigenschaften der resultierenden Werkstücke. Der Mensch verkommt in diesen zunehmend digitalisierten Arbeitsumgebungen beisweilen sogar schon eher zum Störfaktor, denn dessen manuelle Arbeit ist bei weitem nicht so effizient wie die, die von Automaten verrichtetet wird. Mal ganz davon abgesehen, dass Menschen im Gegensatz zu Robotern ungünstigerweise auch noch schlafen müssen und trotzdem noch für ihre Arbeit Geld verlangen...\\

    \textbf{Noch ein Beispiel: Machine Learning und KI}

    Mit steigendem Rechen- und Speichervermögen von Computern bieten sich der Wissenschaft umfangreichere Möglichkeiten zur Datenauswertung als jemals zuvor. Dank künstlicher Intelligenz und maschinellen Lernverfahren können heutzutage Terabytes an Daten, wie sie beispielsweise in den Sensoren von Partikelbeschleunigern auftreten, innerhalb kurzer Zeit ausgewertet werden und ermöglichen somit ganz neue Forschungsfelder, zum Beispiel eben in der Partikelphysik, oder auch in der Hirnforschung. Aber nicht nur in den Laboren der Wissenschaft fallen große Datenmengen an. Spätestens seit Verbreitung des Internets befüllen auch Staaten und Dienstleistungsunternehmen beständig wachsende Datensammlungen, die mittels Lernalgorithmen auf Zusammenhänge untersucht werden. Das kann Menschen, die diese Dienste nutzen, das Leben sehr einfach machen, da Computer auf diese Weise immer besser verstehen lernen, worauf es einzelnen Nutzern ankommt -- zugleich versuchen jedoch die hinter diesen Diensten stehenden Unternehmen oder Staaten zunehmend, dieses Wissen um die individuellen Vorlieben gewinnbringend zu nutzen. Denn je besser man seine Kunden (oder Bürger) kennt, desto einfacher wird es, diese mit minimalem Aufwand und ohne menschliches Zutun zu manipulieren.\\

    \textbf{Informatik und Gesellschaft}

    Im Laufe von nur wenigen Jahrzehnten hat es die Informatik vollbracht, nahezu in jeden Bereich der Gesellschaft vorzudringen. Anhand der obigen Beispiele (und man könnte noch viele, viele mehr nennen) soll verdeutlicht werden, wie groß der Einfluss der Informatik auf unseren Alltag mittlerweile ist. In vielen Fällen handelt es sich um Errungenschaften, die unser Leben angenehmer machen und erleichtern, aber oftmals kommen diese mit Kosten, die wir bisweilen nicht direkt einsehen oder abschätzen können -- auch als Informatiker*innen nicht. Wir sind also vielmals sprichwörtlich die unbedarften Kinder, die mit der Schusswaffe spielen, wenn wir maßgeblich zu einer neuen technologischen Errungenschaft beitragen. Vielfach tendieren wir nämlich dazu, nur die wünschenswerten Einsatzzwecke in Betracht zu ziehen und den Schaden zu übersehen, der gleichzeitig mit der Technologie in useren Händen angerichtet werden kann. Die Informatik ist also weit mehr als nur eine (Ingenieurs-)Wissenschaft, denn letztendlich werden insbesondere wir Informatiker*innen diejenigen sein, die zukünftig die Richtung vorgeben, in die sich unsere Gesellschaft bewegt -- eine große Chance, aber auch eine große Bürde und eine Tatsache, die uns immer bewusst sein sollte.}
{Tobias Otterbein, Stefan Gries}
\newpage
