\artikel{Ein phantastisches Wesen und wo es zu finden ist}
{Klassifizierung durch die Fachschaft:\\
XXX – Von fähigen Informatiker*innen zu bändigen}
{Das Wesen ist ein kleines Kind mit Maschinenpistole, das auf einem Hocker sitzt. Es hat seinen Urspung in einem Buch namens "`Wo soll das alles enden"' von Gerhard Seyfried. Auffällige Merkmale sind die zwei nach oben abstehenden Stoppelhaare mitten auf dem Kopf, der kurze, dünne Hals und die zwei Milchzähne, die über die Lippe stehen. Einige nennen es auch liebevoll den "`Bengel"'.\\

Von dieser Art existieren einige Unterarten, von denen man vor allem einer regelmäßig begegnet: den Ophasenwesen. Jedes Semester lässt sich eins in der Woche vor Vorlesungsbeginn auf dem Cover des OInforz nieder, während auf den regulären Ausgaben des Inforz jedes Mal eine andere Version posiert.\\

Bis Februar 2016 hielt sich die Fachschaft das Wesen als Logo. Aufgrund häufiger Fehlinterpretationen, darunter der, das Wesen verharmlose Gewalt, hat sie sich inzwischen ein neues Logo zugelegt, das nun die Briefköpfe und Plakate der Fachschaft bewohnt. In diesen Regionen ist das Wesen seitdem ausgestorben.\\

Hauptverbreitungsgebiet ist nach wie vor die Informatiker*innenzeitschrift Inforz. Dort wurde 1986 auch die erste Sichtung bestätigt. In den ersten Jahren nach dieser Sichtung suchte sich das Wesen den Raum über dem Inforz-Schriftzug aus, um sich öffentlich zu präsentieren. Nachdem es an Beliebtheit gewonnen hatte, begann es, sich auch an anderen Stellen zu zeigen. Anscheinend hat es Gefallen daran gefunden, sich zu verkleiden, denn anlässlich der Ophase lässt es sich auch auf T-Shirts und Namensschildern nieder, immer an das Thema der jeweiligen Ophase angepasst. So findet man die ursprüngliche Art mit Maschinenpistole heutzutage eher selten. Ein Exemplar dieser urprünglichen Art hat sich allerdings überreden lassen, sich in diesem Artikel zu präsentieren.

Die ersten Sichtungen im Jahr 1986 stießen bei den Informatiker*innen nicht immer auf Begeisterung. Die übliche Ablehnung bei der Neuentdeckung von Arten wurde durch die Maschinenpistole noch geschürt. So sah sich die Fachschaft genötigt, dem Wesen nachträglich eine Deutung zukommen zu lassen. Das Wesen steht für die noch junge Wissenschaft Informatik, die sich ihrer Auswirkungen auf die Gesellschaft nicht immer bewusst ist und die ungewollt großen Schaden anrichten kann - wie ein Kind mit Maschinenpistole. Diese Deutung hat bis heute Bestand und hat maßgeblich dazu beigetragen, dass das Wesen von den Informatiker*innen akzeptiert und ihm der Lebensraum auf dem Inforz eingeräumt wurde.
Aufgrund ihrer heutigen Beliebtheit ist die Art noch lange nicht vom Aussterben bedroht, obwohl sie von Briefköpfen und Plakaten der Fachschaft verschwunden ist.
Mit seiner Deutung ist das Wesen heutzutage eine Warnung an alle Informatiker*innen, wie die folgenden Beispiele noch einmal verdeutlichen.

\vspace{3mm}

\bildmitunterschrift{../grafik/wesen/wesen_transparent}{width=\columnwidth}{Das Wesen in seiner urprünglichen Form}{}

\textbf{Beispiel: Wegfindung}\\
Eines der bekanntesten und ältesten Probleme in der Informatik ist die Wegfindung. Ziel ist es, von einem Startpunkt A möglichst günstig zum Ziel B zu gelangen. \glqq Möglichst günstig\grqq kann in diesem Fall einiges sein: kürzester Weg, schnellster Weg, Weg mit den angenehmsten Bedingungen oder auch Weg mit den meisten Sehenswürdigkeiten auf der Strecke. Eine Anwendung sind ganz offensichtlich Navigationssysteme für Autos oder andere Transportmittel wie Schiffe und Flugzeuge, um dem Menschen, der diese Geräte steuert, Zeit- und Geldeinsparungen zu ermöglichen. Aber auch schwer bewaffnete Kampfdrohnen verlassen sich auf Wegfindungsalgorithmen, um einfach, effizient und automatisiert gesuchte Terroristen oder andere unliebsame Personen liquidieren zu können.\\

\textbf{Ein 2. Beispiel: Roboter}\\
Roboter werden schon sehr lange in der Industrie eingesetzt, um Tätigkeiten zu übernehmen, die für Menschen zu schwer oder gefährlich sind. Die Automobilindustrie ist hier ein gutes Beispiel. Stationäre Roboterarme wuchten schwere Teile durch die Gegend, setzen sie zusammen und überprüfen hin und wieder auch schon in Zusammenarbeit mit vernetzten Sensoren verschiedene Qualitätseigenschaften der resultierenden Werkstücke. Der Mensch verkommt in diesen zunehmend digitalisierten Arbeitsumgebungen bisweilen sogar zum Störfaktor, denn dessen manuelle Arbeit ist bei weitem nicht so effizient wie die, die von Automaten verrichtet wird. Mal ganz davon abgesehen, dass Menschen im Gegensatz zu Robotern ungünstigerweise auch noch schlafen müssen und Geld für ihre Arbeit verlangen...\\

\textbf{Informatik und Gesellschaft}\\
Im Laufe von nur wenigen Jahrzehnten hat es die Informatik geschafft, in fast alle Bereiche der Gesellschaft vorzudringen. Anhand der obigen Beispiele (und man könnte noch viele, viele mehr nennen) soll verdeutlicht werden, wie groß der Einfluss der Informatik auf unseren Alltag mittlerweile ist. In vielen Fällen handelt es sich um Errungenschaften, die unser Leben angenehmer machen und erleichtern, aber oftmals kommen diese mit Kosten, die wir nicht direkt einsehen oder abschätzen können -- auch als Informatiker*innen nicht. Wir sind also sprichwörtlich die unbedarften Kinder, die mit der Schusswaffe spielen, wenn wir maßgeblich zu einer neuen technologischen Errungenschaft beitragen. Vielfach tendieren wir nämlich dazu, nur die wünschenswerten Einsatzzwecke in Betracht zu ziehen und den Schaden zu übersehen, den die Technologie in unseren Händen anrichten kann. Die Informatik ist also weit mehr als nur eine Wissenschaft, denn letztendlich werden insbesondere wir Informatiker*innen diejenigen sein, die zukünftig die Richtung vorgeben, in die sich unsere Gesellschaft bewegt -- eine große Chance, aber auch eine große Bürde und eine Tatsache, die uns immer bewusst sein sollte.}
{Dorothea Treitz}

\vspace{2mm}

\bildmitunterschrift{../grafik/comics/constructive}{width=\textwidth}{}{xkcd.com}

\newpage
