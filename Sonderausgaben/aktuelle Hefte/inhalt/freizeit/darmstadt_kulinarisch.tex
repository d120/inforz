\artikel{Darmstadt kulinarisch}
{Darmstadt bietet einige Essens- und Ausgehmöglichkeiten. Einige davon verstecken sich aber.
}{
\textbf{Frühstücken…}

Besonders während der vorlesungsfreien Zeit möchtest du sicher gerne einmal mit Kommilitoninnen und Kommilitonen gemütlich frühstücken.

Hier bietet sich das Caf\'e Chaos an, bis 24 Uhr wirst du hier mit frischen Brötchen versorgt. Am Marktplatz befinden sich das Caf\'e Extrablatt sowie Bäckerei Bormuth - beide bieten ein reiches Frühstücksbuffet.

Auf der anderen Seite der Universität gibt es noch das 3klang am Riegerplatz mit einem sonntäglichen Buffet der Spitzenklasse.\\
Für einen gemütlichen Kaffee zwischendurch bietet sich außerdem das 221qm (Teil des 806qm) direkt auf dem Unigelände an (Eingang an der Alexanderstraße).
\\\\
\textbf{Einfach nur essen…}

Wer mittags Hunger bekommt, geht meist in die Mensa, denn dort gibt's brauchbares, günstiges Essen. Aber womit den Magen füllen, wenn die Mensa schon geschlossen hat oder du einfach mal Abwechslung von der Mensaspeisekarte brauchst?

In der Stadtmitte hast du eine große Auswahl an Alternativen: Dönerläden, Asia-Imbisse, Fastfood-Ketten - alle kaum zu übersehen. Bei manchen gibt es sogar spezielle Studierendenangebote.

Noch deutlich näher an der Uni sind Havana, Hotzenplotz (nur abends geöffnet) und Hobbit. Alle Kneipen liegen in der Lauteschlägerstraße (östlich vom Kantplatz), wobei es im Hobbit mittags Pizzen günstiger gibt. Hinter dem Mathebau liegt das Petri (nur abends geöffnet) mit Biergarten und bayerischer Küche. Die Auswahl ist nicht sehr groß, dafür ist das Essen gut.

Wer vegetarische/vegane Küche bevorzugt, dem sei das Café Habibi in der Landwehrstraße (direkt am Willy-Brandt-Platz) und das Mondo Daily in der Grafenstraße ans Herz gelegt. Im Habibi gibt es zu studentischen Preisen eine Vielzahl an vegetarischen und veganen Gerichten in schöner Atmosphäre zwischen Darmstädter Altbauten, während man im Mondo Daily ein täglich wechselndes Angebot an orientalischen Gerichten genießen kann.
Beide bieten auch einen günstigen Mittagstisch an.

Zuletzt ein wahrer Geheimtipp für Suppenliebhaber: Suppkult Elisabeth in der Schulstraße.\\

\textbf{Etwas trinken…}

Für ein (oder mehr) Bier am Abend bieten sich Pubs wie das Green Sheep in der Erbacher Straße an. Hier gibt es außerdem von 18 bis 20 Uhr Pizza zum halben Preis.

Wer es weniger "`englisch"' mag, kann das Schlossgartencaf\'e aufsuchen: es befindet sich direkt auf der Bastion am Schloss (über die Brücke und dann gleich rechts).
Ein weiteres Urgestein der Darmstädter Kneipenkultur ist die "`Goldene Krone"'. Hier ist es immer voll, die Stimmung ist immer gut und es gibt neben Musik und verschiedenen Events (fast) die ganze Nacht Getränke zu günstigen Preisen.

Draußen sitzen kann man im Sommer im Biergarten Lichtwiesn direkt bei der Mensa Lichtwiese, sowie im Biergarten Darmstadt in der Dieburger Straße.

Wenn du Bier lieber direkt von der Brauerei trinken möchtest, hast du in Darmstadt große Auswahl: die Grohe-Brauerei an der Nieder-Ramstädter Straße, den Ratskeller am Marktplatz und das Braustüb'l am Hauptbahnhof warten auf deinen Besuch.

Für Cocktail-Liebhaber empfiehlt sich das Enchilada (mexikanisch, Happy Hour bis 20 Uhr und montags Cocktailwürfeln) und das Corroboree (australisch, montags Cocktails für die Hälfte) in der Kasinostraße (Haltestelle Rhein-/Neckarstraße). Außerdem gibt es noch die Havana-Bar in der Lauteschlägerstraße und das Sausalitos (Happy Hour bis 20 Uhr) in der Nähe von S3$|$06.
}
{Tobias Freudenreich, Martin Tschirsich, Stefan Gries,
überarbeitet von Julian Haas}

\newpage
