\artikel{Sport}
{Wie keine andere Freizeitaktivität eignet sich Sport dazu, den Kopf frei zu bekommen und die Kreativität zu fördern: Gesellig ist es allemal, die Sportart sollte allerdings zu den eigenen Bedürfnissen passen.
}{
    Wer wettkampforientiert ist, tendiert eher zu Ball- und Kampfsportarten, wer beim Sport treiben lieber seine Ruhe hat und die Natur genießen möchte, fährt mit dem Rad zur Burg Frankenstein oder geht im Park joggen.
    Insbesondere auch für die Unentschlossenen bietet sich ein Blick in den Katalog des Unisport-Zentrums (USZ) an - die perfekte Anlaufstelle für Aktiv- und Gelegenheitssportler*innen. Das Unisport-Zentrum bietet für alle Studierenden und Angestellten rund 250 Sportangebote in 90 Sportarten pro Woche. Von Fitnessveranstaltungen wie Aerobic oder Schwitz-Fit über Ballsportarten wie Badminton und Fußball bis hin zu den außergewöhnlicheren Sportarten wie Einradhockey, Kanupolo, Unterwasser-Rugby oder Quidditch ist fast alles vertreten.
    Das größtenteils kostenlose Hochschulsportangebot wird jedes Semester in einem Programm-Handzettel und im Internet unter der Adresse \footnotemark[1]  veröffentlicht, dort findet sich auch eine Online-Anmeldung für alle Kurse. Das Unisport-Zentrum betreibt zudem eine eigene Golf-Übungsanlage und das Sport- und Gesundheitszentrum, ein Fitnessstudio für Studierende und Angestellte. Neben diesen permanenten Einrichtungen werden zusätzlich noch einzelne Workshops wie Tauchen oder Stepptanz angeboten.
    Am besten gehst du einfach hin und meldest dich kurz nach Semesterbeginn an, lediglich einige spezielle Kurse verlangen zusätzlich die Zahlung einer geringen Gebühr. Das beliebteste Angebot war in den vergangenen Semestern das Uni-Freibad am Hochschulstadion. Darüber hinaus führt das studentische Sportreferat in jedem Semester interne Hochschulmeisterschaften (IHM) in verschiedenen Sportarten wie Fußball, Badminton, Tischtennis und Volleyball durch. Wettkampfinteressierte Studierende können außerdem an den Deutschen Hochschulmeisterschaften (DHM) teilnehmen. Die Ausschreibungen und Meldetermine findest du auf den Internetseiten des USZ (IHM) oder unter \footnotemark[2] (DHM).
    Leider sind einige Angebote des USZ überlaufen und eignen sich tatsächlich nur zum Kennenlernen. Hier bietet es sich dann an, einem der lokalen Sportvereine beizutreten. Aus Platzgründen können wir hier keine Übersicht geben, aber eine kurze Suche im Internet führt hier schnell zum Erfolg. Oft bieten diese Vereine für Studierende auch vergünstigte Beiträge an.
    Solltest du bisher noch nicht fündig geworden sein, warten in Darmstadt neben der Eissporthalle und einem Kletterwald am Hochschulstadion noch diverse Parks und weitere Schwimmbäder sowie viele andere Angebote auf dich.
    Dazu kommen Gruppen wie das Juggerteam von Darmstadt, Pink Pain [3], die ganz ohne Verein Sport machen.
}
{Tobias Freudenreich, Martin Tschirsich, Stefan Gries}

\footnotetext[1]{\url{https://online-anmeldung.usz.tu-darmstadt.de/sportarten/aktueller_zeitraum}}
\footnotetext[2]{\url{http://www.adh.de}}
\footnotetext[3]{\url{http://www.jugger-darmstadt.de}}
\newpage
