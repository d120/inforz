\artikel{GnoM - The LAN party without a power supply}
{Travel through time with the Fachschaft of computer science and rediscover long lost traditions.
}{
    When even the most ingrained computer scientists leave their coffee cups behind and move towards the upper levels, away from the computer pool, there must be something special going on.
    They're migrating towards Room E202 in the Piloty building, armed with candy and mysterious cardboard boxes without USB-Ports or power supplies.
    Outsiders might wonder what computer scientists may intend to do with such boxes. The answer is easy: They just want to play.
    Upon arrival in room E202, even mathematics and physics students can sometimes be seen around.
    Some long-term student may feel like being transported back in time, back to days before the invention of electricity, and is tempted to light a candle.
    But that is not necessary: the no-power-supply rule only applies to entertainment electronics. This is "Games no Machines", or in short -- GnoM.
    For $10100_2$ years already, the Fachschaft of computer science organizes this legendary evening where PCs are left in the pool, and only good old board and card games are being pulled off the dusty shelf.
    Everyone is welcome and invited to participate and see for themselves how much fun a LAN party without power can be.
    Of course, one can bring their own games as well.
    More information about the scheduled dates can be obtained from the GnoM mailing list \footnotemark[1], the Fachschaft's forums \footnotemark[2], the blog "das Wesentliche" \footnotemark[3] and especially (as these websites are mostly in German) everywhere helpful members of the Fachschaft can be found, primarily in the Fachschaft's room, D120.
    General information, again in German, can be found here: \footnotemark[4].
}
{
    Alexandra Weber\\edited by Julian Haas\\edited and translated by Nils M.\\translation edited by Stefan Gries
}

\footnotetext[1]{\url{http://www.d120.de/mailman/listinfo/gnom}}
\footnotetext[2]{\url{https://www.fachschaft.informatik.tu-darmstadt.de/forum/}}
\footnotetext[3]{\url{http://daswesentliche.d120.de}}
\footnotetext[4]{\url{http://www.d120.de/gnom}}

\artikel{RPGnoM}
{GnoM's little sibling}
{For $1_2$ year already, GnoM (again) has a little brother: RPGnoM.
    This group focuses on role-playing games, the respective gaming evenings take place in the same rooms as the regular GnoM.
    The primary difference between board games and role-playing games is that, while still based on a set of game rules, role-playing games emphasise group-focused interactive storytelling over traditional competitive game mechanisms.
    RPGnoM evenings are sheduled via survey and then a date will be published by means of the dedicated mailing list \footnotemark[5].
    On most dates, the participants organize an order of food so that players don't go hungry.
    General information, again in German, can be found here: \footnotemark[6].
}
{
    Anna-Katharina Wickert,\\Stefan Gries
}

\footnotetext[5]{\url{http://www.d120.de/mailman/listinfo/rpgnom}}
\footnotetext[6]{\url{https://www.d120.de/de/studierende/rpgnom/}}

%\vfill
%\bildmitunterschrift{../grafik/comics/nerd_sniping}{width=\textwidth}{}{xkcd.org}
%\newpage
