\artikel{Sports}
{Exercise is a great way to clear your head and promote creativity: It usually is very social but everyone has to find their own favorite type of exercise.
}{
Someone feeling competitive might enjoy ball sports or martial arts more than someone who'd rather have a quiet time and takes the bike to Castle Frankenstein or goes jogging.
Still undecided? Take a look at the catalogue of the University Sports Center. They offer about 90 different sports courses and exercise activities every week for students and employees of the TU Darmstadt. Some examples are fitness courses like Aerobic or Zumba, ball sports like badminton or Football but also exotic sports like Canoe-Polo, Underwater-Rugby or Ultimate Frisbee.
Some courses may cost a small fee but most of them are free of charge to participate. You can find registration forms and more information here: \footnotemark[1] . Unfortunately, the website is currently only offered in German, so if you need help registering or getting more information, just ask anyone in the Fachschaft (Room D120 in the Piloty-Building) or someone else that understands German. The University Sports Center also operates a golf practice course and a gym. From time to time they offer exclusive workshops like diving or step dance.
One of the more popular offers is the outdoor pool, especially during the summer, which is free of charge for students.
Each term there are internal championships in various sports like Football, Badminton or Volleyball. Those are more for fun but if you want to get real competitive, you can try-out for participation in the German University Sports Federation Championships. Information about these championships can be found on the websites of the University Sports Center \footnotemark[1] and the German University Sports Federation \footnotemark[2] respectively.
Due to limited space the gym fills up quickly, so register early if you want to get a membership! Other offers do fill up quickly, too in which case it might be preferable to join a local sports team instead. Often, these clubs offer memberships at discounted prices for students as well.
Darmstadt also offers a skating rink and a climbing forest as well as several parks and indoor/outdoor swimming pools.
If you're up for a challenge, you can join the local Jugger team, Pink Pain \footnotemark[3] . Their web presence is only in German but the team consists of quite a few computer science students and we're always happy to introduce you to the world of Jugger. 
}
{Tobias Freudenreich, Martin Tschirsich, Stefan Gries \\ edited and translated by Nils M.}

\footnotetext[1]{\url{http://www.usz.tu-darmstadt.de}}
\footnotetext[2]{\url{http://www.adh.de/en.html}}
\footnotetext[3]{\url{http://www.jugger-darmstadt.de}}
\newpage
