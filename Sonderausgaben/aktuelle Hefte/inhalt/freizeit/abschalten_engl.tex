\artikel{All switches off}
{One of the most pleasant options to spend one's leisure time is to turn all switches off and just relax which is especially pleasant outside on warmer days during the summer.
}{
Darmstadt's inhabitants find quite a few hidden gems in their surroundings that sometimes even the mature students don't know about: In the north of the city lies the Citizens Park (German: Bürgerpark) right behind the Nordbad (an indoor (winter)/outdoor (summer) swimming pool), in the south at Heidelberger Street is the Prince-Emil-Garden and the Orangerie, and close to the train station to the east (Ostbahnhof) one can find the Zoo "Vivarium" and the Rosenhöhe (another smaller park).
The Herrngarten, Darmstadt's largest park can't be missed by computer science students as it is right behind the Piloty-Building. Another point of interest is the Mathildenhöhe (park) featuring the towns landmark, the wedding tower, and its regular art and cultural program.

During the hot summer months you can cool down in one of the many swimming pools and lakes in and around Darmstadt. Most interesting for students beside the above mentioned Nordbad (very cheap for students) is the university's own outdoor pool located in the south right beside the University Sports Center. This pool is free of charge for students (bring your student ID) and is in range of the university's WiFi network. Perfect for soaking up the sun while studying a bit and going for a dip every once in a while.
If you prefer to swim in lakes, the Woog, located fairly central in the city, is for you. There is also a few other lakes outside of Darmstadt, namely the Arheilige Mühlchen and the Grube Prinz von Hessen, both free of charge.
}
{Tobias Freudenreich, Martin Tschirsich, Stefan Gries \\ edited and translated by Nils M.}
\newpage