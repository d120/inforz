\artikel{GnoM - Die LAN-Party ohne Strom}
{Reise mit der Fachschaft Informatik durch die Zeit und entdecke längst vergessene Traditionen wieder.
}{
    Wenn selbst die eingefleischtesten Informatiker*innen abends ihre Kaffeetassen vom Computerpool in überirdische Räume bewegen, dann muss dort schon etwas ganz besonderes geboten werden. In Scharen wandern sie zum Raum E202, beladen mit Süßigkeiten und mysteriösen Kisten ohne USB-Anschlüsse und Netzteile. Außenstehende mögen sich nun vielleicht fragen, was Informatikstudierende mit so einer Kiste anfangen wollen. Die Antwort ist einfach: Die wollen doch nur spielen.
    Am Raum E202 angekommen sieht man auch Mathematik- und Physikstudierende ins Piloty-Gebäude strömen. Manche Langzeitstudent*innen fühlen sich zurückversetzt in ihre Jugendzeit vor der Erfindung der Elektrizität und sind schon versucht, eine Kerze anzuzünden, doch das ist nicht nötig, denn das Anti-Strom-Gebot erstreckt sich nur auf die Unterhaltungsmedien. Hier ist das Motto "`Games no Machines"' (kurz GnoM) Programm.
    Bei diesem legendären Spieleabend der Fachschaft Informatik werden schon seit $10101_2$ Jahren die guten alten Brett- und Kartenspiele hervorgekramt und die PCs im Pool gelassen.
    Alle Informatik-, Physik- und Mathematikstudierenden und auch andere Lebensformen sind herzlich eingeladen, sich selbst davon zu überzeugen, wie viel Spaß eine LAN-Party ohne Strom machen kann. Eigene Spiele dürfen dabei auch gerne mitgebracht werden.\\
    Die erste Möglichkeit, diese ungewöhnliche Erfahrung zu machen, hast du übrigens am Freitag der Ophase in C301, dem Raum über dem Bistro Athene.\\
    Mehr zu den Terminen gibt es über die GnoM-Mailingliste \footnotemark[1], im Fachschaftsblog "`das Wesentliche"' \footnotemark[2] und sonst überall, wo sich hilfsbereite Fachschaftler*innen aufhalten. Generelle Infos findet man zudem auch unter \footnotemark[3].
}
{Alexandra Weber,\\
    überarbeitet von Julian Haas}

\footnotetext[1]{\url{http://www.d120.de/mailman/listinfo/gnom}}
\footnotetext[2]{\url{http://daswesentliche.d120.de}}
\footnotetext[3]{\url{http://www.d120.de/gnom}}

\vfill
\bildmitunterschrift{comics/nerd_sniping}{width=\textwidth}{}{xkcd.org}
\newpage
