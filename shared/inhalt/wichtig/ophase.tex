\artikel{Die Orientierungsphase}
{Die Woche vor Beginn der Vorlesungen könnte die Wichtigste Deines Studiums sein.
}{Sofern Du nicht schon einmal studiert hast, dürften Dir die Abläufe und Regelungen an der Uni kaum geläufig sein. Es gibt sehr viele Dinge, die im Studium anders laufen als beispielsweise in der Schule – von selbst dahinterzukommen ist sehr zeitaufwändig und, wenn man nicht weiß, wo man suchen muss, oftmals auch äußerst schwierig. Um Dir die Sucherei zu ersparen und Dir den Einstieg ins Studium, sowie die Orientierung an der Uni zu erleichtern, gibt es die Orientierungsphase für Erstsemester (Ophase). In dieser etwa eine Woche dauernden Veranstaltung erklären Dir Studierende, die schon seit ein paar Semestern an der Uni sind und sich entsprechend auskennen, was Du übers Studium wissen musst, beispielsweise was eigentlich CPs sind, wie Prüfungen an der Uni ablaufen, welche Fächer und Inhalte auf Dich zukommen, wie Du Deinen Stundenplan zusammenstellst und noch vieles mehr. Nebenbei finden in dieser Zeit auch viele soziale Veranstaltungen statt, in denen es weniger um Wissensvermittlung geht, sondern viel mehr darum, dass Du Deine künftigen Mitstudierenden (Kommiliton*innen) besser kennen lernen kannst.

    So oder so kann Dir die Ophase also für Dein Studium viel bieten. Im kommenden Semester findet sie vom \ophaseAnfang ~ bis zum \ophaseEnde ~ \the\year ~ statt, weitere Informationen findest Du unter \url{www.D120.de/ophase}.

}{}
