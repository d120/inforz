\artikel*{QR-Code zur Ophasenwebsite}{Willkommen an der Universität, Willkommen zur Ophase}%
{
    Die Fachschaft Informatik heißt dich zu deinem Studienbeginn herzlich willkommen!

    Du wirst es voraussichtlich schon an mehreren Stellen gelesen und gehört haben und auch an dieser Stelle wird es dir nicht erspart: Herzlich willkommen im Studium, willkommen an der Technischen Universität Darmstadt, willkommen zum Beginn eines neuen Lebensabschnittes! Vieles wird nun neu für dich sein, denn an der Universität läuft einiges anders als in der Schule, während einer Ausbildung oder welcher Tätigkeit du auch immer vorher nachgegangen bist. Aber verzweifle nicht, denn wenn dir hier eines nicht passieren kann, dann, dass du auf dich allein gestellt sein wirst. Wir wünschen dir, dass du auch während deines Studiums jede Menge Freund:innen findest, die mit der gleichen Motivation studieren wie du und mit denen du gut auskommst. Und versuche, daran zu denken, dass das Studium aus mehr besteht als nur Lernen: es ist ein Lebensabschnitt, den du auch genießen solltest – und dafür wünschen wir dir nur das Beste.

    Wenn du diesen Artikel liest, dann stehen die Chancen gut, dass deine Ophase gerade begonnen hat oder du bereits mitten drin bist. Der Sinn dieser Veranstaltung ist es, dir die Universität und den Fachbereich Informatik näher zu bringen, dich mit deinen Kommiliton:innen bekannt zu machen und dir einen guten Start ins Studium zu ermöglichen. All das, was dir diese Woche geboten wird, ist von Studierenden organisiert und durchgeführt, um dir einen optimalen Start ins Studium zu ermöglichen. Die Tutor:innen, die dich diese Woche betreuen, sind Studierende, die, genau wie du jetzt, auch einmal eine Ophase mitgemacht haben und sich ein paar Semester später dazu entschlossen haben, das, was sie damals gelernt haben, an neue Studierende wie dich weiterzugeben. Die Webseiten hinter diesem QR-Code sind auch Teil der Ophase und beinhalten alle Informationen, die du im Laufe dieser Woche vermittelt bekommst – und noch viele mehr. Dennoch ist es empfehlenswert, die Veranstaltungen, die in der Ophase angeboten werden, zu besuchen. Dort hast du nämlich die Möglichkeit, Fragen zu allen wichtigen Themen rund um Universität und Studium zu stellen und natürlich auch die Möglichkeit, mit anderen Studierenden in Kontakt zu kommen. Für Letzteres gibt es auch noch zusätzliche Freizeitangebote in der Ophase, wie Kneipentouren oder ein Spieleabend, welche im kommentierten Stundenplan (verbirgt sich auch hinter dem QR-Code) genauer erklärt sind. Wir wünschen dir noch viel Spaß beim Entdecken unserer Online-Angebote und eine gelungene Ophase!
}
{
    \vspace{\fill}
    \begin{figure}[ht!]
        \centering
        \qrcode[hyperlink, height=.3\textwidth]{https://d120.de/ophase}
        \caption*{\large\url{https://d120.de/ophase}\\QR-Code zur Ophasenwebsite}
    \end{figure}
    \vspace{\fill}
}{}

\newpage
