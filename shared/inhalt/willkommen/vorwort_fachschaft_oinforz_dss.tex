\artikel{Welcome to TU Darmstadt}
{The Fachschaft Informatik welcomes you to your studies at TU Darmstadt.}{
    By now, you will probably have read and heard it several times already, but we, the Fachschaft Informatik\footnote{What -- or rather who -- we are and what we do, is described in the Article \textit{Questions and answers regarding the Fachschaft} on page~\pageref{FSarticle}}, would nonetheless like to heartily welcome you as well: to your master's degree studies, to Technische Universität Darmstadt, to Germany, and to a new chapter of your life in general!

    You may have already encountered a number of things in and about Germany that may feel strange, foreign or simply new to you.
    German culture is potentially very different from your own, and the university's as well as your study programme's structures are likely to be, too.
    Do not despair, though, for you are neither the first nor the only one struggling with these changes.
    Since many other students from abroad have been facing these same challenges in recent years already, the university and the computer science department have installed plenty of helpful facilities supposed to aid you with acclimatising to your master's degree programme.
    One of those facilities is the orientation phase, or Ophase, which you are currently (or possibly soon) attending and which aims to provide you with all the knowledge needed to start your studies successfully.
    In addition, there will be mentors (experienced students from the same programme as yours) to help you throughout your first semester who may also offer advise on other issues you may be facing in this country.
    Last but not least, there is us, the Fachschaft Informatik (or \textit{student council}), a bunch of active volunteers who are always eager to help out all computer science students with advice concerning affairs related to the department or the various study programmes here.

    It is our hope that during your studies you'll find new friends among your fellow students and that you may also find some time to expand your knowledge beyond this field of study alone.
    After all, your studies here mark the beginning of a new episode of your life in which you have the chance to get to know a new country and to experience a whole different culture.
    As such, we wish you all the best for your time here.
}
{Fachschaft Informatik}

%\vfill
\noindent
\bildmitunterschrift{willkommen/fsbild}{width=.9\textwidth}{Some members of the \textit{Fachschaft} during winter term 2019/2020}{Tim Pollandt und Stefanie Blümer}

%\newpage
