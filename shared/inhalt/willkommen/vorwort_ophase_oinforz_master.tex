\artikel{Willkommen zur Ophase}
{Wenn Du dieses Heft in der Hand hältst und diesen Artikel liest, dann stehen die Chancen gut, dass Deine Orientierungsphase gerade begonnen hat oder Du bereits mitten drin bist.}
{
    Die Ophase am Fachbereich Informatik ist eine Veranstaltung mit langer Geschichte: In etwa seit es den Fachbereich Informatik gibt, gibt es auch eine Orientierungsphase für Erstsemester. Der Sinn dieser Veranstaltung ist es, Dir die Universität, und insbesondere den Fachbereich Informatik, näher zu bringen, Dich mit Deinen Kommilitoninnen und Kommilitonen, also denjenigen Leuten, mit denen zusammen Du Dein Studium bestreiten wirst, bekannt zu machen und Dir einen guten Start ins Studium zu ermöglichen.

    All das, was Dir diese Woche geboten wird, ist von Studierenden organisiert und durchgeführt, um Studienanfängerinnen und \\ -anfängern wie Dir einen optimalen Start ins Studium zu ermöglichen. Die Organisator*innen, die Dich diese Woche betreuen, sind Studierende, die, genau wie Du jetzt, auch einmal eine Ophase mitgemacht haben und sich ein paar Semester später dazu entschlossen haben, das, was sie damals gelernt haben, an neue Studierende wie Dich weiterzugeben.

    Auch dieses Ophasen-Inforz, eine Sonderausgabe der Fachschaftszeitung Inforz, ist Teil der Ophase und beinhaltet alle Informationen, die Du im Laufe dieser Woche vermittelt bekommst – und noch viele mehr. Dennoch ist es natürlich empfehlenswert, die Veranstaltungen, die in der Ophase angeboten werden, insbesondere die Kleingruppen und den Vortrag zur Studienorganisation zu besuchen. Dort hast Du nämlich, im Gegensatz zum reinen Lesen dieses Heftes, die Möglichkeit, Fragen zu allen wichtigen Themen rund um Universität und Studium zu stellen und natürlich auch, mit anderen Studierenden in Kontakt zu kommen. Für Letzteres gibt es auch noch zusätzliche Freizeitangebote in der Ophase, wie Kneipentouren, welche im kommentierten Stundenplan (ein paar Seiten weiter) genauer erklärt sind.
    Wir wünschen Dir noch viel Spaß mit diesem Heft und eine gelungene Ophase!
}
{Die OInforz-Redaktion}
