% \artikel{Workshops}{Hier finden ihr mehr Informationen über die einzelnen Workshops}{
% --Anmerkung--
% Workshops werden hier definiert. Dafür wird der Command \workshop mit seinen 9 Argumenten verwendet:

% \workshop
% 1: Titel
% 2: Anbietende Personen
% 3: Zeitslots
% 4: Beschreibungstext
% 5: Maximale Teilnehmer*innenzahl
% 6: Benötigtes Material
% 7: Voraussetzungen zur Teilnahme
% 8: Sonstige Infos
% 9: Raum

% Damit die Anzahl der Workshops korrekt angezeigt wird, müssen alle Workshops vor der Verwendung von \workshopCount{} definiert werden

% Workshop Definitions
% Linux-Workshop (wird auf arch gehalten, btw)
\workshop{% Titel
    Linux-Workshop\faLinux{}\textcolor{white}{~(Arch, btw)}
}{% Anbietende Personen
    Jonas, Nath, Yanick, Ruben
}{% Zeitslot
    TBD
}{% Beschreibungstext
    Interaktiver Einsteiger-Workshop für Linux. Hast du keine Lust mehr auf Betriebssysteme, die sich selbst updaten und dabei ungefragt neue \enquote{Features} installieren? Du möchtest die Kontrolle darüber welche Daten von deinem OS erhoben werden? Dann besuche diesen Workshop!
}{}{}{}{}{% Raum
    TBD
}

% LaTeX-Workshop
\workshop{% Titel
    LaTeX-Einsteiger-Workshop
}{% Anbietende Personen
    Ruben
}{% Zeitslot
    TBD
}{% Beschreibungstext
    Einsteiger-Workshop für \LaTeX.
    Hast du schonmal ein größeres Projekt mit Word geschrieben, oder eine Präsentation mit Powerpoint erstellt, aber das schreiben von Mathematischen Formeln, das einheitliche Formatieren und das einhalten von Designrichtlinien ist kompliziert? Dann ist \LaTeX genau das richtige für dich!

    \LaTeX ist eine turing-vollständige Programmiersprache zum erstellen von PDF-Dateien.
}{}{% Benötigtes Material
    Internetfähiger Laptop oder Laptop mit installierter \TeX-Distribution
}{}{}{% Raum
    TBD
}
% Print Workshops
\artikel*{Workshops}{In den insgesamt \workshopCount{} Workshops werden einerseits Einblicke in interessante Themen des Universitätslebens und der Informatik geboten, andererseits auch aktive Programmpunkte zum Auspowern angeboten. Die Anzahl der Teilnehmenden für die Workshops ist begrenzt, der genaue Ablauf der Anmeldung wird im Anfangsplenum erläutert. Auf den folgenden Seiten werden die einzelnen Workshops mit Voraussetzungen vorgestellt.}{
    \mbox{}
    \printWorhshops{}
    \todo{add remaining workshops}
}{}
\clearpage
