\artikel{Vorwort des Dekanats}{Liebe Studierende,}{

    Herzlich willkommen am Fachbereich Informatik der TU Darmstadt!
    Sie haben eine sehr gute Wahl getroffen. Der Fachbereich Informatik ist einer der drei ältesten Informatik-Fachbereiche in Deutschland, gleichzeitig aber auch einer der modernsten. So haben wir als erster Fachbereich  auf Bachelor- und Master-Studiengänge umgestellt und bieten Ihnen heute eine sehr große Auswahl an spezialisierten Master-Studiengängen und Vertiefungsfächern an, die in dieser Form in Deutschland einzigartig ist. Doch Vielfalt ist nichts ohne Qualität. Diese wird durch die seit Jahren sehr hohe Einschätzung unter den besten Informatik-Studiengängen in Deutschland in den Rankings der Personalchefs belegt.


    %Die Leerzeichen sollen verhindern, dass LaTeX die Zeilenumbrüche zu lang zieht.
    \

    Doch das kommt nicht von ungefähr. Das Studium der Informatik an der TU Darmstadt ist anspruchsvoll, auch wenn Ihnen das in den ersten zwei bis drei Wochen des Grundstudiums möglicherweise noch nicht so vorkommen wird. Lassen Sie sich nicht täuschen. Anspruch und Tempo der Lehrveranstaltungen steigen danach schnell an. Nutzen Sie daher die ersten Wochen, um sich zu orientieren und gut in das neue Umfeld einzuleben. Die von der Fachschaft Informatik organisierte Orientierungsphase bietet dazu eine hervorragende Gelegenheit.
    Im Gegensatz zur Schule sind Sie an der Universität für Ihr Vorankommen selbst verantwortlich. Einer unserer ehemaligen Dekane hat den Unterschied von der Schule zur Universität mit den verschiedenen Arten von Wegen auf einen Berg verglichen. Die Schule ist ein Wanderweg auf eine Alm, der breit und gut beschildert ist. Auf dem Weg kommen Sie vielleicht manchmal in Atemnot und der Schweiß rinnt Ihnen in die Stirn, aber nachträglich können Sie sich vermutlich an wenige außerordentliche Schwierigkeiten mehr erinnern.

    %Die Leerzeichen sollen verhindern, dass LaTeX die Zeilenumbrüche zu lang zieht.
    \ \\ \

    \bildmitunterschrift{grafik/willkommen/haehnle_dekan}{width=35mm}{Prof. Dr. rer. nat. Reiner Hähnle}{}

    An der Universität geht es dagegen darum, von der Alm auf den felsigen Gipfel zu klettern. Sie bietet dazu ein Gewirr von Kletterpfaden auf diesen Gipfel an, aus denen Sie sich einen auswählen und ihn, begleitet von Bergführern (das sind die Dozenten, Tutoren und Mentoren), erklimmen. Die Bergführer stellen Ihnen die notwendige Ausrüstung zur Verfügung. Sie werden Sie jedoch niemals hochziehen, sondern Ihnen nur die nächsten Griffe zeigen. Klettern müssen Sie selbst!

    Das bedeutet, dass Sie sich jede Woche selbst motivieren müssen, zu den Vorlesungs- und Übungsstunden zu gehen, die Übungsaufgaben zu bearbeiten und sich auf die Klausurprüfungen vorzubereiten. Dabei ist der Lehrstoff sehr viel umfangreicher und schwieriger als in der Schule, so dass die Bearbeitung eines einzelnen Übungsblattes leicht einen Tag oder mehr beanspruchen kann. Und eine gute Prüfungsvorbereitung erfordert Wochen sorgfältiger Planung und konsequenter Durchführung.

    Wenn Sie sich nun bei dieser Klettertour sorgen sollten, dass der derzeitige Klettersteig oder die verwendete Klettertechnik nicht zum Gipfel führen oder Ihre eigenen Kräfte übersteigen sollten, dann ist es Zeit, die Route mit den Bergführern im Detail zu studieren. Vielleicht wäre eine andere Route besser für Sie geeignet, vielleicht war ein Fehler in der Wegbeschreibung, vielleicht gab es ein Missverständnis bei der letzten Besprechung, vielleicht sollten Sie ein Trainingslager aufsuchen. Es kann viele Gründe geben, frustriert zu sein. Da hilft dann nur die Analyse: Wo stehe ich im Studium, wo will ich hin und wie kann ich meine Fähigkeiten bestmöglich einsetzen und weiterentwickeln, um dorthin zu kommen? Dabei lassen wir Sie nicht allein, sondern stehen bereit, Sie mit verschiedenen Angeboten zu unterstützen. Scheuen Sie sich daher nicht, sich an Ihre Professoren, Mentoren, Tutoren und Studienberater zu wenden, damit wir gemeinsam mit Ihnen Lösungen für etwaige Probleme erarbeiten können.

    \bildmitunterschrift{grafik/willkommen/koch_studiendekan}{width=35mm}{Prof. Dr.-Ing. Andreas Koch}{}

    Vergessen Sie aber bei aller Anstrengung auch nicht, sich umzublicken und die Aussicht zu genießen. Ihr Studium soll für Sie interessant sein und Spaß machen. Gleichzeitig ist es der Beginn eines neuen Lebensabschnittes mit exzellenten Möglichkeiten, den eigenen Horizont enorm zu erweitern, neue Erfahrungen und neue Freundschaften fürs Leben zu gewinnen. Nützen Sie die vielfältigen Möglichkeiten, um einen Ihren Interessen angepassten Weg zu finden! Bewegen Sie sich nicht immer auf ausgetretenen Trampelpfaden. Schweifen Sie gelegentlich auch einmal bewusst ab vom Klettersteig und pflücken ein paar Blumen, aber behalten Sie dabei Ihr Ziel im Auge.

    Im Hinblick auf das Ziel, den Studienabschluss, sollten Sie auch nicht vergessen, dass reine Fachidioten nur im Studium scheinbar erfolgreich sind und der spätere Erfolg beim Berufseinstieg und im Beruf nur zu 50\% von rein fachlichen Eigenschaften und guten Noten im Abschlusszeugnis bestimmt wird. Die anderen 50\% des Erfolgs machen überfachliche Qualifikationen aus: Kann jemand gut argumentieren und überzeugend formulieren in Wort und Schrift? Ist man teamfähig? Kann man mit Kritik umgehen? Ist man befähigt ein Team zu führen? Wir empfehlen Ihnen daher, diese wichtigen, überfachlichen Qualifikationen im Studium nicht aus den Augen zu verlieren. Spätestens dann, wenn Sie die schwierigen Hürden der ersten drei bis vier Fachsemester im Bachelor-Studium geschafft haben, sollten Sie auch wieder mehr nach links und rechts blicken.

    Eine sehr gute, von vielen anderen guten Möglichkeiten ist das Engagement in der Fachschaft, die nicht nur diese wichtige Orientierungsphase für Sie organisiert hat, sondern die sich auch in vielfältiger Weise sehr konstruktiv für die Belange der Studierenden einsetzt und wertvolle und wichtige Beiträge zur Entwicklung des Fachbereichs Informatik leistet.

    Wir wünschen Ihnen einen guten Start in das Informatikstudium an der TU Darmstadt!
}
{Prof. Dr. rer. nat. Reiner Hähnle (Dekan) \\Prof. Dr.-Ing. Andreas Koch (Studiendekan)}

\newpage
