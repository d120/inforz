\artikel{Implementation Terms of the Study program}
{The implementation terms for your study program describe how your study program is organized.
}{
	As it is for all texts in this booklet, errors and omission are excepted. Legally binding are the General Examination Terms the Implementation Terms of the study program and the information given by Student Advisory Service or the Dean or Dean of Studies or the Examination Office.
	Please also remember that the English versions of the mentioned documents aren't legally binding either, only the German versions are.\\

	\noindent\textbf{Preliminary remark}
	To successfully finish your study program you have to achieve at least 120 Credit Points in accordance with the implementation terms. You also have to pass every imposed condition within your first year of studies. After your graduation you receive the academic title Master of Science (M. Sc.).\\

	\noindent\textbf{The goals of your studies}
	The Master in Distributed Software Systems is designed to enable you to develop high-quality business applications that are scalable, flexible, secure and liable. This study program will also enable you to autonomously develop software and work on a scientific basis.\\

	\noindent\textbf{Credit Points}
	Credit Points are a measure for the time complexity of a course. One Credit Point equals about 30 hours you invest into a course during the term. A student is expected to achieve about 30 Credit Points during one term. This means 40 hours of work per week.\\

	\noindent\textbf{Plan of studies}
	The plan of studies regulates the courses that you can take. It consists of four areas of mandatory choice and the Master Thesis. The first area is Distributed Systems. It focuses on specific knowledge needed to build distributed applications. You have achieve at least 18 Credit Points in this area. The second area is Networking and Systems Software, where you learn about the foundations of distributed applications. Here you need at least 18 Credit Points too. The third area is Formal Methods, Programming Languages and Software engineering. This will enable you to create software for high quality requirements. You have to get at leats 18 Credit Points in this area as well. In these three areas exams will be used for grading. The fourth area is Achievements in parallel to studies. These are mostly seminars or practical labs. You have to take at least one, at most two seminars. Also at least one of the forms practical lab, project and similiar forms has to be chosen. In this are you must achieve 12 to 15 Credit Points. The remaining 21 to 24 Credit Points you can achieve with courses from the first three areas. Please note that there is a list of available courses: \footnotemark[2].\\
	%
	%\noindent\textbf{Examination Plan}
	%You are required to create and hand in an examination plan. In this plan you have to name all the courses you will be taking during your studies. This plan has to be handed in to the coordinator of your study program before you can register for exams. For Distributed Software Systems this is Dr. Eichberg. After he approves your examination plan you also have to hand it in to the Examination Office. The examination plan is used to ensure that you choose courses in accordance with the implementation terms. Of course you can still change your examination plan if you want to choose a different course but then you will have to hand in the new plan. But remember that if you already took the exam of a course it has to stay in the examination plan. It doesn't matter whether you passed or failed it. There is a tool that helps you create your examination plan: \footnotemark[1]. Of course you also have to check that the courses you choose in the examination plan tool agree with the list of available courses for the DSS program: \footnotemark[2].

	\noindent\textbf{Master Thesis}
	The Master Thesis is the final part of your study program. This Thesis is worth 30 Credit Points which means 900 hours of work you will have to invest during one semester. Although you can start your Master Thesis anytime during your studies it is highly recommended to write it at the end of your studies when you have finished every other course and you already know all the research topics so that you can choose a good topic for your thesis. The Master Thesis is normally worked on alone. It is supposed to show that you can autonomously work an a problem of computer science. But of course you have an advisor for your task.\\

	\noindent\textbf{Coursework \ Study achievements}
	Courseworks can be repeated until they are passed. This often applies to practical labs or seminars. Courseworks are in Area of mandatory choice D of the plan of studies.\\

	\noindent\textbf{Exams}
	Exams can be repeated only a limited number of times. You have a maximum of three attempts per exam. If you fail two times at an exam you will be invited to an advisory appointment where you will get help in analysing why you failed two times and how to pass the final attempt. Exams can be written or oral. This normally depends on the number of students attending the course. For every exam you have to register in TUCaN during the exam registration phase.\\

	\bildmitunterschrift{artikel/pruefungsablauf_alternativ}{width=\linewidth}{}{}

	\noindent\textbf{Grades}
	The grades at German universities are the following with 1.0 being the best achievable grade and 5.0 meaning you failed the course:
	\begin{itemize}
		\item 1.0, 1.3 (excellent)
		    %	\item 1.3 (excellent)
		\item 1.7, 2.0, 2.3 (good)
		    %	\item 2.0 (good)
		    %	\item 2.3 (good)
		\item 2.7, 3.0, 3.3 (satisfactory)
		    %	\item 3.0 (satisfactory)
		    %	\item 3.3 (satisfactory)
		\item 3.7, 4.0 (sufficient)
		    %	\item 4.0 (sufficient)
		\item 5.0 (fail)
	\end{itemize}

	\noindent\textbf{TUCaN}
	The Campus Network portal TUCaN is where you register for every exam. In TUCaN you can also unregister from an exam. It is important to check the registration period every semester. You can only register for an exam during this time.
}{Johannes Alef}

\footnotetext[1]{\url{http://inferno.dekanat.informatik.tu-darmstadt.de/pp/}}
\footnotetext[2]{\url{https://www.informatik.tu-darmstadt.de/de/studierende/studiengaenge/masterstudiengaenge/spezialisierte-masterstudiengaenge/distributed-software-systems/lehrveranstaltungen/}}

%\newpage
