\artikel{Wie und wo lernen?}
{In vielerlei Hinsicht unterscheiden sich die Meinungen von Studierenden, vor allem aber wenn es ums Lernen geht. Jede*r Student*in lernt anders, denn jeder nimmt Dinge in seinem Leben anders wahr. Und das ist auch wichtig so! Meistens unterscheidet man aber diejenigen Studierenden, die zu Hause lernen, und die, die lieber in der Uni lernen.
}{
    \noindent\textbf{Arbeitsräume}

    Prinzipiell ist es egal, wo man lernt. Auch wenn das schwer zu glauben ist, eigentlich braucht man keine Computer, um das Informatikstudium zu überstehen. Wenn man zum Lernen in die Uni geht, kann man sich entweder in den großen Arbeitsräumen aufhalten oder man sucht sich einen Raum in einem anderen Gebäude und trifft sich dort mit seinen Kommiliton*innen zum Lernen.\\ \\
    \noindent\textbf{Im Piloty (S2$|$02)}\\
    Da hätten wir erstmal die PC-Pools, den C- und E-Pool. Dort stehen Computer und Monitore zur Verfügung. Die Pools sind allerdings oft relativ voll. Außerdem gibt es das Bistro Athene (C301), das aber wegen der guten Atmosphäre und dem Kiosk (aktuell geschlossen) auch oft voll ist. Zum Schluss haben wir noch den Gemeinschaftsraum - Verzeihung, das Lernzentrum Informatik (LZI) im Untergeschoss des A-Trakts, wo früher unsere Fachbereichsbibliothek war.\\ \\
    %\newpage
    \noindent\textbf{Mensa Stadtmitte}

    Die Räume der Mensa Stadtmitte (S1$|$11) - sind nicht nur während der Essenszeit geöffnet, sondern von 7 bis 19 Uhr. In der Otto-Berndt-Halle hat man dort außerhalb der Mittagszeit (von ca. 11 bis 15 Uhr) auf zwei Etagen viel Platz und meist auch Ruhe.

    Auch im Bistro gibt es reichlich Raum zum Lernen sowie Kaffee, belegte Brötchen usw., die eine längere Lernzeit sinnvoll unterstützen können. Hört sich perfekt an? Ist es leider aber nicht, denn meistens ist es relativ laut, wenn es voll ist.\\

    \bildmitunterschrift{artikel/lernen_utilities}{width=\linewidth}{}{Henry Klingberg / PIXELIO}

    \noindent\textbf{Universitäts- und Landesbibliothek}

    Für alle, die gerne mit Büchern arbeiten, ist die Bibliothek (S1$|$20) durch die direkte Nähe zu stapelweise Literatur und die langen Öffnungszeiten gut geeignet (momentan Januar bis März und Juni bis August täglich 24 Stunden, die restlichen Monate: 07:00 - 01:00 Uhr täglich). Allerdings gelten hier ebenfalls die Regeln einer Bibliothek, sprich: stilles Arbeiten. In der ULB kann man sich auch kostenlos "`Büros"' z. B. für seine Abschlussarbeit mieten, dies sollte man allerdings schon ein paar Monate vorher planen.\\

    \noindent\textbf{Andere Räume}

    Neben den "`offiziellen"' Arbeitsräumen hat man natürlich die Möglichkeit, sich einfach selbst einen Raum irgendwo auf dem Universitätsgelände zu suchen, der nicht bereits belegt oder gebucht ist. Ein guter Ansatzpunkt ist dafür das Alte Hauptgebäude (S1$|$03).\\

    \noindent\textbf{Zuhause lernen}

    Viele nehmen aber auch Vorlieb damit, zu Hause zu lernen. Man hat es ruhiger, ist in einem gewohnten Umfeld und kann sich manchmal besser konzentrieren. Allerdings ist die Taktik des bzw. der "`Einzelkämpfer*in"' beim Lernen nicht immer zu empfehlen, man sollte sich also auch dann mit anderen Leuten zu treffen versuchen, wenn man lieber zuhause lernt.\\

    \noindent\textbf{Arbeitsaufwand}

    Wichtig ist auf jeden Fall, den scheinbar "`leeren"' Stundenplan, den du wahrscheinlich zu Anfang hast, nicht zu unterschätzen. Es ist durchaus möglich, dass man für einige Fächer nicht viel Aufwand betreiben muss und das Besuchen der Vorlesung und der Übung genügen. Ebenso ist es aber möglich, dass man für jede Vorlesung eines Faches genauso viel oder mehr Zeit aufwenden muss, um den Stoff nachzuarbeiten. Dies hat vielleicht nicht mal mit einer*m selbst zu tun, sondern weil man Schwierigkeiten hat, Aufgabenstellungen zu verstehen oder man auf Rücksprache mit eine*r Ansprechpartner*in warten muss. Man sollte deshalb jedes einzelne Fach von Anfang an Ernst nehmen und selbst damit experimentieren, wieviel Zeit man aufwenden muss – denn Zeitumkehrer gibt es bei uns leider nicht und weniger machen kann man immer noch. Wer zu Beginn oder im Lauf des Semesters in einem Fach den Anschluss verliert, kann das oft nur noch schwer aufholen.
}
{Andreas Marc Klingler, überarbeitet von Patrick Toschka und Tim Pollandt}
%\newpage
