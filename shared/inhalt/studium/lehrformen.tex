
\artikel{Lehr- und Lernformen}
{Im Gegensatz zur Schule unterscheiden sich die Lehrformen an der Uni erheblich. In diesem Artikel stellen wir die an einer Universität üblichen Lehr- und Lernformen vor.
}{
    In großen Studiengängen herrscht über\-wiegend Massenbetrieb, so dass keine Kontrolle stattfindet. Die Verantwortung zum Lernen ist jedem selbst überlassen. Zum anderen sind die Anforderungen bezüglich der Lehrinhalte höher als in der Schule. Deshalb möchten wir dir die gebräuchlichsten Lehrformen an der Uni vorstellen. Es ist schließlich wichtig, sich über den eigenen Lernstil bewusst zu werden. Auch hierbei möchten wir ein paar gängige Methoden umreißen.

    Allgemein spricht man von einer Veranstaltung als Summe aller ihrer Lernformen. Eine Veranstaltung kann zum Beispiel nur aus einer Vorlesung bestehen, aus einer Vorlesung und einer Übung oder aus einer Vorlesung, einer Übung und einem Praktikum.
}{}

\noindent\textbf{Vorlesung}
\begin{multicols}{2}
    \bildmitunterschrift{artikel/lul_vorlesung}{width=\linewidth}{}{}
    \begin{itemize}
        \item Gebräuchlichste Form am Fachbereich Informatik.
        \item Ein*e Dozent*in präsentiert vorne im Hörsaal und die Studierenden hören zu.
        \item In den meisten Vorlesungen werden Folien-Präsentationen gezeigt.
    \end{itemize}
\end{multicols}


\noindent\textbf{Übung}
\begin{multicols}{2}
    \bildmitunterschrift{artikel/lul_uebung}{width=\linewidth}{}{Andreas Marc Klingler}

    \columnbreak

    \begin{itemize}
        \item Dient der praktischen Einübung und Vertiefung des Stoffes aus der Vorlesung.
        \item In kleineren Gruppen werden vorgegebene Aufgaben bearbeitet und durch eine*n Tutor*in betreut.
        \item Den Stoff aus der Vorlesung anwenden.
        \item Manchmal ist das Erreichen einer bestimmten Mindestpunktzahl sogar Voraussetzung für die Klausurzulassung.
    \end{itemize}
\end{multicols}

\newpage

\noindent\textbf{Praktikum}
\begin{multicols}{2}
    \bildmitunterschrift{artikel/lul_praktikum}{width=\linewidth}{}{}
    \begin{itemize}
        \item Dient zur Einübung "`praktischer"' Fertigkeiten.
        \item Erste Semester: meist Programmieraufgaben, alleine oder in Gruppen
        \item Oft wird die Abgabe danach von einem*r Tutor*in testiert.
        \item In einigen Veranstaltungen Voraussetzung zur Klausurzulassung.
    \end{itemize}
\end{multicols}


\noindent\textbf{Sprechstunde}
\begin{multicols}{2}
    \bildmitunterschrift{artikel/lul_sprechstunde}{width=\linewidth}{}{}
    \begin{itemize}
        \item Zu jeder Veranstaltung werden Sprechstunden angeboten.
        \item Keine Angst, "`dumme"' Fragen zu stellen!
        \item Vorbereitung trotzdem empfohlen.
        \item Oft schwache Auslastung, man kann meist ohne Anmeldung hingehen.
    \end{itemize}

\end{multicols}
