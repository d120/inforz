\artikel{Mehr als nur Unterricht?}
{Im Gegensatz zur Schule unterscheiden sich die Lehrformen an der Uni erheblich. Wir möchten dir hier kurz die gebräuchlichsten Lehrformen an der Uni vorstellen.}
{Allgemein spricht man von einer Veranstaltung als Summe aller ihrer Teile. Eine Veranstaltung kann zum Beispiel nur aus einer Vorlesung bestehen, aus einer Vorlesung und einer Übung oder aus einer Vorlesung, einer Übung und einem Praktikum.\\

    \textbf{Vorlesung}\\

    Vorlesungen sind die Veranstaltungsart, die dir im Studium am häufigsten begegnen wird. Klassischerweise finden Vorlesungen in Hörsälen statt, in denen der*die Dozent*in (üblicherweise ein*e Professor*in) die Inhalte der Veranstaltung mit Hilfe der zur Verfügung stehenden Medien (üblicherweise durch Präsentationen am Beamer, in Matheveranstaltungen aber oft auch nur an Tafel oder Overhead) vermittelt. Die Zuhörer*innen werden dabei nicht notwendigerweise eingebunden, wenngleich die meisten Dozierenden zumindest Zwischenfragen beantworten.
    \bildmitunterschrift{artikel/lul_vorlesung}{width=\linewidth}{}{}

    \textbf{Übung}\\

    In der Regel sind Übungen an Vorlesungen angebunden. Sie dienen dazu, den theoretischen Stoff, wie er in der Vorlesung vermittelt wird, durch praktische Aufgaben zu vertiefen. In vielen Fächern gibt es auch bewertete Hausübungen, für deren Bearbeitung nur begrenzt Zeit ist. Oft werden die Übungen von Studierenden betreut, die die Veranstaltung bereits in einem vorangegangenen Semester bestanden haben. Methodisch ähneln Übungen noch am ehesten dem klassischen Schulunterricht, Gruppenarbeit ist allerdings hier üblicher als in der Schule.
    \bildmitunterschrift{artikel/lul_uebung}{width=\linewidth}{}{Andreas Marc Klingler (3)}

    \newpage

    \textbf{Praktikum}\\

    Als Praktikum bezeichnet man bei uns ein Gruppenarbeitsprojekt mit starkem Praxisbezug, meistens handelt es sich bei diesen Veranstaltungen um Programmierprojekte. Praktika gibt es sowohl als eigenständige Veranstaltungen als auch in kleinerer Form integriert in den Übungsbetrieb von Vorlesungen. In jedem Fall sind allerdings die Aufgabenstellungen und zu verwendenden Methoden vorgegeben.
    \bildmitunterschrift{artikel/lul_praktikum}{width=\linewidth}{}{}

    \textbf{Seminar}\\

    Das Ziel von Seminaren ist primär die Vermittlung wissenschaftlicher Arbeitsmethoden im Sinne von Literaturarbeit, wobei man von einem*einer erfahreneren Wissenschaftler*in betreut wird. In der Regel arbeitet man sich innerhalb eines Seminares entlang einer wissenschaftlichen Fragestellung durch verschiedene Publikationen. Danach fasst man seine Erkenntnisse in einer schriftlichen Ausarbeitung zusammen und präsentiert seine Arbeit anschließend den restlichen Seminarteilnehmern.\\

    \textbf{Studienarbeit, Bachelorarbeit}\\

    Studienarbeiten sind Einzelarbeiten, in deren Rahmen eine wissenschaftliche Aufgabenstellung weitestgehend selbstständig bearbeitet werden soll (wenngleich man natürlich dennoch einen erfahrenen Betreuer hat). Auch bei Studienarbeiten sind Dokumentation, schriftliche Ausarbeitung und eine Präsentation gefordert. Im Bachelorstudium sind Studienarbeiten unüblich – von der Bachelorarbeit, die technisch gesehen einen etwas größeren Umfang hat, einmal abgesehen.
}{}
