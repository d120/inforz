\artikel{Die Ordnung des Studiengangs M.Sc. Informatik}
{Dieser Abschnitt beschäftigt sich speziell mit der Ordnung des Studiengangs M.Sc. Informatik.
}{
    Die folgenden Angaben sind wie immer ohne Gewähr. Verbindlich sind nur die offiziellen Versionen der Ordnung und der Allgemeinen Prüfungsbestimmungen. Verbindliche Informationen geben außerdem die (Fach-)Studienberatung, der*die Dekan*in, die*der Studiendekan*in und das Studienbüro.\\

    \noindent\textbf{Wahlbereiche}

    Der allgemeine Master umfasst zwei Wahlbereiche. Der erste Bereich umfasst vor allem Fachprüfungen. In diesem Bereich sind Veranstaltungen aus drei bis vier der Schwerpunkte einzubringen. Dabei müssen in jedem der Gebiete mindestens sechs CP erbracht werden. Insgesamt müssen in diesem Wahlbereich 45 bis 48 CP erbracht werden.

    Der zweite Wahlbereich umfasst Seminare, Praktika, Projektpraktika, Projekte, Studienarbeiten und Praktika in der Lehre. In diesem Bereich müssen 12 bis 15 CP erbracht werden. Es muss mindestens eins, maximal zwei Seminare besucht werden. Außerdem muss mindestens eine der Formen Praktika, Projektpraktika und ähnlicher Veranstaltungen besucht werden. Es kann höchstens ein Praktikum in der Lehre eingebracht werden.\\

    \columnbreak
    \noindent\textbf{Die Schwerpunkte}

    Die Lehrveranstaltungen der Informatik an der TU Darmstadt werden in die folgenden sechs Schwerpunkte unterteilt:
    \begin{itemize}
        \item \textbf{IT-Sicherheit}
        \item \textbf{Netze und verteilte Systeme}
        \item \textbf{Robotik, Computational und Computer Engineering}
        \item \textbf{Softwaresysteme und formale Grundlagen}
        \item \textbf{Visual \& Interactive Computing}
        \item \textbf{Web, Wissens- und Informationsverarbeitung}
    \end{itemize}

    \noindent\textbf{Nebenfach}
    Im allgemeinen Master muss ein Nebenfach gewählt werden. Dieses Nebenfach umfasst Veranstaltungen im Umfang von 24 CP. Es soll den Blick auf andere Fachkulturen richten und somit spätere interdisziplinäre Arbeit erleichtern und fördern. Für das Nebenfach muss ein Studienplan eingehalten werden. Dieser muss aber nicht genehmigt oder abgegeben werde. Das Nebenfach kann einmalig begründet gewechselt werden. Eine Liste der aktuell wählbaren Fächer gibt es im Ophasen-Infoheft der Nebenfächer und unter \footnotemark[2]. Es können auch weitere Nebenfächer vorgeschlagen werden, die Annahme ist aber nicht gewährleistet.\\
}
{Johannes Alef, überarbeitet von Stefan Pilot}

\footnotetext[2]{\url{https://www.informatik.tu-darmstadt.de/de/studierende/studiengaenge/masterstudiengaenge/master-informatik/nebenfaecher/}}
