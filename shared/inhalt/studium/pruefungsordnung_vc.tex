\artikel{Die Ordnung des Studiengangs Visual Computing}
{Dieser Abschnitt beschäftigt sich speziell mit der Ordnung des Studiengangs Visual Computing.
}{
    Die folgenden Angaben sind wie immer ohne Gewähr. Verbindlich sind nur die offiziellen Versionen der Ordnung und der Allgemeinen Prüfungsbestimmungen. Verbindliche Informationen geben außerdem die (Fach-)Studienberatung, der*die Dekan*in, die*der Studiendekan*in und das Studienbüro.\\

    \noindent\textbf{Visual Computing}

    Der Studiengang Visual Computing befasst sich speziell mit Themen der Computergraphik und Vision.

    Der Koordinator des Studiengangs Visual Computing ist Prof. Stefan Roth.\\

    \noindent\textbf{Pflichtbereich}

    Es müssen vier Pflichtveranstaltungen absolviert werden: \textit{Graphische Datenverarbeitung 1}, \textit{Graphische Datenverarbeitung 2}, \textit{Computer Vision 1} und \textit{Statistisches Maschinelles Lernen}.\\
    \columnbreak

    \begin{minipage}{\columnwidth}
        \noindent\textbf{Wahlbereiche}

        Der Studiengang Visual Computing umfasst fünf Wahlbereiche. Hier sind in jedem Bereich mindestens 6 und maximal 31 CP zu erbringen, in allen zusammen insgesamt 46 bis 49 CP. Diese Bereiche sind \textit{Computer Graphik} [sic], \textit{Computer Vision} und \textit{Maschinelles Lernen}, \textit{Integrierte Methoden von Vision und Graphik} und \textit{Anwendungen}. Der fünfte Wahlbereich, Studienbegleitende Leistungen, umfasst Seminar, Praktikum, Projektpraktikum, Projekt, Studienarbeit und Praktika in der Lehre. In diesem Bereich müssen 17 bis 20 CP erbracht werden. Es muss mindestens eins, maximal zwei Seminare besucht werden. Außerdem muss mindestens eine der Formen Praktika, Projektpraktika und ähnlicher Veranstaltungen besucht werden. Es kann höchstens ein Praktikum in der Lehre eingebracht werden.\\
    \end{minipage}
}
{Johannes Alef}
