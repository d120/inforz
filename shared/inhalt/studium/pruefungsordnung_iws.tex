\artikel{Die Ordnung des Studiengangs Internet- und Web-basierte Systeme}
{Dieser Abschnitt beschäftigt sich speziell mit der Ordnung des Studiengangs Internet- und Web-basierte Systeme.
}{
    Die folgenden Angaben sind wie immer ohne Gewähr. Verbindlich sind nur die offiziellen Versionen der Ordnung und der Allgemeinen Prüfungsbestimmungen. Verbindliche Informationen geben außerdem die (Fach-)Studienberatung, der*die Dekan*in, die*der Studiendekan*in und das Studienbüro.\\

    \noindent\textbf{Internet- und Web-basierte Systeme}

    Der Studiengang Internet- und Web-basierte Systeme befasst sich speziell mit Vernetzten und verteilten Systemen sowie Informations- und Wissensverarbeitung.

    Die Koordinatoren des Studiengangs Internet- und Web-basierte Systeme sind Prof. Max Mühlhäuser (Internet-basierte Systeme) und Prof. Iryna Gurevych (Web-basierte Systeme).\\

    \noindent\textbf{Pflichtbereich}

    Es gibt vier Pflichtbereiche aus denen jeweils eine von zwei Veranstaltungen bestanden werden muss. Im Bereich \textbf{Natural Language Processing} kann zwischen \textit{Natural Language Processing and the Web} und \textit{Natural Language Processing and eLearning} gewählt werden. In \textbf{Information Retrieval and Machine Learning} zwischen \textit{Web Mining} und \textit{Data Mining und Maschinelles Lernen}, in \textbf{Ubiquitous and Distributed Computing} zwischen \textit{TK3: Ubiquitous / Mobile Computing} und \textit{Verteilte Systeme und Algorithmen} und in \textbf{Communication Networks} zwischen \textit{Kommunikationsnetze I} und \textit{Mobile Netze}.\\

    \noindent\textbf{Wahlbereiche}

    Der Studiengang Internet- und Web-basierte Systeme umfasst drei Wahlbereiche. Die Bereiche \textit{Internet-basierte Systeme} und \textit{Web-basierte Systeme} umfassen vor allem Fachprüfungen. Hier sind in jedem Bereich mindestens 18 CP und maximal 36 CP zu erbringen, in diesen Bereichen zusammen insgesamt 51 bis 54 CP. Der dritte Wahlbereich, \textit{Studienbegleitende Leistungen}, umfasst Seminar, Praktikum, Projektpraktikum, Projekt, Studienarbeit und Praktika in der Lehre. In diesem Bereich müssen 12 bis 15 CP erbracht werden. Es muss mindestens eins, maximal zwei Seminare besucht werden. Außerdem muss mindestens eine der Formen Praktika, Projektpraktika und ähnlicher Veranstaltungen besucht werden. Es kann höchstens ein Praktikum in der Lehre eingebracht werden.\\
}
{Johannes Alef}
