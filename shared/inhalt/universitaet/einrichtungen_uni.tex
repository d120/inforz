\artikel{Einrichtungen an der Uni}
{Dass sich Schule und Uni in vieler Hinsicht unterscheiden, wirst du schon gemerkt haben. Ein wesentliches Merkmal der Universität ist, dass sie kein geschlossenes System ist, sondern vielmehr aus einer Ansammlung mehr oder minder stark gekoppelter Institutionen besteht und sogar externe Organisationen in vielen Bereichen präsent sind. Über diejenigen, die du kennen solltest, kannst du in diesem Artikel mehr erfahren. Kommst du von einer anderen Uni, werden dir viele Einrichtungen, die eine Uni hat, bekannt sein.
}{\textbf{Fach- und Studienbereiche}

    Auch wenn es für dich wahrscheinlich schon offensichtlich ist: die TU Darmstadt ist in so genannte Fachbereiche unterteilt, deren jeweilige Schwerpunkte auf mindestens einer wissenschaftlichen Disziplin liegen. Es gibt insgesamt 13 Stück, die mit Zahlen zwischen 1 und 20 nummeriert sind. So stehen zum Beispiel der Fachbereich 1 für Rechts- und Wirtschaftswissenschaften, FB 18 für die Elektrotechnik und FB 20 für die Informatik. Es gibt allerdings offensichtlich ein paar Lücken bei der Nummerierung, so gibt es beispielsweise keinen Fachbereich mit der Nummer 19. Was es allerdings noch gibt, sind diverse so genannte Studienbereiche wie Computational Engineering (CE) oder Informationssystemtechnik (IST), welche kleiner als Fachbereiche und in der Hauptsache durch einen sehr spezifischen Studiengang definiert sind. Dadurch sind die wissenschaftlichen Profile dieser Studienbereiche eingeschränkter als die der Fachbereiche.

    Nun sind aber Fach- und Studienbereiche nicht das untere Ende der Organisationsstruktur: Jeder Fachbereich ist wiederum in eine Menge Fachgebiete oder Arbeitsgruppen (die Benennung variiert zwischen den Fachbereichen) unterteilt, welche sich mit einem bestimmten Teilgebiet des jeweiligen Faches befassen und selbst wiederum in kleinere Gruppen unterteilt sein können, die dann jeweils aus einem bzw. einer Professor*in und seinen bzw. ihren Mitarbeiter*innen bestehen.
    \\\\
    An der Spitze eines jeden Fachbereiches steht ein*e so genannte*r Dekan*in, welche*r gleichzeitig auch Professor*in des Fachbereichs ist. Seine bzw. ihre Aufgabe ist die Geschäftsführung des Fachbereiches und die Leitung des Dekanats, die Verwaltung des Fachbereiches, sowie auch die Vertretung des Fachbereiches nach außen. Zudem gibt es auch noch eine*n Studiendekan*in, welche*r speziell die Lehre koordiniert und dadurch für die Studierenden oftmals von größerer Bedeutung ist als der*die Dekan*in.\\

    \textbf{Verwaltung und Präsidium}

    Nun besteht die Uni zwar aus vielen Fachbereichen, die alle durchaus auch ihre eigenen Regeln und Strukturen besitzen, nichtsdestotrotz sind sie zu einer größeren Struktur, nämlich der Universität, zusammengeschlossen. Ähnlich den Dekanaten der einzelnen Fachbereiche ist das Präsidium die Geschäftsführung und Leitung der Verwaltung der Universität, nur eben eine Stufe höher. Es ist in acht Dezernate aufgeteilt, die wiederum für verschiedene Aspekte der Verwaltung zuständig sind: so ist zum Beispiel das Dezernat II zuständig für Prüfungsangelegenheiten und koordiniert Klausurtermine und -räume aller Fachbereiche. Dem Präsidium steht der*die Präsident*in der Universität vor, welche*r auch die Uni nach außen repräsentiert (weiteres dazu im Artikel "`Politik an der Uni?"').

    Ebenfalls zur Verwaltung gehörig ist das Studierendensekretariat, welches die erste Anlaufstelle für Studierende ist, was Verwaltungsfragen angeht. Beispielsweise solltest du dich dorthin wenden, wenn sich deine Adresse geändert hat oder du irgendeine Bescheinigung von der Uni brauchst.\\

    \textbf{Serviceeinrichtungen}

    Das Studierendensekretariat ist nicht die einzige Serviceeinrichtung an der Uni, die Studierenden offen steht. Ebenfalls zur Uni gehörig ist das Hochschulrechenzentrum (HRZ), welches unter anderem für die Rechner- und Netzwerkinfrastruktur der TU zuständig ist. Zudem ist das HRZ auch für das e-Learning-Zentrum der Uni verantwortlich, in welchem Aufzeichnungen verschiedener Veranstaltungen öffentlich zugänglich gemacht werden. Auch verleiht das HRZ diverse Medientechnik wie Beamer, Leinwände und ähnliches.

    Nicht mehr der Uni allein zugehörig ist das Studierendenwerk Darmstadt, eine eigenständige, uniübergreifende Organisation, die unter anderem die Mensen der Darmstädter Unis betreibt, wie auch die Darmstädter Studierendenwohnheime. Auch das Darmstädter Amt für Ausbildungsförderung (also BAföG) ist ins Studierendenwerk integriert. Zudem bietet das Studierendenwerk auch Beratungs- und Hilfsangebote, zum Beispiel Rechtsberatung, Kinderbetreuung, psychologische Beratung und diverse weitere Services, insbesondere auch für internationale Studierende. Das Unisportzentrum, welches ein umfangreiches Sportangebot für Studierende bietet, ist ebenfalls nicht an die TU alleine gebunden.

    Genauso nicht rein zur Uni gehörig ist die Universitäts- und Landesbibliothek, die, wie der Name auch schon vermuten lässt, zusätzlich noch als Bibliothek und Archiv des Landes Hessen fungiert und damit auch mit vielen anderen öffentlichen Stellen in der Region verknüpft ist.\\

    \textbf{Forschung}

    Neben der Forschung und Lehre, die an der Universität betrieben werden, gibt es in Darmstadt noch eine große Anzahl nicht-universitärer Forschungsinstitutionen, die jedoch trotzdem in mancher Hinsicht mit der TU zu tun haben. Sei das lediglich, indem sie Praktika oder HiWi-Jobs speziell für Studierende der TU anbieten, Forschungskooperationen pflegen oder gar ins Lehrangebot an der Uni eingebunden werden.
    Für uns Informatiker*innen sind von diesen Instituten hauptsächlich das Fraunhofer-Institut für grafische Datenverarbeitung (IGD) und das Center for Advanced Security Research Darmstadt (CASED) von Relevanz, da Mitarbeiter*innen dieser Institute oftmals auch Pflichtveranstaltungen in der Informatik halten. Ebenfalls in Darmstadt ansässig ist auch noch das Fraunhofer-Institut für Sichere Informationstechnologie (SIT), welches allerdings weniger stark an der TU vertreten ist.

    Ebenfalls noch erwähnenswert (wenn auch für Studierende der Informatik von eher sekundärer Bedeutung) ist die Gesellschaft für Schwerionenforschung (GSI) mit Sitz im zum Darmstadt gehörigen Wixhausen. An diesem Forschungszentrum, das einen der größten Ringbeschleuniger der Welt betreibt, wurde unter anderem das nach der Stadt seines Sitzes benannte Element Darmstadtium nachgewiesen.}
{Stefan Gries, \\überarbeitet von Johannes Alef}
\newpage
