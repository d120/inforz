\artikel{Questions and Answers Regarding the Fachschaft (student council)}
{In the previous articles you have already got a rough understanding of the Fachschaft. In this article you will learn what the Fachschaft really is, what the Fachschaft can do for you and how you can participate.
}{
    \label{FSarticle}
    \textbf{What does "die Fachschaft"? mean?}

    The term Fachschaft originally means all students of a department, in this case the computer science department. Therefore, if you study at the computer science department, you are a part of the Fachschaft.\\

    \textbf{Why "originally"?}

    In everyday language use the students of computer science use the term "Fachschaft" for a few students who are actually part of the "active Fachschaft". A proper translation is departmental students representatives committee.\\

    \textbf{What is the difference between the members of the Fachschaft and the rest of the students.?}

    Primarily the members of the Fachschaft use their free time to help improve the studying situation for other students or at least prevent it from getting worse.\\

    \textbf{And what can the Fachschaft actually do?}

    If something happens at the department of computer science that has negative effects on your studies or if you have suggestions on how to improve the studying situation or your study program you can talk to the members of the Fachschaft about it. For example, if something is going wrong in a lecture, e.g. the organization is really bad, you (and maybe others who see the same problems) can talk to the Fachschaft and they can talk to the lecturer about how to improve the situation. This may of course take longer than talking to the lecturer yourself, but this way you can usually stay anonymous and since the Fachschaft has a good relationship to most lecturers, the matter will most likely be resolved faster. Additionally, the members of the Fachschaft can influence the politics of the department and bring the request to higher level of the hierarchy if necessary.\\

    \textbf{Do I have to bring everything that bothers me about a lecture to the Fachschaft?}

    Of course not. The Fachschaft mainly takes care of larger problems that affect many students or problems that need a lot of "lobbying" to be solved. If you only have a small issue it is normally much faster and less complicated to just talk to the lecturer about it yourself.\\

    \textbf{How can I contact the Fachschaft if I have a severe problem with the studying situation?}

    There are several ways: Most of the time you can find some members of the Fachschaft in the room of the Fachschaft S2$|$02 D120. Even if the person available can't directly help solve your issue they can usually tell you who you should talk to or inform the rest of the Fachschaft about it. If you don't want to go to D120 personally or simply aren't able to do so you can write an e-mail to the mailing list of the Fachschaft: wir@D120.de.\\

    \textbf{You are mentioning the term "studying situation" all the time. What does the Fachschaft take care of exactly?}

    The Fachschaft is active in the university politics, mainly (but not exclusively) on the departmental level: three representatives of the students are elected for the departmental council and can therefore influence decisions at the department. Furthermore some members of the Fachschaft are working in committees at the department, for example the QSL-committee that decides on the use of money given by the state Hessen for improving the studying situation, the committee for teaching and studying that discusses all subjects related to teaching. Naming and explaining all committees would take way too much space. If you want to see what committees members of the Fachschaft are active in you can take a look at \footnotemark[1].\\

    \textbf{That sounds like serious business – is politics the only thing the Fachschaft does?!}

    University politics is an important part of the work done by the Fachschaft, but not the only thing. Apart from the activities mentioned above there is a lot more that the Fachschaft organizes entirely or is at least partly responsible for. You are experiencing the most obvious one right now: the orientation phase is organized by the Fachschaft and many of the tutors are members of the Fachschaft. This booklet you are reading right now is a special issue of the journal of the students of computer science which is managed, organized and published during the semester by an editing staff that mostly consists of members of the Fachschaft. Currently, however, the journal is published only irregularly as there are too few people who are willing to write articles for the Inforz.\\
    Additionally, the Fachschaft organizes the evaluation of the lectures every term and analyses the result. But the Fachschaft is also responsible for leisure time activities: every year the Fachschaft organizes the summer party and a St Nicholas party and the Fachschaft also organizes the regular game evening Games no Machines (GnoM) and RPGnoM, which are explained in the leisure part of this booklet. There are even more activities organized by the Fachschaft. You can get an overview here: \footnotemark[2].\\

    \textbf{The word "organized" is in almost every sentence! Don't you need a lot of people for all these activities?}

    Yes! A lot of people are needed and there are not enough to keep all this running. The entire work is done voluntarily and only a few students are motivated to invest some time into these things, as studying alone already takes up a lot of time. But imagine what your studies would be like if there was no orientation phase or nobody would try to ensure that bad lectures have to be improved. Mainly thanks to the work of the Fachschaft over the last years and decades the studying situation is as good as it is. But there is still plenty of room for improvement. A problem often not seen by normal students is that the members of the Fachschaft are students, too, who at some point finish their studies and leave the university. Because of that the Fachschaft always needs new members.\\

    \textbf{With this much work to do it is no surprise that people are discouraged from helping.}

    That is a very one-sided view. Of course there is a lot to do but on the other side this work can be split up into many small pieces if more people are helping. And there are also many small or one-time activities to be done which require some helping hands and don't need those helpers to get a deeper understanding of any subject handled by the Fachschaft.\\

    \textbf{Really? And how do I get to know about these events so that I may help some time?}

    There are several ways to get information about these events. You can always ask members of the Fachschaft in room S2$|$02 D120 about current activities. If you want to get a deeper understanding of the work done by the Fachschaft, then the meeting of the Fachschaft, which takes place every Wednesday at 18:00, is a good place. In these meetings all currently important topics are discussed. Please note that due to the fact that currently there are only German members of the Fachschaft, the meetings are held in German. So it would be good to understand some German to be able to follow the conversation. If you don't speak German very well it is probably easier to ask a member of the Fachschaft to explain the discussed matters to you a few days after the meeting. If there is no meeting held in the room of the Fachschaft you can always spend some time there and get informed. Other possibilities to get informed are the web page of the Fachschaft \footnotemark[3], the blog of the Fachschaft (called "das Wesentliche") \footnotemark[4], the facebook page of the Fachschaft \footnotemark[5] or the Twitter-Account @d120de \footnotemark[6]. Of course most of these are in German for the above mentioned reason. If you want to participate from time to time in a certain area you can also register yourself on one of the mailing lists. An overview can be found here: \footnotemark[7].
}
{Stefan Gries\\
    edited by Julian Haas\\
    edited and translated by Johannes Alef}


\footnotetext[1]{\url{http://www.d120.de/gremien/}}
\footnotetext[2]{\url{https://www.d120.de/en/students/}}
\footnotetext[3]{\url{http://www.d120.de}}
\footnotetext[4]{\url{http://daswesentliche.d120.de/}}
\footnotetext[5]{\url{https://www.facebook.com/d120.de}}
\footnotetext[6]{\url{https://twitter.com/d120de}}
\footnotetext[7]{\url{http://www.d120.de/mailman/listinfo}}

%\newpage
