\artikel{Und noch mehr Programm: O fährt bald wieder}
{Nach den ersten Erfahrungen von der letzten Ophase sind wir uns sicher: wir wollen wieder eine Ofahrt anbieten!}
{
    Als Ofahrt bezeichnen wir einen Wochenendausflug zusammen mit den Erstsemestern im Rahmen der Ophase. Der allgemeine Ablauf orientiert sich dabei
    an der Ophase selbst und der Konferenz der Informatikfachschaften (KIF).
    Konkret werden wir bei der diesmaligen Ofahrt in den Odenwald (Südhessen) fahren, genauer gesagt nach Heppenheim an der Bergstraße auf die dortige Starkenburg. Der Aufenthalt in der Jugendherberge ist für Anfang Mai von Freitag den 08.05.2020 bis Sonntag den 10.05.2020 geplant.
    Dort sind bereits ausreichend Plätze für euch gebucht. Der Teilnehmerbetrag beläuft sich nach aktuellem Stand auf 32 \euro.\\

    Nachdem ihr alle (hoffentlich) vollständig und pünktlich angekommen seid, wird das eigentliche Programm beginnen. Erwähnenswert ist dabei, dass die Jugendherberge im Gültigkeitsbereich des Semestertickets liegt. Somit ist eine selbstorganisierte Anreise mit öffentlichen Verkehrsmitteln kein Hindernis finanzieller oder organisatorischer Natur.\\

    Nun zu den einzelnen Programmpunkten:
    Eine Mischung aus informativen, praktischen und lustigen Workshops mit Freizeit-Veranstaltungen sind für das Tagesprogramm am Freitag und Samstag geplant.
    Generell gilt immer "`Alles kann, nichts muss"': Niemand wird zu irgendetwas gezwungen.\\

    Letztes Jahr gab es 15 Workshops, zum Beispiel Graphische Modellierung mit Blender, Informationen über Fachschaftsarbeit, Linux-Install-Party, "`Ich habe den Programmiervorkurs verpasst"' und viele weiter. Wie viele und welche es dieses Mal geben wird, stand zum Redaktionsschluss des OInforz noch nicht fest, wird euch aber bei Zeiten gesagt werden. Ihr könnt euch in jedem Fall darauf gefasst machen, dass es nicht langweilig wird.\\

    Neben den Workshops, die in kleineren Gruppen stattfinden, werden wir auch noch eine gemeinsame Nachtwanderung mit Lagerfeuer und Stockbrot anbieten.\\

    \textbf{Und wofür das Ganze?}\\
    Zum einen möchten wir, dass ihr euch untereinander noch besser kennen lernt, da man zu Beginn des Studiums auf vollkommen unbekannte Menschen trifft. Zusätzlich zur Ophase geben wir euch damit die Möglichkeit, neue Kontakte zu knüpfen, was durch ein Jugendherbergswochenende in einer entspannten Atmosphäre begünstigt wird.\\

    Doch bei all der Produktivität sollte nicht vergessen werden, dass Spaß ein zentraler Teil der Ofahrt ist!\\

    \textbf{Haben wir euer Interesse geweckt?}\\
    Dann könnt ihr euch ab Freitag, 11.10., unter www.d120.de/ofahrt offiziell anmelden! Dort werdet ihr auch immer die aktuellsten Informationen bekommen.
}
{Steffen Klee, \\ überarbeitet von Fabian Damken}
\newpage
