\artikel{Selbstbestimmende Studierendenschaft}
{In der Universität haben die Studenten das Sagen? Das mag zwar etwas übertrieben klingen, es entspricht aber insofern der Wahrheit, als dass wir Studenten einiges mitzureden haben.
    Wo man hinschaut, arbeiten Studenten Hand in Hand im Unibetrieb, sei es bei der Organisation von Partys oder bei wichtigen hochschulpolitischen Entscheidungen.
}{
    Wo man hinschaut, arbeiten Studenten Hand in Hand im Unibetrieb, sei es bei der Organisation von Partys oder bei wichtigen hochschulpolitischen Entscheidungen.\\

    \textbf{Fachschaft}

    Die Fachschaft war wahrscheinlich eure erste Begegnung mit anderen Studierenden -  und natürlich die vielen Helfer die die Ophase. Ohne diese wären so große Veranstaltungen höchstwahrscheinlich gar nicht möglich. Die Fachschaft hat durch ihre Präsenz in verschiedenen Komissionen der Universität ein geltendes Mitstimmungsrecht und ist je-derzeit Ansprechpartner für die Studierenden des Fachbereichs. Sie handelt im Interesse der Studierenden und hat sich schon das ein oder andere Mal auch vor Professoren für sie ein-gesetzt. Und das alles unentgeltlich.\\

    \textbf{AStA}

    Eine weitere wichtige Gruppe ist der "`AStA"'. Dieser setzt sich für hochschulpolitische Ziele ein und unterstützt Studenten, wo er kann. Hier in Darmstadt kämpft der AStA zum Beispiel auch für bessere Wohnverhältnisse für Studenten.\\

    \textbf{Lehre}

    Im Lehrbetrieb übernehmen die Studenten ebenfalls sehr wichtige Rollen, hier merkt man meist erst den Unterschied zur Schule. An nahezu jeder Veranstaltung sind Studenten organisatorisch eingespannt. Ohne sie fiele deren Durchführung zumindest wesentlich schwerer.

    Viele Professoren haben Assistenten die ihnen bei der Vorbereitung des Stoffs helfen und für sie einspringen, wenn sie krank sind. Übungsgruppen (wie im Artikel übers Lernen beschrieben) werden von Tutoren geleitet. Das sind ebenfalls Studenten, die das jeweilige Fach schon abgeschlossen haben und anhand von Übungen den Stoff aus der Vorlesung nahebringen.

    Und natürlich sind auch bei vielen Uniprojekten Studenten involviert, die mit genauso viel Engagement Forschung betreiben.\\

    Wie es im Unibetrieb abläuft, werdet ihr in den nächsten Wochen noch kennen lernen, ihr könnt euch aber ab jetzt auch als Teil der selbstorganisierten Studierendenschaft betrachten und euren Teil dazu beitragen.
}
{Patrick Toschka}
\newpage
