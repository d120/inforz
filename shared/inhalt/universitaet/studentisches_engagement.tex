\artikel{Wie wichtig ist studentisches Engagement?}
{Wahrscheinlich bist du neu an der Uni, vielleicht kommst du direkt aus der Schule, hast vorher gearbeitet oder eine Ausbildung gemacht. In dem Fall wird in den nächsten Wochen vieles neu für dich sein. Eine der größten Überraschungen ist hierbei oft, wie viel Mitspracherecht und Gestaltungsmöglichkeiten Studierende im Vergleich zu Schülerinnen und Schülern haben. Die Gestaltungsmöglichkeiten der Studierenden sind groß und sollten genutzt werden.
}{
    Viele Studierende scheinen ihr ganzes Studium mit Scheuklappen vor den Augen zu verbringen, denn alles was sie sehen ist ihr eigener Studienerfolg und der möglichst schnell angestrebte Abschluss. Dabei lohnt es sich gerade im Studium, einmal über den eigenen Tellerrand hinaus zu schauen, denn als Student*in hast du eine Vielzahl an Möglichkeiten, um dich aktiv in den Unibetrieb einzumischen. Ein paar von ihnen will ich hier im Folgenden auflisten und etwas näher erläutern.\\

    \textbf{Fachschaft}

    Falls du an unserer Ophase teilgenommen hast, dann war die Fachschaft höchstwahrscheinlich dein erster Kontakt zu anderen Studierenden. Als Fachschaft organisieren wir eine Vielzahl von Veranstaltungen (wie unter anderem die Ophase),  beobachten aktuelle Geschehnisse am Fachbereich, haben immer ein offenes Ohr für Probleme von Studierenden und entsenden Mitglieder in verschiedene Gremien am Fachbereich. So haben wir zum Beispiel ein Mitspracherecht, wenn es um die Berufung von neuen Professor*innen geht, oder wenn die Ordnung eines Informatik-Studiengangs geändert werden soll.

    Leider sind viele dieser Aktivitäten sehr arbeitsintensiv, sodass wir immer dankbar über alle sind, die ihren Teil dazu beitragen wollen, sei er auch noch so klein. Wenn du dich also für die Geschehnisse am Fachbereich interessierst, ist die Fachschaft eine tolle Anlaufstelle!

    Mehr über die Fachschaft erfährst du im Artikel "`Fragen und Antworten rund ums Thema Fachschaft"' und auf der Fachschaftswebseite \footnotemark[1].\\

    \textbf{Hochschulpolitik}

    Wenn du dich für Politik interessierst und gerne die Universität aktiv mitgestalten möchtest, dann ist vielleicht auch die Hochschulpolitik etwas für dich. Du könntest zum Beispiel an den Treffen einer der vielen politischen Listen an der TU teilnehmen und wer weiß, vielleicht steht ja schon bei der nächsten Hochschulwahl auch dein Name auf den Stimmzetteln?

    Falls du lieber unabhängig bleibst, aber trotzdem politische Projekte mitgestalten willst, dann könntest du dich stattdessen auch für einen Referatsposten beim AStA bewerben. Die Referate werden von jährlich auf der AStA-Webseite \footnotemark[2] ausgeschrieben, du kannst aber auch einfach direkt mit deiner Projektidee auf den AStA zugehen.\\

    \textbf{Hochschulgruppen}

    Weniger politisch, aber in keinem Fall weniger engagiert, sind auch die Hochschulgruppen an der TU. Du könntest dich zum Beispiel mit der HG Nachhaltigkeit für einen nachhaltigen Campus einsetzen oder bei "`Studieren ohne Grenzen"' dazu beitragen, dass junge Menschen auf der ganzen Welt Zugang zu Bildung erhalten.
    Doch das ist noch lange nicht alles: Ob Hackertreff, Sportclub, Chor, Orchester, Vernetzung mit internationalen Studierenden oder Filmkreis, hier ist wohl für jede*n etwas dabei.
    Eine Übersicht aller Gruppen findest du unter \footnotemark[3].\\

    \textbf{Lehre}

    Letztendlich wird auch die Lehre zu einem gewissen Teil von Studierenden mitgestaltet. So leiten zum Beispiel erfahrene Studierende, die das jeweilige Fach schon abgeschlossen haben, als Tutor*innen deine Übungsgruppen. Darüber hinaus unterstützen sie als Assistent*innen die Professor*innen bei der Konzeption und Durchführung ihrer Lehrveranstaltungen und sind natürlich auch in aktuellen Forschungsprojekten involviert.
    Auch das Mentorensystem, welches du in deinem ersten Semester absolvieren musst und über das du an anderer Stelle noch mehr erfahren wirst, wäre ohne die vielen studentischen Mentor*innen nicht möglich.\\

    Wie du siehst, sind Studierende ein essentieller Bestandteil des Unibetriebs und sie haben durchaus auch die Macht, die Uni zu verändern. Vielleicht bist du ja nun selbst motiviert, dich neben dem Studium ein wenig zu engagieren und sorgst in Zukunft dafür, dass die Uni ein kleines bisschen besser wird.}
{Julian Haas}

\footnotetext[1]{\url{http://www.d120.de}}
\footnotetext[2]{\url{https://www.asta.tu-darmstadt.de}}
\footnotetext[3]{\url{https://www.tu-darmstadt.de/studieren/campusleben/engagement_student/hochschulgruppen.de.jsp}}
