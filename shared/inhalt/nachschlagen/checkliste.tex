\addsec{Checkliste für die erste(n) Woche(n) an der Uni}

\subsection*{Ophasenveranstaltungen}

\checkbox{Erstie-Tüte erhalten}{Du erhältst diese Tasche bei der Ophasen-Begrüßungsveranstaltung oder falls du daran nicht teilnehmen konntest, im Fachschaftsraum D120.}

\checkbox{Vorträge zur Studienorganisation besucht}{Diese Vorträge finden am Dienstag statt. Insbesondere der Vortrag zum Mentorensystem ist sehr wichtig!}

\checkbox{Stundenplan zusammenstellen}{Wie das geht, wird dir in der Kleingruppe am Mittwoch erklärt.}

\subsection*{Accounts}

\checkbox{HRZ-Account aktiviert}{Folge den Anweisungen unter \footnotemark[1], Abschnitt "`HRZ-Account aktivieren und TU-ID erhalten"'. Das Passwort dazu steht auf dem Brief, mit dem du deinen Studienausweis bekommen hast.}

\checkbox{ISP/RBG-Account aktiviert}{Melde dich unter \footnotemark[2] mit deinem HRZ-Nutzernamen und Passwort an, und folge den Anweisungen, um deinen ISP/RBG-Account zu aktivieren.}

\checkbox{Im Informatik-Moodle angemeldet}{Im Informatik-Moodle kannst du dich mit deinen HRZ-Daten einloggen: Klicke unter \footnotemark[3] auf den Button "`Log In"' und dann "`Nutzer\_innen mit TU-ID"'.}


\newpage

\subsection*{Anmeldungen}

\checkbox{In Moodle und TUCaN zum Mentorensystem angemeldet}{Die Anmeldung findet im Informatik-Moodle statt, möglicherweise wird sie erst nach der Ophasenwoche freigeschaltet.}

\checkbox{Im Moodle für die Übungen der Veranstaltungen des 1. Semesters angemeldet}{Neben der Anmeldung der Übung im TUCaN muss man sich bei den meisten Veranstaltungen noch zusätzlich für die Übungen im Moodle anmelden. Hier erhält man meist erst die endgültigen Termine der Übungsgruppen.}

\checkbox{In TUCaN für die Module des 1. Semestern angemeldet}{Eine Einführung in die Benutzung von TUCaN erhältst du in der Kleingruppe am Donnerstag.}

\checkbox{In TUCaN für die Veranstaltungen des 1. Semesters angemeldet}{Du musst dich zunächst für die entsprechenden Module anmelden, bevor du dich für Vorlesungen und Übungen anmelden kannst. Erst wenn du für die Veranstaltung angemeldet bist, kannst du dich während des Prüfungsanmeldezeitraums für die Fachprüfungen anmelden.}

\subsection*{Sonstiges}

\checkbox{Bild für Athene-Karte hochgeladen}{Möchtest du eine Athene-Karte haben, musst du unter [4] ein Passbild hochladen, das auf deine Athene-Karte aufgedruckt wird.}


\footnotetext[1]{\url{https://d120.de/a}}
\footnotetext[2]{\url{https://support.rbg.informatik.tu-darmstadt.de}}
\footnotetext[3]{\url{https://moodle.informatik.tu-darmstadt.de}}
\footnotetext[4]{\url{https://www.idm.tu-darmstadt.de/ando}}

\newpage
