\artikel{Häufig gestellte Fragen}
{}
{\textbf{Bis wann muss ich mich bewerben?}

    Die Bewerbungsfrist für ein Informatikstudium im kommenden Wintersemester endet am 15. Juli. Ein Masterstudium kann auch zum Sommersemester begonnen werden. Die Frist ist hier jeweils der 15. Januar.

    Es ist allerdings sehr zu empfehlen, die Unterlagen möglichst früh einzureichen. Falls Du Unterlagen vergessen hast, können diese dann noch rechtzeitig nachgereicht werden.\\

    \textbf{Gibt es Zulassungsbeschränkungen oder einen Numerus Clausus?}

    Für die Informatik-Studiengänge an der TU Darmstadt (B.Sc., M.Sc.) gibt es keine Zulassungsbeschränkungen und auch keinen Numerus Clausus. Wenn Du die formalen Voraussetzungen erfüllst und die geforderten Unterlagen rechtzeitig einreichst, wirst Du auf jeden Fall immatrikuliert.\\

    \textbf{Muss ich vor dem Studium schon Programmieren können?}

    Das Programmieren lernst Du von Grund auf im 1. Semester und vertiefst es im laufe des Informatikstudiums in verschiedenen Programmiersprachen. Im Informatikstudium werden sehr viele grundlegende Techniken zum Programmieren gelehrt. Es ist daher einfacher, wenn man nicht durch "`Programmierhalbwissen"' vorbelastet ist. Um den unterschiedlichen Wissensstand auszugleichen wird das Programmieren im 1. Semester mittels Programmiersprachen gelehrt, die die wenigsten vorher schon benutzt haben werden.\\

    \textbf{Ich hatte kein Informatik in der Schule}

    Aber Du hattest doch sicherlich Mathematik, oder? Das bringt Dich am Anfang des Studiums sehr viel weiter als das bisschen Informatik, das meist in der Schule beigebracht wird. Die Informatik wird ab dem 1. Semester von Grund auf gelehrt und die Inhalte sind deutlich weitergehender als in der Schule.\\

    \textbf{Muss ich gut Mathe können?}

    Mathematik sollte in der Schule nicht unbedingt Dein Hassfach gewesen sein, in dem Du im Grundkurs immer schlechte Noten hattest. Interesse an Mathematik ist auf jeden Fall von Vorteil. Die ersten Semester Informatik an der Uni bestehen zu einem erheblichen Teil aus Mathematik. In Vertiefungsveranstaltungen wird später auf diese Kenntnisse aufgebaut.\\

    \textbf{Brauche ich ein eigenes Notebook?}

    Nein. Ein eigenes Notebook oder ein Tabletcomputer ist keine Vorausetzung für ein erfolgreiches Studium. An der Uni stehen alleine im Fachbereich Informatik zwei große PC-Poolräume mit etwa hundert Computern zur Verfügung. Ein Pool ist sogar 24 Stunden täglich geöffnet. Informatik ist nicht mit Computern gleichzusetzen. Gerade in den ersten Semestern mit großem Theorieanteil (Mathematik, Theoretische Informatik) wirst Du sehr viel auf Papier arbeiten. Ein eigenes Notebook kann trotzdem sehr praktisch sein. Wenn Du Dir ein Notebook für das Studium anschaffst, solltest Du bei der Auswahl des Gerätes auch Größe und Gewicht betrachten, damit es nicht zu unhandlich ist.\\

    \textbf{Welchen Abschluss bekomme ich?}

    Das kommt ganz darauf an, in welchen Studiengang Du Dich einschreibst. Bei Bachelor Studiengängen erhältst Du nach Absolvieren des Studiengangs einen Bachelor-Abschluss, z.B. B.Sc. = Bachelor of Science. Die Regelstudienzeit beträgt sechs Semester.
    In einem Masterstudiengang, der auf einen Bachelor aufbaut, erhältst Du nach Absolvieren des Studiengangs einen Master-Abschluss, z.B. M.Sc. = Master of Science. Die Regelstudienzeit beträgt vier Semester.
    Diese Abschlüsse kannst Du dann hinter Deinem Namen führen, das sähe dann so aus:\\ "`Hans Mustermann, M.Sc."'


}{}
\newpage
