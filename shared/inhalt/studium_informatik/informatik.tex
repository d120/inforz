\artikel{Def.: Informatik, die}
{Als Informatiker*in hat man das zweifelhafte Glück, selten gefragt zu werden, was denn die Inhalte dieses Studienganges sind. Der Begriff der Informatik hat sich mittlerweile eingebürgert, aber oftmals treffen die Vorstellungen der Leute nicht so ganz auf die Inhalte des Informatikstudiums zu und es besteht noch viel Klärungsbedarf.
}
{Sprachlich betrachtet ist der Begriff der Informatik eine Wortschöpfung aus den 1960er-Jahren, zusammengesetzt aus den Worten Information und Automatik. Allein die Zusammensetzung dieses Wortes sagt aber noch viel mehr über diese wissenschaftliche Disziplin aus: Während viele Leute dazu tendieren, Informatik direkt mit Computern gleichzusetzen, geht es hier hauptsächlich um Automatisierung, nicht unbedingt das Mittel, mit dem diese Automatisierung üblicherweise umgesetzt wird. Die Informatik ist nämlich an sich eine sehr theoretische Wissenschaft, die deutlich mehr mit Mathematik als mit Elektrotechnik zu tun hat – ein Sachverhalt, den Du auch im Laufe Deines Studiums feststellen wirst.

    Prinzipiell geht es in der Informatik darum, Probleme zu lösen. Was gelehrt wird ist hauptsächlich Methodik, um gegebene Probleme zu analysieren, (abstrakt) zu modellieren und davon ausgehend möglichst allgemeine Lösungsmethoden zu entwickeln – und gegebenenfalls zu zeigen, dass diese Lösungsmethode in jedem Fall so funktioniert, wie es von ihr erwartet wird. Die für Analyse und Modellierung notwendigen Fertigkeiten sind hauptsächlich mathematisch-logischer Natur, erst im Schritt der Lösungsentwicklung greift man dann auf "`konkrete"' Konstrukte wie Programmier- oder Hardwarebeschreibungssprachen zurück.

    Natürlich wirst Du auch lernen, wie ein Elektronenrechner (vulgo: Computer) "`unter der Haube"' aufgebaut ist und funktioniert, ebenso wird Dir auch beigebracht werden, was Betriebssysteme eigentlich tun. Das bedeutet aber im Umkehrschluss nicht, wie oftmals angenommen, dass Du als Informatiker*in Rechner zusammenbauen oder Windows debuggen kannst. Zumindest sind derartige Beschäftigungen keine Studieninhalte – natürlich gibt es auch (sogar nicht gerade wenige) Informatik-Studierende, die das können, die haben sich das aber üblicherweise selbst beigebracht oder woanders gelernt.

    Wir wollen Dich hier natürlich nicht von Deiner Studienfachwahl abbringen – allerdings sollte man sich hin und wieder bewusst machen, wo die eigentlichen Ziele des Informatikstudiums liegen. Und das tun sie an der Uni in der Regel im theoretischen Bereich, in dem es darum geht, die grundlegenden Mechaniken und Ideen zu verstehen und weiterentwickeln zu können. Die oftmals assoziierten praktischen Arbeiten sind dagegen eher Domäne von Fachinformatiker*innen.
}
{}

\vfill
\bildmitunterschrift{comics/random_number}{width=9cm}{}{xkcd.org}

\newpage
