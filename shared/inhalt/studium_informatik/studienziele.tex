\artikel{Studienziele}{Auszug aus der Ordnung des Studiengangs Informatik:
}
{Im Studiengang Bachelor of Science (B.Sc.) „Informatik“ an der Technischen Universität Darmstadt erwerben die Studierenden sowohl fachliche als auch fachübergreifende Kompetenzen. Diese Kompetenzen sind charakteristisch für den Anspruch des Studiengangs und auch wesentliche Voraussetzung für die Fortsetzung des Studiums in einem darauf aufbauenden Master-Studiengang.

    Nach Abschluss eines Bachelor-Studienganges sind die Absolventinnen und Absolventen in der Lage,
    \begin{itemize}
        \item{ihr Fachwissen zu den mathematischen, technischen, theoretischen und anwendungsorientierten Grundlagen der Informatik einzusetzen,}
        \item{weitgehend selbständig Aufgabenstellungen zu Inhalten der Pflichtveranstaltungen des Studienganges zu bearbeiten,}
        \item{weitgehend selbständig, anspruchsvolle Probleme und Aufgabenstellungen aus der Praxis mit wissenschaftlichen Methoden zu analysieren und zu lösen,}
        \item{die erforderlichen Methoden und Arbeitstechniken zu identifizieren und korrekt umzusetzen,}
        \item{verschiedene Medien zur Informationsbeschaffung zu nutzen und deren Zuverlässigkeit sicher einzuschätzen,}
        \item{die Ergebnisse ihrer Analysen bzw. die ausgearbeiteten Lösungen sicher an Fachleute und Laien zu kommunizieren,}
        \item{ein begrenztes Thema aus dem Bereich der Informatik mit wissenschaftlichen Methoden in begrenzter Zeit selbständig zu bearbeiten,}
        \item{flexibel in Projektteams zu arbeiten und solche Teams effizient zu organisieren und Führungskompetenz zu erwerben,}
        \item{die gesellschaftliche Verantwortung ihrer Tätigkeit einzuschätzen und angemessen zu berücksichtigen,}
        \item{die Arbeit auf verschiedenen Zeitskalen selbständig zu organisieren,}
        \item{weiterführende Lernprozesse selbständig zu gestalten und lebenslang zu lernen}
    \end{itemize}
}{}

\newpage
