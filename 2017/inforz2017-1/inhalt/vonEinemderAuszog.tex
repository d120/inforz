\artikel{Von einem der auszog, ein Arch System aufzusetzen}
{"'Arch Linux ist eine flexible und leichtgewichtige Distribution für jeden erdenklichen Einsatz-Zweck. Ein einfaches Grundsystem kann nach den Bedürfnissen des jeweiligen Nutzers nahezu beliebig erweitert werden."' \footnote{entnommen von https://www.archlinux.de/ am 05.04.2017}
}{
Arch Linux verspricht für Informatiker*innen von Welt Herausforderung und höchste Freude zugleich. Es ist Prüfstein und Auszeichnung aller Vollblut-Nerd*inen, Quell größter Verwirrung und fürchterlichster Verzweiflung zugleich. Aber wer es geschafft hat, alle Schritte auf dem Weg zu einem Arch Linux System zu gehen und ein laufendes System besitzt, kann mit vollem Recht auf all die Schmalspur-Informatiker*innen herabblicken, die noch ein Ubuntu, oder gar ein Windowssystem auf ihrem Rechner haben. \\


Na gut, bei dieser Einleitung mag der Poet mit mir durchgegangen sein. Aber die Installation eines eigenen Arch Linux Systems ist für viele Informatiker*innen tatsächlich ein lohnendes Ziel. Nicht umsonst war die Distribution Teil der Linux Vorstellungen im vergangenen Inforz. Von dieser Vorstellung habe auch ich mich inspirieren lassen, mein altes Ubuntu zu löschen und ein neues System from Scratch zu installieren. Ein Informatiker, 5. Semester mit eingeschränkten Linux Vorkentnissen, gegen den Rechner. Zwei Tage habe ich installiert, Config-Files geschrieben und mir ein Haufen Haare ausgerissen. Es folgt ein Erlebnisbericht. \\

Der erste Schritt zu einem funktionierenden Betriebssystem ist wie immer das Brennen eines Installationsmediums. Im Zeitalter von USB-Sticks und Internet ist das, Gott sei Dank, kein großes Problem mehr. Unter https://wiki.archlinux.de/title/Download erhält man eine aktuelle .iso-Datei. Diese kann zum Besipiel mit dd oder UNetbootin auf ein USB Stick (oder eine CD, für Traditionalist*innen) geschrieben werden. 
}
{Claas Voelcker}