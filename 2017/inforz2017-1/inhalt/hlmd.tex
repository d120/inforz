\artikel{Ausflugstipps in Darmstadt: Das HLMD}
{Das Hessische Landesmuseum Darmstadt (HLMD) am Karolinenplatz wurde ab 2007 für sieben Jahre (zwei Jahre über Regelstudienzeit) und viel Geld (mehr als 300.000 Semesterbeiträge) komplett saniert. Nach der Wiedereröffnung im Herbst 2014 ist das Universalmuseum auf jeden Fall einen Besuch wert.
}{Und schon das erste Fremdwort. Ein kleiner Exkurs für Interessierte: Museum. Altgriechisch für Musentempel. Da nun nicht alle Expertenwissen in griechischer Geschichte haben, sei erwähnt, dass sie die Schutzgöttinen der Künste, Kultur und Wissenschaften waren. Ein Museum ist also ein Ort, an dem bedeutsame oder lehrreiche Gegenstände aus Kunst, Kultur und Wissenschaft aufbewahrt, erforscht und ausgestellt werden.

Also ein Museum, dass sich nicht auf einen Bereich wie Kunst oder Naturkunde spezialisiert, sondern wesentlich breiter aufgestellt ist. Dieses Konzept ist mittlerweile recht selten, was das HLMD in dieser Hinsicht besonders macht.

Aber was kann man dort denn nun alles sehen? Fangen wir von vorne an. Im Anfang schuf..., na gut so weit zurück dann doch nicht. Nur die letzten 50 Mio. Jahre. Fossilien aus allen möglichen Erdzeitaltern und Regionen können betrachtet werden. Zu den bekanntesten Fundstücken zählen sicherlich die Urpferdchen, die in der Grube Messel gefunden wurden. Und wenn man dann ein paar Millionen Jahre ins Land gehen lässt, landet man schon beim Mastodon, das freundlich aus dem 2. Stock ins Foyer blickt.

Mastodonten sind übrigens schon etwas länger ausgestorben. Und ja, auch ein Skelett kann freundlich gucken.

Kommen wir nun zur Zoologischen Sammlung, die viele interessante, faszinierende oder leider ausgestorbene Tiere und Vögel zeigt, sowie diverse Lebensräume als Dioramen mehr oder weniger realistisch darstellt. Es sei nochmal betont, dass sich die gezeigten riesigen Laufvögel weder als Haustier noch zum Reiten eignen. Außerdem sind sie -- richtig -- ausgestorben.

Genug Text über den Bereich Wissenschaft, der aber noch einiges mehr zu bieten hat. Nun folgt Kunst und Kultur. Auch hier gibt's eine breite Spanne von Epochen, Regionen und Künstler*innen zu sehen. Beginnend in der Steinzeit und im alten Ägypten geht es weiter nach Griechenland (schon wieder die Griechen) und in das römische Imperium. Uschebti, Vasen, ein Sarkophag, noch mehr Töpfe und Vasen, ein Mosaik, ein paar Säulen und einiges mehr. Zwischendrin dann noch ein paar Samurai. Die wollen schließlich auch gesehen werden.

Das finstere Mittelalter ist im HLMD gar nicht mehr so finster, sorgt doch der Catwalk mit den Ritterrüstungen für Glanz und Glamour, unterstützt durch handwerkliche Meisterwerke aus Glas, Holz und Metall.

Freunde der Malerei können sich auf Werke aus über 700 Jahren freuen, die im Anbau des Gebäudes ausgestellt werden. Und in diesem Fall dürften sich die meisten einig sein, dass das Kunst ist. Wer es moderner mag, muss den Anbau verlassen und unter das Dach des Hauptgebäudes steigen. Dort findet sich unter anderem die Müllhalde des Museums, oh Verzeihung, der Block Beuys. Kein weiterer Kommentar dazu. Moderne Kunst ist halt Geschmackssache. Ist doch völlig normal, dass man erstmal die Aufsicht fragen muss, ob hier noch aufgebaut wird oder nicht.

Ach ja, Jugendstil gibts auch in hinreichend großer Menge zu sehen. Wir sind ja schließlich in Darmstadt, einer der Hauptstädte des Jugendstils.

Zusammenfassend lässt sich sagen, dass das HLMD definitiv einen Besuch wert ist. Mit 4 Euro ist der Eintritt für Studierende recht günstig und Eintrittskarten eigenen sich auch recht gut als Weihnachtsgeschenk für Eltern o.ä. Die Erfahrung hat gezeigt, dass Viele hunderte Kilometer in andere Städte und Länder reisen, um Dinge zu sehen und zu erleben, aber irgendwie nie auf die Idee kommen, die eigene Stadt und die nähere Umgebung zu erkunden. Das ist recht schade, denn auch die (temporäre) Heimat bietet viele Dinge, die man nicht erwarten würde. Das HLMD liegt quasi immer auf dem Weg zur Uni, doch drin waren sicherlich die wenigsten. Daher ein Appell: Erkundet Darmstadt und seine Umgebung, denn auch hier kann man tolle / faszinierende / interessante Dinge entdecken.}{Tobias Otterbein}

\bildmitunterschrift{grafik/HLMD.jpg}{width=\textwidth}{Hessisches Landesmuseum Darmstadt}{Jan Bambach}
