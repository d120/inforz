\artikelgruppe{Linuxdistributionen Teil 2}
{Nach dem wir letztes Mal bereits angefangen haben einige unser Lieblingsdistributionen vorzustellen, wollen wir euch hier nochmal zwei Alternativen präsentieren. Elementary OS, das Mac unter den Linuxdistros, und Mint, das Windows unter ihnen. Wer eine objektive Meinung erwartet, der erwartet zu viel. Hier eine rein subjektive Meinung einiger Individuen. }
{
	\subartikelohnevorspann{elementaryOS}
	{Elementary OS ist ein auf Ubuntu basierendes Betriebssystem, dass sich vor allem an Personen wendet, die mehr auf "`Eyecandy"' setzen, als auf Funktionalität. Dabei orientiert sich das System sehr stark an OSX/macOS. Das Betriebssystem ist leichtgewichtig und kommt mit seinem eigenen Desktop Environment (Pantheon), welches die Ziele der Entwickler perfekt umsetzt. Die Grundidee der Entwickler hinter ihrem Design ist nämlich wie folgt definiert: "`Sei prägnant, vermeide Konfiguration und gebe minimale Dokumentation"', sprich es soll intuitiv sein, soll "`Out of the Box"' funktionieren und gut vorbereitet sein und es soll verständlich sein, sodass man als Benutzer keine große Dokumentation braucht. \\

		Das versuchen die Entwickler auch möglichst gut umzusetzten, so setzen sie
		auf eine einfache Bedienoberfläche und einige Standardprogramme. \linebreak Leider sind nicht alle davon fehlerfrei oder zu 100 Prozent ausgearbeitet und durchdacht. So gibt es zum einen viele Bugs, die mit der Zeit die Nutzung stören oder erschweren, aber auch einige Designentscheidungen die ich nicht nachvollziehen kann. Hierbei fallen mir unter anderem die vordefinierte Dockingbar ohne File Browser ein, so wie das Blockieren des Systems bei der Installationen von Software, die nicht in ihrem "`AppCenter"' angeboten werden (ist etwas umständlich möglich). Auch wenn die Entwickler ein "`Out of the box"'-Feeling geben wollen, so sollten sie mich als Nutzer nicht einschränken, was die Konfiguration und Anpassung des Systems an mich betrifft. Dies passiert allerdings viel zu häufig und stört mich dann doch wieder. \\

		Von seiner Schwierigkeit ist es sehr leicht zu bedienen und eignet sich somit auch für Anfänger, wenn ich dort auch Mint vorziehen würde, gerade wenn ich mir die Fehleranfälligkeit von Elementary anschaue. \\

		Was das Betriebssystem möchte ist sicherlich sinnvoll, viele Nutzer wünschen sich ja schließlich ein "`schönes Linux"', welches wenig aufwand macht, leicht zu bedienen ist und nicht viel Hintergrundwissen erfordert. All das verspricht einem Elementary OS und mit etwas mehr Zeit wird es das eventuell auch schaffen, doch in seinem jetzigen Zustand ist es noch keine Alternative. \\
		Denn so schön die Oberfläche auch ist, so vermisse ich viele Funktionen, die mir andere Distributionen wie Mint, Arch oder Fedora zur Verfügung stellen. Auch ist es mir noch zu instabil und hat zu viele Fehler. Ich sehe aber Potenzial in der Idee. }
	{Jannis Blüml}
	\subartikelohnevorspann{Linux Mint}
	{Es gehört zu den beliebtesten Linux der Zeit und ist auf Platz eins der Distrowatch Liste. Durch Projekte wie den Cinnamon Desktop und
		die Nutzung der Ubuntu Paketquellen hat sich Mint schnell als benutzerfreundlicher Ersatz für Canonicals Betriebssystem erwiesen
		und viele Menschen zum Umsteigen gebracht. In diesem kurzen Test möchte ich zeigen, warum das so ist und für wen sich diese Distro lohnt.\\
		Mint ist für Linuxverhältnisse recht alt. Bereits 2006 wurde die erste Version veröffentlicht. Schon damals wurde auf die Ubuntu Paketquellen
		gesetzt, was relativ aktuelle Software und etablierte Tools wie die Paketverwaltung apt und dem bekannten Installer
		dem/der Nutzer*in zur Verfügung stellt. Auch gibt sich das Projekt viel Mühe, eine vollständige Kompatibilität mit Ubuntu aufrecht zu erhalten.
		Dadurch sind viele Anleitungen und Programme auf beide Distros
		äquivalent anwendbar.\\
		Einen großen Unterschied gibt es jedoch: Die angebotenen Desktopumgebungen. Hier steht bei Mint nämlich nicht Unity sondern standardmäßig
		Cinnamon, eine Eigenentwicklung, und Mate zur Verfügung. Zudem werden XFCE und KDE angeboten, aber meist mit größerer Verzögerung.
		Cinnamon ist ein aufgeräumter Desktop, welcher sich an Windows orientiert. Als Fork von GNOME wurde versucht, ein klassisches Bedienkonzept
		zu fahren, welches dem Nutzer möglichst nicht im Weg steht, aber trotzdem einiges an Konfigurationsmöglichkeiten bietet. Beim Testen erwies
		sich der Desktop als sehr schnell und elegant, hat aber standardmäßig die nervige Angewohnheit, bestimmte Aktionen des Nutzers mit Tönen zu
		untermalen. Die lässt sich jedoch in den aufgeräumten Einstellungen ausschalten. Auch viele weitere Möglichkeiten der Konfiguration gibt es
		hier, Cinnamon lässt sich ganz den Wünschen seiner Nutzer*innen anpassen.\\
		Die mitgelieferte Software entspricht den Standardprogrammen der meisten Distros. Firefox als Browser und Thunderbird als Mailprogramm
		kommen standardmäßig zum Einsatz. Zudem wird LibreOffice mitgeliefert,
		sowie eine Auswahl an Hilfsprogrammen wie Transmission und Gimp.\\
		Insgesamt handelt es sich größtenteils um FOSS\footnote{Free Open Source Software}, jedoch bietet der Installer an,
		unfreie Plugins und Codecs wie z.B. Flash nachzuladen und zu
		installieren.\columnbreak\\
		Bezüglich Updates hat sich das Mintteam leider für eine recht harsche Politik entschieden. Sowohl das Betriebssystem als auch die
		installierten Programme werden nicht geupdatet, wenn keine hunderprozentige Kompatibilität gewährleistet ist. Dadurch wurden in der
		Vergangenheit bereits Sicherheitsupdates erst mit großer Verspätung ausgeliefert. Andererseits macht das Mint gerade für Anwender*innen
		attraktiv, welche hohe Stabilität brauchen. Der daraus teilweise
		resultierende Sicherheitsverlust lässt sich dann eventuell verschmerzen.
		\medskip\\
		Linux Mint ist eine sehr schöne und stabile Distribution. Der robuste Kern und der gut konfigurierbare Desktop machen diese Linuxversion
		attraktiv für eine große Breite an Anwender*innen, sowohl Anfänger*innen als auch Expert*innen. Das Betriebssystem wirkt fertig und durchdacht,
		die Grundfunktionen sind schnell erreichbar und laufen sehr flüssig, auch auf älteren Rechnern. Die teilweise etwas veraltete Software
		lässt sich verschmerzen, die Updatepolitik ist hingegen eher fragwürdig. Insgesamt ist Mint jedoch absolut empfehlenswert und definitiv für mich
		eine Installation wert.
	}
	{Heiko Carrasco}}

\vfill
\bildmitunterschrift{grafik/comics/screenshots.png}{width=\textwidth/4*3}{}{xkcd.com/1863}


\newpage
