% !TeX spellcheck = de_DE


\artikel{Interview mit Professor Koch}
{Als Professor Koch vor zwölf Jahren an die Uni kam und das Fachgebiet ESA gründete, interviewten wir ihn zum ersten Mal. Dieses Interview könnt ihr in der Ausgabe April 2005 finden. Nun wollten wir wissen, wie sich der Fachbereich und das Fachgebiet ESA entwickelt haben.
\bildmitunterschrift{grafik/koch}{width=\linewidth}{Prof. Koch im Interview April 2017}{}
}
{
\subsubsection*{Warum sind Sie Professor geworden?}
Aus zwei Gründen, einmal bin ich an der Wissenschaft und speziell an der Ingenieurstätigkeit sehr interessiert, andererseits lehre ich auch sehr gerne. Ich habe damit recht früh angefangen und habe schon in meiner Schulzeit meinen Mitschülern und Lehrern C beigebracht. Aus diesem Grund erschien mir die Tätigkeit als Professor passender als zum Beispiel Wissenschaftler in einem Forschungsinstitut zu werden. 

\subsubsection{War die Informatik bzw. die Elektrotechnik schon immer Ihr Ziel?}
Ich habe mich schon sehr früh für Technik und Elektronik interessiert und habe damals Kurse für Elektronik an unserer Volkshochschule belegt. Irgendwann kam dann aber der Punkt, an dem mir mathematische und physikalische Grundlagen gefehlt haben. Auch hatte ich großen Respekt vor diesen Transistoren, denn wenn man damit rumspielte tendierten sie dazu, kaputt, zu gehen und das war nicht gut für mein Taschengeld. Deswegen war die Elektronik für mich nur so was wie eine Einstiegsdroge. Als Anfang der 80er die Computer aufkamen, war ich sofort begeistert, denn erstens konnte man durch Programmieren nichts kaputt machen und zweitens war man nicht durch den Zwang des Aufbaus neuer Hardware behindert. Dieser neuartige Prozess, von der Idee zur fertigen Umsetzung, ohne beispielsweise einen Lötkolben in die Hand nehmen zu müssen, hat mich so gefesselt, dass ich mich schlussendlich für die Informatik entschied.

\subsubsection{Sie sind der einzige Professor dieses Fachbereichs, der sich so speziell mit Hardware beschäftigt. Fehlen uns hier Kompetenzen und sollten wir in Zukunft in diesen Bereich mehr investieren?}
Das ist eine der wenigen Dinge, die ich in den letzten Jahren bedauere. Der Fachbereich hat einige sehr attraktive Forschungsgebiete unterstützt, dabei ist leider die technische Informatik stark geschrumpft. Als ich hier angefangen habe, gab es noch drei Professuren in diesem Bereich. Zwar tun wir unser Möglichstes, ein gewisses Lehrangebot aufrecht zu erhalten, aber es gibt auch einige Vorlesungen, die ich aus Zeitgründen schon seit Jahren nicht mehr halten konnte. Ich könnte mir vorstellen, gerade in Zusammenhang mit sehr aktuellen Themen, diesen Bereich wieder aktiver zu gestalten, zum Beispiel im Bereich "`\textit{High Performance Computing}"'. Hier erreichen wir so langsam die Grenzen der bisherigen Ansätze. Es gibt aber frische Forschungsideen, die sich mit neuartigen (Rechner-)Architekturen beschäftigen, um einige aktuelle Probleme anders zu lösen. Hier wäre eine gute Zusammenarbeit zwischen den Anwendern und den "`Hardwareaffinen"' notwendig, um eine möglichst gute Optimierung zu finden. Allerdings sehe ich auch das Problem der begrenzten Professorenstellen, man kann nicht alle Bereiche gleich gut ausbauen. Dass es in den letzten Jahren die technische Informatik aber so stark getroffen hat, finde ich persönlich sehr schade. 

\subsubsection{Sie haben vor knapp zwölf Jahren das Fachgebiet ESA gegründet. Hat es sich seitdem so entwickelt, wie Sie es sich erhofft haben?} 
Inzwischen ja, obwohl wir immense Schwierigkeiten in der Aufbauphase hatten. Ich habe damals ja nur mit einem Mitarbeiter angefangen. Und da wir doch recht ingenieursnah arbeiten, war die Arbeit recht aufwendig und wir litten doch unter unserer unterkritischen Masse. Mit der Zeit erkannte der Fachbereich allerdings die überhöhte Arbeitsbelastung auch in der Lehre und hat für Unterstützung gesorgt. Mit der aktuellen Größe von 8-10 Leuten bin ich recht zufrieden, auch wenn wir zwischenzeitlich mal etwas größer waren. Da hatte ich aber dann das Gefühl, dass ich meine Doktoranden nicht mehr intensiv genug betreuen konnte und ich zu weit weg von ihrer Arbeit war. Ich bin nicht Professor geworden, um nur Manager zu sein, sondern ich möchte mich mit meinem Fachgebiet auch unmittelbar befassen. Mit der aktuellen Größe bin ich sehr zufrieden, da ich dadurch nah genug an meinen Mitarbeitern bin, um auch fachlich mitzuhalten und mitzulernen.

\subsubsection{Im aktuellen Aufbau der Vorlesung \textit{Architektur und Entwurf von Rechnersystemen} (AER) lernt man "`Bluespec Systemverilog (BSV)"' oder kurz Bluespec kennen. Was ist denn aktuell Ihre Lieblings Hardware Description Language?}
Das ist eine sehr gute Frage, allerdings muss man hier mit "`Für welchen Zweck?"' noch präzisieren. Wenn es darum geht, hoch produktiv und höchst optimiert zu arbeiten, dann ist Bluespec die beste HDL, die wir zurzeit zur Verfügung haben. Aber auch Bluespec ist nicht ganz ohne: Durch seine Abstraktion ist es nicht immer ganz leicht zu verwenden und man benötigt einiges an zusätzlichem Wissen. Deswegen wäre es am besten, wenn man Hardware für spezifische Anwendungsgebiete in der Sprache beschreiben könnte, die in dem jeweiligen Anwendungsgebiet Verwendung findet. Hätten wir beispielsweise eine Perl-artige Sprache, die speziell in der Netzsicherheit verwendet wird, so wäre ein Compiler von z.B. dem relevanten Perl-Subset zu BSV sinnvoll, wenn sich die Sicherheitsexperten nicht mit Bluespec oder Verilog auskennen. Hier entwickelt man sogenannte Domain Specific Languages (DSL): Diese orientieren sich dann an den natürlichen Beschreibungen ihrer Anwendungsdomäne und  schaffen es, daraus Hardware zu erzeugen. Dies ist natürlich der eleganteste Weg, Hardware zu erstellen. Solche DSLs werden daher in Zukunft noch eine große Rolle spielen.

\subsubsection{Sie halten aktuell regelmäßig die Veranstaltungen AER, EiCB, RO und DT. Diese vier Veranstaltungen liegen allerdings alle im Pflichtbereich. Was wären denn Veranstaltungen, die Sie gerne mal wieder halten würden, die Sie zeitlich aber aktuell nicht schaffen?}
Ich würde gerne eine Veranstaltung zum Thema Moderne Rechnerarchitekturen anbieten, die in unserem Curriculum derzeit fehlt. In RO schauen wir uns MIPS an, in AER bekommen die Studierenden einen kleinen Einblick in ARM. In den letzten Jahren ist das Gebiet aber stark gewachsen. Wenn ich Zeit hätte würde ich gerne eine Vorlesung vorbereiten, in der wir uns genau diese neuen Trends einmal anschauen. Dabei sollten vom kleinen embedded Prozessor über Multi-Core CPUs und Grafikkarten alle Architekturen und Geräte wenigstens übersichtsartig behandelt werden. Es ist wichtig, die Funktion dieser Recheneinheiten auch auf Hardware-Ebene zu kennen, um sie besser programmieren zu können. Leider fehlt mir hierfür gerade einfach die Zeit.

\subsubsection{In der Hoffnung, wir behalten Sie weitere zwölf Jahre, was wären Ihre Pläne für diese?}
Das hängt ganz davon ab, wie sehr ich in Zukunft in Pflichtaufgaben eingebunden werde. Wenn ich weiter, wie in den letzten Jahren, so viele Pflichtveranstaltungen betreue, dann bremst das doch schon sehr aus. Sowohl in Bezug auf weiterführende Lehrveranstaltungen, als auch in der Forschung. Man kann leider nicht alles gleichzeitig machen. Ein konkretes Ziel ist, in Zukunft enger zusammen mit den Anwendungsexperten aus speziellen Domänen an DSLs zu arbeiten und zu forschen. Aus diesen DSLs kann man dann anwendungsspezifische Hardware generieren, die für ihre Domäne so optimiert ist, dass sie herkömmliche Prozessoren in den Schatten stellen kann. Eine Domäne, in der das aktuell schon sehr gut funktioniert, ist das Maschinelle Lernen. Ich denke aber, da geht noch mehr!

\subsubsection{Wenn Sie sich die letzten zwölf Jahre anschauen, was waren denn Ihre Highlights in Ihrer Zeit als Studiendekan?}
Zweifellos die Reakkreditierung, speziell das Auseinandersetzen mit unserem Curriculum und seine Weiterentwicklung, damit es heutigen Anforderungen genügt. Dabei musste ich aber auch im Auge behalten, dass es Studierenden und Dozenten gleichermaßen liegt.

\subsubsection{Was ist denn Ihr Highlight aus den kompletten 12 Jahren?}
Singuläre Ereignisse wie die Reakkreditierung gibt es eigentlich wenige, höchstens noch den Erhalt des Athenepreises für gute Lehre. Das hat mich sehr gefreut und ich werte das ganz hoch. Wir freuen uns aber natürlich auch, wenn wir unsere Forschungsergebnisse in guten Tagungen oder Journals unterbringen.

\subsubsection{Wenn wir uns mal kurz etwas von der Uni entfernen, was sind denn Ihre Hobbys außerhalb Ihres Berufs?}
Leider erschreckend wenig. Ich bin ein spielender Mensch, das kann man schon sagen. Dabei interessiert mich weniger das Glücksspiel, wohl aber Brett- und Computerspiele. Gerade letztere haben mich auch zur Informatik selber gebracht. Ich war doch fasziniert, wie so ein Atari VCS funktionierte. Ich probiere gerne vieles aus und schaue, wie es mir gefällt. Ich gehe gerne Wandern und lese viel, wenn ich Zeit habe. Hier dann idealerweise keine Fachbücher, sondern eher fantastische Literatur.

\subsubsection{Was ist denn Ihre Lieblingsklausurfrage?}
"Was sind die Stufen des Pipeline MIPS Prozessors, die wir kennen gelernt haben?". Die stelle ich zum Aufwärmen immer wieder gerne.

\subsubsection{Und was ist Ihre Lieblingsantwort?}
Ich habe am liebsten Antworten, bei denen der Studierende nachgedacht und nicht nur aus der Vorlesung reproduziert hat. Das kommt leider viel zu selten vor. 
Gerade wenn ein Studierender die Antwort nicht kennt, sollte er versuchen, die Antwort selbst zu entwickeln und die richtigen Schlüsse zu ziehen, anstatt die Frage einfach auszulassen.

\subsubsection{Im letzten Interview sagten Sie, Ihre Lieblingsprogrammiersprache sei Smalltalk. Sehen Sie dass immer noch so?}
Das ist immer noch richtig, aber die Sprache ist leider tot. Nicht, weil sie schlecht ist, sondern weil die Anbieter von Entwicklungswerkzeugen und ähnlichem mit ihrer Lizenzpolitik die Sprache geschlachtet haben. Die Eleganz, die Smalltalk hatte, habe ich bisher noch in keiner anderen Sprache gefunden. Aktuell oben auf meiner Liste von noch zu lernenden Sprachen stehen aber Scala und Haskell.

\subsubsection{Was verwenden Sie, Windows, Linux oder Mac?}
Alle drei. Es kommt darauf immer an, was man machen will. Ich bin aber persönlich ein Freund von Unix. Wenn ich die Wahl habe, verwende ich Linux, aber wir wissen alle, dass das nicht immer ganz so gut klappt. 

\subsubsection{Was ist Ihr liebster Texteditor / TeX-Editor?}
Als Editor für Texte verwende ich Kile,  wobei ich aktuell auf Texstudio wechsle. Das liegt daran, dass Kile Texte so umformatiert, dass die Funktionalität von Git o.ä. etwas behindert wird.

\subsubsection{Was ist für Sie die 42?}
Die Antwort auf die all bewegende Frage, die Deep Thought gefunden hat und für mich, neben der 23, die Standardkonstante für alle meine Vorlesungen.

\subsubsection{Warum die 23?}
Ich weiß es ehrlich gesagt nicht, das habe ich mir von anderen Kollegen angewöhnt. Allerdings kenne ich die Geschichte dahinter nicht.\footnote{Die Zahl geht wohl auf die {\em Illuminatus}-Romanreihe von Shea und Wilson zurück.}

\subsubsection{Letztes mal sagten Sie, Sie haben keine Zeit zum Kochen. Hat sich das inzwischen geändert?}
Ich bemühe mich und werde auch immer besser.

\subsubsection{Fehlt Ihnen irgendwas an den aktuellen Inforzen?}
Es gab früher immer eine Seite mit Glossen meiner Kollegen und von mir, die fand ich immer sehr unterhaltsam.

\subsubsection{Was wollen Sie den Studenten noch mitteilen?}
Sie sollen keine Scheu haben zu denken. Rufen sie nicht nur die gecachten Antworten aus dem Gedächtnis ab, sondern verstehen sie diese auch. 	

}
{Heiko Carrasco \& Jannis Blüml}

	\vfill
\begin{minipage}{0.5\textwidth} 
	\bildmitunterschrift{grafik/comics/quantum.png}{width=\textwidth}{}{xkcd.com/1861}
\end{minipage}
\begin{minipage}{0.5\textwidth} 
	\bildmitunterschrift{grafik/comics/physics_confession.png}{width=\textwidth}{}{xkcd.com/1867}
\end{minipage}

\newpage
