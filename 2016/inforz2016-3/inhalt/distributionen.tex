\artikelgruppe{Linuxdistributionen Teil 1}
{Wer mit Linux anfangen möchte, kennt das Problem: Es gibt nicht DAS Linux,
    sondern eine unüberschaubare Anzahl an Distributionen. Das sind
    Zusammenstellungen von Software, welche um den Linux Kernel herum ein
    Betriebssystem aufbauen. Um bei der Auswahl ein wenig zu helfen möchten wir
    euch in einer kleinen Reihe einige Distributionen vorstellen. Los geht's mit Fedora und Arch Linux. }
{
    \subartikelohnevorspann{Fedora}
    {Fedora ist eine Distribution, welche von Red Hat zusammen mit einer Community
        entwickelt wird. Sie ist RPM-basiert, das heißt, dass Software in sogenannten
        RPM-Paketen ausgeliefert wird. Die Zielgruppe dieser Linuxdistribution sind
        etwas erfahrenere User. Fedora zeichnet sich durch einen relativ schnellen
        Entwicklungszyklus aus; eine neue Version wird zweimal im Jahr veröffentlicht.
        Dies hat den Vorteil, dass die Software relativ neu ist, trotzdem aber gut
        getestet wurde. Die Paketquellen halten relativ viele Programme bereit, die, im
        Gegensatz zu vielen anderen Distributionen, mit sehr guten
        Standardeinstellungen auf dem Rechner landen. In der Grundkonfiguration wird
        eine leicht angepasste Version der GNOME Desktopumgebung installiert, jedoch
        finden sich auch für KDE, LXQT, XFCE und sogar i3 Liebhaber*innen Pakete in den
        offiziellen Paketquellen. Sollte der Fall auftreten, dass ein gesuchtes
        Programm nicht dort zu finden ist, gibt es das sogenannte copr, was, ähnlich zu
        launchpad unter Ubuntu, anderen Usern die Möglichkeit gibt, Programme dort
        einzustellen.
        \\
        Insgesamt ist Fedora für etwas erfahrenere User ausgelegt, die schon einmal mit
        Linux gearbeitet haben. Für eine Distribution mit festen Releasezyklen ist die
        installierte Software immer sehr neu und fügt sich gut in das Gesamtsystem
        ein.}
    {Heiko Carrasco}
    \vspace{3em}
    \subartikelohnevorspann{Arch Linux}
    {Arch Linux ist im Gegensatz zu Fedora eine Rolling-Release Distribution -- es
        gibt keine festen Releasezyklen, neue Software kommt in die Paketquellen, sobald
        sie fertig ist. Da es außerdem nur eine sehr kurze Testphase gibt, kommt es
        daher ab und zu vor, dass installierte Software Bugs enthält, welche es bei
        anderen Distributionen nie in die Paketquellen geschafft hätten. Dafür bekommt
        man immer die aktuellste Software.\\
        Auch bei Arch Linux gibt es eine Lösung für den Fall, dass Software nicht in den
        offiziellen Paketquellen verfügbar ist: das Arch User Repository (AUR).
        Im AUR kann jeder \texttt{PKGBUILD} Dateien veröffentlichen, mit denen
        automatisch die Sourcecodes von Programmen heruntergeladen, kompiliert und in
        Pakete gepackt werden können. Durch die recht einfache Struktur der
        \texttt{PKGBILD} ist es leicht Pakete im AUR zu veröffentlichen, so findet man
        fast alles an Software, was nicht in den offiziellen Quellen enthalten ist im
        AUR.\\
        Arch kommt -- was recht ungewöhnlich ist -- ohne Installer oder grafische
        Konfigurationswerkzeuge. Um Arch zu
        installieren, muss man von einem Livesystem (oder einer bestehenden
        Linuxinstallation) die Partitionen anpassen, Pakete auf den gewünschten
        Partitionen installieren, wichtige Einstellungen setzen, einen Bootloader
        installieren und einrichten.
        Für all diese Schritte gibt es einen recht ausführlichen \enquote{Installation
            Guide}.\\
        Nach dieser Installation hat man nun aber noch keine grafische Oberfläche,
        sondern nur eine Konsole zur Verfügung, das heißt aber auch dass es keine
        Standard Desktopumgebung gibt, stattdessen hat man -- wenn man denn möchte --
        die Auswahl aus einer sehr langen Liste an Desktopumgebungen und Window
        Managern.\\
        Ihr werdet gemerkt haben, dass Arch Linux auch eher für erfahrene Nutzer*innen ausgelgt
        ist. Das Schöne an Arch ist allerdings, dass man nach der Installation besser
        nachvollziehen kann, wie die verschiedenen Komponenten des Systems
        ineinandergreifen.
        \\~\\
        Aufmerksame Leser*innen werden sich fragen, warum wir hier zum Einstieg direkt
        zwei Distributionen vorstellen, welche eher erfahrenere Nutzer*innen anspricht.
        Das Problem ist, wir haben niemanden der über Ubuntu oder ähnliche
        Distributionen schreiben wollte.
        Also haben wir einfach über die Distributionen geschrieben, die uns am meisten
        zusagen, und auch wenn der Anfang nicht sonderlich leicht sein dürfte kann man
        mit etwas Ausdauer und Interesse auch diese beiden Distributionen ganz ohne
        Linux-Vorkentnisse bedienen und nutzen.
    }
    {Fabian Franke}}

\vfill
\bildmitunterschrift{grafik/linux_xkcd}{width=\textwidth}{}{xkcd.com/456}
\vfill

\newpage
