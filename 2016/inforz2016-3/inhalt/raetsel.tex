\artikelohnevorspann{Auflösung: Auto trifft Gummibärchen }
{Wie euch sicher allen bewusst ist, haben wir uns in der letzten Ausgabe die
Frage gestellt, wie viele handelsübliche Gummibärchen in einen VW Käfer der
ersten Generation passen. Natürlich hat uns diese Frage keine Ruhe gelassen und
so wendeten wir uns hoffnungsvoll an die Mathematiker*innen und ließen sie mit
all ihren Formeln und Fachwissen rechnen, doch leider konnten sie uns keine
verbindlichen Aussagen machen. Wir suchten also eine Möglichkeit das Ergebnis
der Mathematiker*innen zu bestätigen, doch fanden wir keine Person, die dazu
fähig war. Am Ende blieb uns nur ein Ausweg, wir mussten es testen. Da wir
allerdings unser Testobjekt gerade verliehen hatten, ging dies auch nicht. Doch
wir wären keine Informatiker*innen, würden wir keine Lösung finden. \\

Nach einigen Mühen und Strapazen, hier nun das Ergebnis unserer Frage: 2203717 \\

Wir danken allen, die sich mit dieser Frage beschäftigt haben und uns ihre
Lösungen zu geschickt haben. Die Gewinner*innen werden wir persönlich benachrichtigen. Hier für euch die Schätzungen, die am nächsten an unserem Ergebnis waren: xxxx,yyyy,zzzz.}
{Heiko Carassco \& Jannis Blüml}

\artikel{Ein kleines Rätsel schadet nie \\
{\small Das Problem mit dem Alter...}}
{Sei es Gollum oder eine Spynx in einem Labyrinth, Rätsel sind und bleiben eine beliebte Aufgabe in so manchen Büchern und Filmen. Vielleicht liebe ich sie genau deswegen. Viele von euch, da wette ich drauf, haben sicher eine ähnliche Einstellung wie ich dazu. Hier deswegen ein kleines Rätsel für die nächste Bahnfahrt oder das nächste Wartezimmer.}
{
Treffen sich zwei alte Freunde, nennen wir sie Alice und Bob, Jahre später wieder und kommen ins Gespräch.
Und glauben wir an das Infinite-Monkey-Theorem, dann wissen wir, dass die folgende Situation das ein oder andere Mal auftreten wird oder aufgetreten ist... \\

Alice: \textit{\enquote{[...] ach du hast drei Kinder, ich freue mich für
dich.}} \\

Alice: \textit{\enquote{Wie alt sind sie denn?}} \\

Da Bob diese Frage schon so oft gehört hat und weiß wie sehr Alice Rätsel
liebt, antwortet er wie folgt: \\

\textit{\enquote{Wenn man das Alter der Dreien addiert, kommt man auf die Zahl
13.}} \\

Alice: \textit{\enquote{Das reicht mir nicht, kannst du mir nicht noch einen
Tipp geben?}} \\

Bob: \textit{\enquote{Aber nur, weil du es bist. Wenn man das Alter der Dreien
multipliziert, dann ergibt das eine Quadratzahl.}} \\

Doch auch das scheint Alice nicht zu reichen. Sie geht auf und ab und überlegt, bis sie letztendlich einen dritten Tipp verlangt. \\

Bob: \textit{\enquote{Von mir aus, einen letzten Tipp sollst du bekommen. Mein
Jüngster ist mindestens 1 Jahr jünger als seine Geschwister.}} \\

Alice lächelt und kennt das Ergebnis, kennst du es auch?
}{Jannis Blüml}
