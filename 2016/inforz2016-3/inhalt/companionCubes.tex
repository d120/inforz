\artikelohnevorspann{Companion Cubes zum Selberbacken}
{\paragraph{Zutaten für den hellen Teig:}
    \begin{itemize}
        \item 300g Mehl
        \item 200g Butter
        \item 100g Puderzucker
        \item 1 Ei
        \item nach Belieben noch etwas Vanille
    \end{itemize}
    \paragraph{für den dunklen Teig:}
    \begin{itemize}
        \item 250g Mehl
        \item 50g Kakaopulver (Backkakao, kein Kaba!)
        \item 200g Butter
        \item 100g Puderzucker
        \item 1 Ei
    \end{itemize}
    \paragraph{außerdem}
    \begin{itemize}
        \item 100g Bonbons (rot oder rosa)
    \end{itemize}
    \bildmitunterschrift{grafik/companion_cookies}{width=\columnwidth}{}{}
    \vspace*{\fill}
    \columnbreak
    Jeweils die Zutaten für die beiden Teige zu einem festen Mürbeteig verkneten
    und zum Ruhen in den Kühlschrank legen.
    \\~\\
    Forme aus etwa der Hälfte des hellen Teigs eine Rolle, die dick genug ist, um
    später aus einer Scheibe davon ein Herz auszustechen oder auszuschneiden.

    Rolle dann den dunklen Teig aus, umwickle die Rolle damit und schneide die
    überschüssigen Reste ab.

    Forme aus dem restlichen hellen Teig vier dicke und
    vier dünne Streifen, bringe sie in gleichmäßigen Abständen auf der Rolle an und
    forme die dicken Streifen zu Ecken.
    \\~\\
    Lege das ganze für 20-30 Minuten ins Gefrierfach.
    \\~\\
    Nun wird es Zeit die Bonbons zu zerkleinern. Das kannst du entweder in einem
    Gefrierbeutel mit einem Hammer machen (back dann vielleicht ein paar mehr, um
    genervte Nachbarn zu besänftigen) oder mit einer Küchenmaschine o.ä.
    \\~\\
    Wenn der Teig sich fest anfühlt, schneide ihn mit einem großen Messer in
    schmale Scheiben.

    Lege die Scheiben auf ein mit Backpapier ausgelegtes Blech
    und steche in der Mitte Herzen aus.

    Fülle die Löcher mit einem Löffel voll
    deiner zerkleinerten Bonbons.
    \\~\\
    Schiebe nun das ganze bei 180\textdegree{}C Umluft oder 200\textdegree{}C Ober/Unterhitze in den
    Backofen, bis der helle Teil der Kekse leicht braun wird (das sollte je nach
    Ofen ca. 12 Minuten dauern).

    Lass sie dann auf jeden Fall noch einen Moment auf
    dem Backpapier auskühlen, bis die Zuckermasse fest geworden ist!
    \\~\\
    Guten Appetit und viel Spaß mit deinen vielen kleinen Companion Cubes.
}
{Simone Schlarhorst}
